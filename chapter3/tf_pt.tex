ในการทำปัญญาประดิษฐ์จะมีการใช้ library ที่เกี่ยวกับการเรียนรู้เชิงลึกที่ส่วนใหญ่นิยมนำมาใช้มีอยู่ 2 library  คือ tensorflow และ pytorch เพราะเนื่องจาก library เป็นแบบที่สามารถดาวน์โหลดแหล่งที่มาของโค็ดได้ ซึ่งเป็นการให้สิทธิเสรีแก่ผู้ที่จะนำไปใช้ และการใช้งานที่ง่ายต่อผู้ที่ทำงานเกี่ยวกับด้านปัญญาประดิษฐ์ โดยที่ความแต่ต่างของ tensorflow และ pytorch มีดังนี้

\begin{table}[!ht]
	\centering
	\begin{tabular}{|c|c|c|}
		\hline
		{ตวามแตกต่าง}&\multicolumn{2}{c|}{library}\\
		\cline{2-3}
		{}								& tensorflow					& pytorch				\\
		\hline
		framework							& theano						& torch				\\
		\hline
		ผู้พัฒนา							& google						& facebook			\\
		\hline
		graph ที่ได้							& static graph					& dynamic graph		\\
		\hline
		ด้านการใช้งาน						& ยากกว่า pytorch				& ง่ายกว่า tensorflow		\\
		\hline
		จำนวนแหล่งข้อมูล					& มากกว่า pytorch				& น้อยกว่า tensorflow		\\
		\hline
		ตัวช่วยแสดงผลของโมเดลปัญญาประดิษฐ์		& tensorboard					& ไม่มี				\\
		\hline
	\end{tabular}
	\caption{ความแตกต่างของ tensorflow และ pytorch}
	\label{tab: ความแตกต่างของ tensorflow และ pytorch}
\end{table}

จากตาราง \ref{tab: ความแตกต่างของ tensorflow และ pytorch}  static graph คือ graph ที่ได้หลักจากผ่านการคำนวณทั้งหมดของโมเดลปัญญาประดิษฐ์ ซึ่งไม่สามารถแก้ไขได้ แต่ dynamic graph นั้นสามารถที่จะปรับเปลี่ยนได้ตลอดเวลา ซึ่งทำให้สามารถนำมาปรับใช้กับข้อมูลหลากหลายแบบได้
