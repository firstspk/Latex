ในการทำปัญญาประดิษฐ์จะมีการใช้ library ที่เกี่ยวกับการเรียนรู้เชิงลึกที่ส่วนใหญ่นิยมนำมาใช้มีอยู่ 2 library คือ Tensorflow และ Pytorch เพราะเนื่องจาก library 
เป็นแบบที่สามารถดาวน์โหลดแหล่งที่มาของโค็ดได้ ซึ่งเป็นการให้สิทธิเสรีแก่ผู้ที่จะนำไปใช้ และการใช้งานที่ง่ายต่อผู้ที่ทำงานเกี่ยวกับด้านปัญญาประดิษฐ์ โดยที่ความแต่ต่างของ Tensorflow 
และ Pytorch มีดังนี้

\begin{table}[!ht]
	\centering
	\begin{tabular}{|c|c|c|}
		\hline
		{ตวามแตกต่าง}&\multicolumn{2}{c|}{library}\\
		\cline{2-3}
		{}								& Tensorflow					& Pytorch				\\
		\hline
		framework							& theano						& torch				\\
		\hline
		ผู้พัฒนา							& google						& facebook			\\
		\hline
		graph ที่ได้							& static graph					& dynamic graph		\\
		\hline
		ด้านการใช้งาน						& ยากกว่า Pytorch				& ง่ายกว่า Tensorflow		\\
		\hline
		จำนวนแหล่งข้อมูล					& มากกว่า Pytorch				& น้อยกว่า Tensorflow		\\
		\hline
		ตัวช่วยแสดงผลของโมเดลปัญญาประดิษฐ์		& tensorboard					& ไม่มี				\\
		\hline
	\end{tabular}
	\caption{ความแตกต่างของ Tensorflow และ Pytorch}
	\label{tab: ความแตกต่างของ Tensorflow และ Pytorch}
\end{table}

จากตาราง \ref{tab: ความแตกต่างของ Tensorflow และ Pytorch} static graph คือ graph ที่ได้หลักจากผ่านการคำนวณทั้งหมดของโมเดลปัญญาประดิษฐ์ 
ซึ่งไม่สามารถแก้ไขได้ แต่ dynamic graph นั้นสามารถที่จะปรับเปลี่ยนได้ตลอดเวลา ซึ่งทำให้สามารถนำมาปรับใช้กับข้อมูลหลากหลายแบบได้
