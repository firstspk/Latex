\subsection*{สิ่งที่ใช้ในการวัดผล}
	\begin{enumerate}
		\item ความเร็วต่อรูปภาพ (วินาที)
		\item ความแม่นยำของกรอบสี่เหลี่ยม (IOU)
	\end{enumerate}
\subsection*{จุดประสงค์}
	\begin{enumerate}
		\item ผู้วิจัยได้ตั้งจุดประสงค์การทดลองการใช้โมเดลปัญญาประดิษฐ์สำหรับตรวจจับวัตถุ เพื่อวัดผลโมเดลปัญญาประดิษฐ์ที่ใช้ในปัจจุบัน
และ หาโมเดลปัญญาประดิษฐ์สำหรับการตรวจจับวัตถุที่มีความเร็วมากที่สุด และ แม่นยำสูงที่สุด เมื่อทดสอบกับชุดข้อมูลองผู้วิจัย
	\end{enumerate}
\subsection*{ตัวแปร}
	\begin{enumerate}
		\item โมเดลปัญญาประดิษฐ์ ซึ่งได้แก่
		\begin{enumerate}
			\item SSD Mobilenet v1 ppn
			\item YOLOv3-tiny
			\item YOLOv3-spp	
			\item YOLOv3-320
			\item Faster rcnn inception v2
		\end{enumerate}
	\end{enumerate}
\subsection*{ตัวแปรควบคุม}
	\begin{enumerate}
		\item ชุดข้อมูล : ชุดข้อมูลสำหรับทดสอบวัดผลที่ผู้วิจัยสร้างขึ้น (สุ่ม 20 เฟรมจากวิดีโอที่ผู้วิจัยใช้สำหรับสร้างชุดข้อมูล)
	\end{enumerate}
\subsection*{วิธีการทดลอง}
	\begin{enumerate}
		\item แบ่งชุดข้อมูลออกเป็น ชุดข้อมูลสำหรับทดสอบ และ ชุดข้อมูลที่มีคำตอบ
			\begin{enumerate}
				\item ชุดข้อมูลสำหรับทดสอบ ประกอบด้วย : ชื่อของวิดีโอ,เฟรม
				\item ชุดข้อมูลที่มีคำตอบ ประกอบด้วย : ชื่อของวิดีโอ,เฟรม,ตำแหน่งของกรอบสี่เหลี่ยม
			\end{enumerate}
		\item เรียกชื่อและเฟรมของวิดีโอจากชุดข้อมูลทดสอบ และนำโมเดลปัญญาประดิษฐ์ทำนายผลลัพธ์ จากนั้นเก็บผลลัพธ์เป็น ชุดข้อมูลผลลัพธ์จากการทำนาย
			\begin{enumerate}
				\item ชุดข้อมูลผลลัพธ์จากการทำนาย ประกอบด้วย : ชื่อของวิดีโอ,เฟรม,ตำแหน่งของกรอบสี่เหลี่ยม
			\end{enumerate}
		\item ประเมินผลการทำงานโดยเทียบระหว่างชุดผลลัพธ์จากการทำนาย และ ชุดข้อมูลที่มีคำตอบ ผ่านฟังก์ชั่นคำนวณค่า IOU		
		\item เปรียบเทียบผลลัพธ์จากแหล่งที่มา
\end{enumerate}
