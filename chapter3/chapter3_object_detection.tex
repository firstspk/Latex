\subsection*{สิ่งที่ใช้ในการวัดผล}
	\begin{enumerate}
		\item ความเร็วต่อรูปภาพ (วินาที)
		\item ความแม่นยำของกรอบสี่เหลี่ยม (IOU)
	\end{enumerate}
\subsection*{สมมุติฐาน :}
\subsection*{ตัวแปร}
	\begin{enumerate}
		\item โมเดลปัญญาประดิษฐ์ ซึ่งได้แก่
		\begin{enumerate}
			\item Tiny YOLO
			\item YOLOv3-tiny
			\item SSD300	
			\item YOLOv3-320
			\item YOLOv2 608x608
		\end{enumerate}
	\end{enumerate}
\subsection*{ตัวแปรควบคุม}
	\begin{enumerate}
		\item ชุดข้อมูล : The validation split of AVA v2.1
	\end{enumerate}
\subsection*{วิธีการทดลอง}
	\begin{enumerate}
		\item ดาว์นโหลดชุดข้อมูล The validation split of AVA v2.1
		\item แบ่งชุดข้อมูลออกเป็น ชุดข้อมูลสำหรับทดสอบ และ ชุดข้อมูลที่มีคำตอบ
			\begin{enumerate}
				\item ชุดข้อมูลสำหรับทดสอบ ประกอบด้วย : ชื่อของวิดีโอ,เฟรม
				\item ชุดข้อมูลที่มีคำตอบ ประกอบด้วย : ชื่อของวิดีโอ,เฟรม,ตำแหน่งของกรอบสี่เหลี่ยม
			\end{enumerate}
		\item เรียกชื่อและเฟรมของวิดีโอจากชุดข้อมูลทดสอบ และนำโมเดลปัญญาประดิษฐ์ทำนายผลลัพธ์ จากนั้นเก็บผลลัพธ์เป็น ชุดข้อมูลผลลัพธ์จากการทำนาย
			\begin{enumerate}
				\item ชุดข้อมูลผลลัพธ์จากการทำนาย ประกอบด้วย : ชื่อของวิดีโอ,เฟรม,ตำแหน่งของกรอบสี่เหลี่ยม
			\end{enumerate}
		\item ประเมินผลการทำงานโดยเทียบระหว่างชุดผลลัพธ์จากการทำนาย และ ชุดข้อมูลที่มีคำตอบ ผ่านฟังก์ชั่นคำนวณค่า IOU		
		\item เปรียบเทียบผลลัพธ์จากแหล่งที่มา
\end{enumerate}
