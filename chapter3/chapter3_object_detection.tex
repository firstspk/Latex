\subsection*{สิ่งที่ใช้ในการวัดผล}
	\begin{enumerate}
		\setlength\itemsep{-0.25em}
		\item ความเร็วในการทำนายต่อรูปภาพ (มิลลิวินาที)
		\item ความแม่นยำ โดยคำนึงถึง IoU
	\end{enumerate}
\subsection*{จุดประสงค์}
	\begin{enumerate}
		\setlength\itemsep{-0.25em}
		\item ผู้วิจัยได้ตั้งจุดประสงค์การทดลองการใช้โมเดลปัญญาประดิษฐ์สำหรับตรวจจับวัตถุ เพื่อวัดผลโมเดลปัญญาประดิษฐ์ที่ใช้ในปัจจุบัน
และหาโมเดลปัญญาประดิษฐ์สำหรับการตรวจจับวัตถุที่มีความเร็วมากที่สุดและแม่นยำสูงที่สุดเมื่อทดสอบกับชุดข้อมูลของผู้วิจัย
	\end{enumerate}
\subsection*{ตัวแปร}
	\begin{enumerate}
		\setlength\itemsep{-0.25em}
		\item โมเดลปัญญาประดิษฐ์ ได้แก่
		\begin{enumerate}
			\setlength\itemsep{-0.25em}
			\item SSD Mobilenet v1 ppn
			\item YOLO-v3 tiny
			\item YOLO-v3 spp	
			\item YOLO-v3 320
			\item Faster RCNN inception v2
		\end{enumerate}
	\end{enumerate}
\subsection*{ตัวแปรควบคุม}
	\begin{enumerate}
		\setlength\itemsep{-0.25em}
		\item ชุดข้อมูล : ชุดข้อมูลสำหรับทดสอบวัดผลที่ผู้วิจัยสร้างขึ้น (สุ่ม 20 เฟรมจากวิดีโอที่ผู้วิจัยใช้สำหรับสร้างชุดข้อมูล)
	\end{enumerate}
\subsection*{วิธีการทดลอง}
	\begin{enumerate}
		\setlength\itemsep{-0.25em}
		\item แบ่งชุดข้อมูลออกเป็นชุดข้อมูลสำหรับทดสอบ และชุดข้อมูลที่มีคำกำกับเพื่อใช้สำหรับวัดผล
			\begin{enumerate}
				\setlength\itemsep{-0.25em}
				\item ชุดข้อมูลสำหรับทดสอบ ประกอบด้วย : ชื่อของวิดีโอ และเฟรม
				\item ชุดข้อมูลที่มีคำกำกับเพื่อใช้สำหรับวัดผล ประกอบด้วย : ชื่อของวิดีโอ เฟรม และตำแหน่งของกรอบสี่เหลี่ยม
			\end{enumerate}
		\item เรียกชื่อและเฟรมของวิดีโอจากชุดข้อมูลทดสอบ และนำโมเดลปัญญาประดิษฐ์ทำนายผลลัพธ์ จากนั้นเก็บผลลัพธ์เป็นชุดข้อมูลผลลัพธ์จากการทำนาย
			\begin{enumerate}
				\setlength\itemsep{-0.25em}
				\item ชุดข้อมูลผลลัพธ์จากการทำนาย ประกอบด้วย : ชื่อของวิดีโอ เฟรม และตำแหน่งของกรอบสี่เหลี่ยม
			\end{enumerate}
		\item ประเมินผลค่าความแม่นยำในการทำงานโดยเทียบระหว่างชุดผลลัพธ์จากการทำนาย และชุดข้อมูลที่มีคำกำกับเพื่อใช้สำหรับวัดผล โดยกำหนดให้ค่า IoU มากกว่าหรือเท่ากับ 0.5	
		จึงจะนับว่าทำนายได้ถูก
		\item เปรียบเทียบผลลัพธ์จากแหล่งที่มา
\end{enumerate}
