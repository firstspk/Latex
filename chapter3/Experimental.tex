\subsection*{สิ่งที่ใช้ในการวัดผล}
	\begin{enumerate}
		\item ความเร็วต่อรูปภาพ (วินาที)
		\item ความแม่นยำ (PASCAL mAP)
	\end{enumerate}
\subsection*{สมมุติฐาน :}ผู้วิจัยได้ตั้งสมมุติฐานว่า ผลลัพธ์ของการทดลองจะมีความแม่นยำเทียบเท่ากับผลลัพธ์จากแหล่งที่มา แต่ความเร็วต่อรูปภาพจะมีความเร็วน้อยกว่าผลลัพธ์จากแหล่งที่มา เนื่องจาก แหล่งที่มาของข้อมูลได้ทำการทดสอบโดยใช้กราฟิกการ์ดรุ่น Nvidia GeForce GTX TITAN X card ซึ่งเป็นกราฟิกการ์ดที่มีประสิทธิภาพการทำงานดีกว่า กราฟิกการ์ดของผู้วิจัย จึงทำให้สามารถทดสอบด้วยความเร็วที่มากกว่า
\subsection*{ตัวแปรควบคุม}
	\begin{enumerate}
		\item ชุดข้อมูล : The validation split of AVA v2.1
		\item Machine learning model : Faster rcnn resnet101 ava v2.1
	\end{enumerate}
\subsection*{วิธีการทดลอง}
	\begin{enumerate}
		\item ดาว์นโหลดชุดข้อมูล The validation split of AVA v2.1
		\item แบ่งชุดข้อมูลออกเป็น ชุดข้อมูลสำหรับทดสอบ และ ชุดข้อมูลที่มีคำตอบ
			\begin{enumerate}
				\item ชุดข้อมูลสำหรับทดสอบ ประกอบด้วย : ชื่อของวิดีโอ
				\item ชุดข้อมูลที่มีคำตอบ ประกอบด้วย : ชื่อของวิดีโอ,เฟรม,ตำแหน่งของกรอบสี่เหลี่ยม,ไอดีของการกระทำ
			\end{enumerate}
		\item เรียกชื่อของวิดีโอจากชุดข้อมูลทดสอบ และนำโมเดลปัญญาประดิษฐ์ทำนายผลลัพธ์ จากนั้นเก็บผลลัพธ์เป็นชุดข้อมูลผลลัพธ์จากการทำนาย
			\begin{enumerate}
				\item ชุดข้อมูลผลลัพธ์จากการทำนาย ประกอบด้วย : ชื่อของวิดีโอ,เฟรม,ตำแหน่งของกรอบสี่เหลี่ยม,ไอดีของการกระทำ,ความมั่นใจ
			\end{enumerate}
		\item ประเมินผลการทำงานโดยเทียบระหว่างชุดผลลัพธ์จากการทำนาย และ ชุดข้อมูลที่มีคำตอบ
		\item เปรียบเทียบผลลัพธ์จากแหล่งที่มา
\end{enumerate}
\clearpage
\subsection{ทดสอบประสิทธิ์ภาพการทำงานของโมเดลปัญญาประดิษฐ์ที่เคยถูกเทรนด์ผ่าน AVA และ ใช้ชุดข้อมูลที่ผู้วิจัยสร้างขึ้น ในการทดสอบและเทียบผลลัพธ์กับแหล่งอ้างอิง}
\subsection*{สิ่งที่ใช้ในการวัดผล}
	\begin{enumerate}
		\item ความเร็วต่อรูปภาพ (วินาที)
		\item ความแม่นยำ (PASCAL mAP)
	\end{enumerate}
\subsection*{สมมุติฐาน :}ผู้วิจัยได้ตั้งสมมุติฐานว่าผลลัพธ์ของการทดลองจะมีความแม่นยำต่ำลงเมื่อเทียบกับความแม่นยำของการทดลองที่ผ่านมา เนื่องจาก ชุดข้อมูลที่ผู้วิจัยสร้างขึ้น ได้มีการตัดหมวดหมู่บางอย่างออกไป ทำให้โมเดลปัญญาประดิษฐ์ที่ถูกเทรนด์ผ่าน AVA มีหมวดหมู่ของการกระทำไม่ตรงกับชุดข้อมูลที่ผู้วิจัยสร้างขึ้น ซึ่งมีผลทำให้ความแม่นยำลดลง ในส่วนของความเร็วต่อรูปภาพจะมีความเร็วน้อยกว่าผลลัพธ์จากแหล่งที่มา เนื่องจาก แหล่งที่มาของข้อมูลได้ทำการทดสอบโดยใช้กราฟิกการ์ดรุ่น Nvidia GeForce GTX TITAN X card ซึ่งเป็นกราฟิกการ์ดที่มีประสิทธิภาพการทำงานดีกว่า กราฟิกการ์ดของผู้วิจัย จึงทำให้สามารถทดสอบด้วยความเร็วที่มากกว่า
\subsection*{ตัวแปรควบคุม}
	\begin{enumerate}
		\item ชุดข้อมูล : ชุดข้อมูลที่ผู้วิจัยสร้าง ถูกสร้างด้วย A.I. assisted labeling tool
		\item Machine learning model : Faster rcnn resnet101 ava v2.1
	\end{enumerate}
\subsection*{วิธีการทดลอง}
	\begin{enumerate}
		\item แบ่งชุดข้อมูลออกเป็น ชุดข้อมูลสำหรับทดสอบ และ ชุดข้อมูลที่มีคำตอบ
			\begin{enumerate}
				\item ชุดข้อมูลสำหรับทดสอบ ประกอบด้วย : ชื่อของวิดีโอ
				\item ชุดข้อมูลที่มีคำตอบ ประกอบด้วย : ชื่อของวิดีโอ,เฟรม,ตำแหน่งของกรอบสี่เหลี่ยม,ไอดีของการกระทำ
			\end{enumerate}
		\item เรียกชื่อของวิดีโอจากชุดข้อมูลทดสอบ และนำโมเดลปัญญาประดิษฐ์ทำนายผลลัพธ์ จากนั้นเก็บผลลัพธ์เป็นชุดข้อมูลผลลัพธ์จากการทำนาย
			\begin{enumerate}
				\item ชุดข้อมูลผลลัพธ์จากการทำนาย ประกอบด้วย : ชื่อของวิดีโอ,เฟรม,ตำแหน่งของกรอบสี่เหลี่ยม,ไอดีของการกระทำ,ความมั่นใจ
			\end{enumerate}
		\item ประเมินผลการทำงานโดยเทียบระหว่างชุดผลลัพธ์จากการทำนาย และ ชุดข้อมูลที่มีคำตอบ	
		\item เปรียบเทียบผลลัพธ์กับผลการทดลองที่ผ่านมา
\end{enumerate}
\clearpage
\subsection{ทดสอบประสิทธิ์ภาพการทำงานของโมเดลปัญญาประดิษฐ์ที่เคยถูกเทรนด์ผ่านชุดข้อมูลสำหรับการเทรนด์ที่ผู้วิจัยสร้างขึ้น และ ใช้ชุดข้อมูลที่ผู้วิจัยสร้างขึ้น ในการทดสอบและเทียบผลลัพธ์กับแหล่งอ้างอิง}
\subsection*{สิ่งที่ใช้ในการวัดผล}
	\begin{enumerate}
		\item ความเร็วต่อรูปภาพ (วินาที)
		\item ความแม่นยำ (PASCAL mAP)
	\end{enumerate}
\subsection*{สมมุติฐาน :}ผู้วิจัยได้ตั้งสมมุติฐานว่าผลลัพธ์ของการทดลองจะมีความแม่นยำสูงขึ้นเมื่อเทียบกับความแม่นยำของการทดลองที่ผ่านมา เนื่องจาก โมเดลปัญญาประดิษฐ์ในการทดลองนี้ เป็นโมเดลปัญญาประดิษฐ์ที่ผู้วิจัยได้สร้างขึ้น ซึ่งจะมีหมวดหมู่ของการกระทำของโมเดลปัญญาประดิษฐ์และชุดข้อมูลทดสอบตรงกัน ในส่วนของความเร็วต่อรูปภาพจะมีความเร็วน้อยกว่าผลลัพธ์จากแหล่งที่มา เนื่องจาก แหล่งที่มาของข้อมูลได้ทำการทดสอบโดยใช้กราฟิกการ์ดรุ่น Nvidia GeForce GTX TITAN X card ซึ่งเป็นกราฟิกการ์ดที่มีประสิทธิภาพการทำงานดีกว่า กราฟิกการ์ดของผู้วิจัย จึงทำให้สามารถทดสอบด้วยความเร็วที่มากกว่า
\subsection*{ตัวแปรควบคุม}
	\begin{enumerate}
		\item ชุดข้อมูล : ชุดข้อมูลที่ผู้วิจัยสร้าง ถูกสร้างด้วย A.I. assisted labeling tool
		\item Machine learning model : Faster rcnn resnet101 ava v2.1
	\end{enumerate}
\subsection*{วิธีการทดลอง}
	\begin{enumerate}
		\item แบ่งชุดข้อมูลออกเป็น ชุดข้อมูลสำหรับทดสอบ และ ชุดข้อมูลที่มีคำตอบ
			\begin{enumerate}
				\item ชุดข้อมูลสำหรับทดสอบ ประกอบด้วย : ชื่อของวิดีโอ
				\item ชุดข้อมูลที่มีคำตอบ ประกอบด้วย : ชื่อของวิดีโอ,เฟรม,ตำแหน่งของกรอบสี่เหลี่ยม,ไอดีของการกระทำ
			\end{enumerate}
		\item เรียกชื่อของวิดีโอจากชุดข้อมูลทดสอบ และนำโมเดลปัญญาประดิษฐ์ทำนายผลลัพธ์ จากนั้นเก็บผลลัพธ์เป็นชุดข้อมูลผลลัพธ์จากการทำนาย
			\begin{enumerate}
				\item ชุดข้อมูลผลลัพธ์จากการทำนาย ประกอบด้วย : ชื่อของวิดีโอ,เฟรม,ตำแหน่งของกรอบสี่เหลี่ยม,ไอดีของการกระทำ,ความมั่นใจ
			\end{enumerate}
		\item ประเมินผลการทำงานโดยเทียบระหว่างชุดผลลัพธ์จากการทำนาย และ ชุดข้อมูลที่มีคำตอบ	
		\item เปรียบเทียบผลลัพธ์กับผลการทดลองที่ผ่านมา
\end{enumerate}





