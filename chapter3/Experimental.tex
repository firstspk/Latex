\subsection*{สิ่งที่ใช้ในการวัดผล}
	\begin{enumerate}
		\setlength\itemsep{-0.25em}
		\item ความเร็วในการทำนายต่อรูปภาพ (มิลลิวินาที)
		\item ความแม่นยำ (PASCAL mAP)
	\end{enumerate}
\subsection*{สมมติฐาน}
ผู้วิจัยได้ตั้งสมมติฐานว่า ผลลัพธ์ของการทดลองจะมีความแม่นยำเทียบเท่ากับผลลัพธ์จากแหล่งที่มา และความเร็วต่อรูปภาพจะมีความเร็วมากกว่ากว่าผลลัพธ์จากแหล่งที่มา 
เนื่องจากแหล่งที่มาของข้อมูลได้ทำการทดสอบโดยใช้กราฟิกการ์ดรุ่น Nvidia GeForce GTX TITAN X ซึ่งเป็นกราฟิกการ์ดที่มีประสิทธิภาพการทำงานดีน้อยกว่ากราฟิกการ์ดของผู้วิจัย
\subsection*{ตัวแปรควบคุม}
	\begin{enumerate}
		\setlength\itemsep{-0.25em}
		\item ชุดข้อมูล : The validation split of AVA v2.1
		\item โมเดลปัญญาประดิษฐ์ : Faster RCNN ResNet101 AVA v2.1
	\end{enumerate}
\subsection*{วิธีการทดลอง}
	\begin{enumerate}
		\setlength\itemsep{-0.25em}
		\item ดาวน์โหลดชุดข้อมูล The validation split of AVA v2.1
		\item แบ่งชุดข้อมูลออกเป็น ชุดข้อมูลสำหรับทดสอบ และชุดข้อมูลที่มีคำกำกับเพื่อเป็นคำตอบ
			\begin{enumerate}
				\setlength\itemsep{-0.25em}
				\item ชุดข้อมูลสำหรับทดสอบ ประกอบด้วย : ชื่อของวิดีโอ
				\item ชุดข้อมูลที่มีคำกำกับเพื่อเป็นคำตอบ ประกอบด้วย : ชื่อของวิดีโอ เฟรม ตำแหน่งของกรอบสี่เหลี่ยม และรหัสของการกระทำ
			\end{enumerate}
		\item เรียกชื่อของวิดีโอจากชุดข้อมูลทดสอบ และนำโมเดลปัญญาประดิษฐ์ทำนายผลลัพธ์ จากนั้นเก็บผลลัพธ์เป็นชุดข้อมูลผลลัพธ์จากการทำนาย
			\begin{enumerate}
				\setlength\itemsep{-0.25em}
				\item ชุดข้อมูลผลลัพธ์จากการทำนาย ประกอบด้วย : ชื่อของวิดีโอ เฟรม ตำแหน่งของกรอบสี่เหลี่ยม รหัสของการกระทำ และความมั่นใจ
			\end{enumerate}
		\item ประเมินผลการทำงานโดยเทียบระหว่างชุดผลลัพธ์จากการทำนาย และชุดข้อมูลที่มีคำกำกับเพื่อเป็นคำตอบ
		\item เปรียบเทียบผลลัพธ์จากแหล่งที่มา
\end{enumerate}
\clearpage
\subsection{ทดสอบประสิทธิภาพการทำงานของโมเดลปัญญาประดิษฐ์ Faster RCNN ResNet101 ที่เคยถูกสร้างด้วยชุดข้อมูลของ AVA และใช้ชุดข้อมูลที่ผู้วิจัยสร้างขึ้นในการทดสอบและเทียบผลลัพธ์กับแหล่งอ้างอิง}
\subsection*{สิ่งที่ใช้ในการวัดผล}
	\begin{enumerate}
		\setlength\itemsep{-0.25em}
		\item ความในการทำนายเร็วต่อรูปภาพ (มิลลิวินาที)
		\item ความแม่นยำ (PASCAL mAP)
	\end{enumerate}
\subsection*{สมมติฐาน}ผู้วิจัยได้ตั้งสมมติฐานว่าผลลัพธ์ของการทดลองจะมีความแม่นยำต่ำลงเมื่อเทียบกับความแม่นยำของการทดลองที่ผ่านมา เนื่องจากชุดข้อมูลที่ผู้วิจัยสร้างขึ้น ได้มีการตัดหมวดหมู่บางอย่างออกไป 
ทำให้โมเดลปัญญาประดิษฐ์ที่ถูกสร้างด้วย AVA มีหมวดหมู่ของการกระทำไม่ตรงกับชุดข้อมูลที่ผู้วิจัยสร้างขึ้น ซึ่งมีผลทำให้ความแม่นยำลดลง ในส่วนของความเร็วต่อรูปภาพจะมีความเร็วมากกว่าผลลัพธ์จากแหล่งที่มา เนื่องจาก แหล่งที่มาของข้อมูลได้ทำการทดสอบโดยใช้กราฟิกการ์ดรุ่น Nvidia GeForce GTX TITAN X card ซึ่งเป็นกราฟิกการ์ดที่มีประสิทธิภาพการทำงานดีน้อยกว่า กราฟิกการ์ดของผู้วิจัย
\subsection*{ตัวแปรควบคุม}
	\begin{enumerate}
		\setlength\itemsep{-0.25em}
		\item ชุดข้อมูล : ชุดข้อมูลที่ผู้วิจัยสร้างด้วยเครื่องมือกำกับคุณลักษณะ
		\item โมเดลปัญญาประดิษฐ์ : Faster RCNN ResNet101 AVA v2.1
	\end{enumerate}
\subsection*{วิธีการทดลอง}
	\begin{enumerate}
		\setlength\itemsep{-0.25em}
		\item แบ่งชุดข้อมูลออกเป็น ชุดข้อมูลสำหรับทดสอบ และชุดข้อมูลที่มีคำกำกับเพื่อเป็นคำตอบ
			\begin{enumerate}
				\setlength\itemsep{-0.25em}
				\item ชุดข้อมูลสำหรับทดสอบ ประกอบด้วย : ชื่อของวิดีโอ
				\item ชุดข้อมูลที่มีคำกำกับเพื่อเป็นคำตอบ ประกอบด้วย : ชื่อของวิดีโอ เฟรม ตำแหน่งของกรอบสี่เหลี่ยม และรหัสของการกระทำ
			\end{enumerate}
		\item เรียกชื่อของวิดีโอจากชุดข้อมูลทดสอบ และนำโมเดลปัญญาประดิษฐ์ทำนายผลลัพธ์ จากนั้นเก็บผลลัพธ์เป็นชุดข้อมูลผลลัพธ์จากการทำนาย
			\begin{enumerate}
				\setlength\itemsep{-0.25em}
				\item ชุดข้อมูลผลลัพธ์จากการทำนาย ประกอบด้วย : ชื่อของวิดีโอ เฟรม ตำแหน่งของกรอบสี่เหลี่ยม รหัสของการกระทำ และความมั่นใจ
			\end{enumerate}
		\item ประเมินผลการทำงานโดยเทียบระหว่างชุดผลลัพธ์จากการทำนาย และชุดข้อมูลที่มีคำกำกับเพื่อเป็นคำตอบ	
		\item เปรียบเทียบผลลัพธ์กับผลการทดลองที่ผ่านมา
\end{enumerate}
\clearpage
\subsection{ทดสอบประสิทธิภาพการทำงานของโมเดลปัญญาประดิษฐ์ ResNet50 ที่ถูกสร้างด้วยชุดข้อมูลที่ผู้วิจัยสร้างขึ้น และใช้ชุดข้อมูลที่ผู้วิจัยสร้างขึ้นในการทดสอบและเทียบผลลัพธ์กับแหล่งอ้างอิง}
\subsection*{สิ่งที่ใช้ในการวัดผล}
	\begin{enumerate}
		\setlength\itemsep{-0.25em}
		\item ความเร็วในการทำนายต่อรูปภาพ (มิลลิวินาที)
		\item ความแม่นยำ (PASCAL mAP)
	\end{enumerate}
\subsection*{สมมติฐาน}ผู้วิจัยได้ตั้งสมมติฐานว่าผลลัพธ์ของการทดลองจะมีความแม่นยำสูงขึ้นเมื่อเทียบกับความแม่นยำของการทดลองที่ผ่านมา เนื่องจากโมเดลปัญญาประดิษฐ์ในการทดลองนี้ 
เป็นโมเดลปัญญาประดิษฐ์ที่ผู้วิจัยได้สร้างขึ้น ซึ่งจะมีหมวดหมู่ของการกระทำของโมเดลปัญญาประดิษฐ์และชุดข้อมูลทดสอบตรงกัน ในส่วนของความเร็วต่อรูปภาพจะมีความเร็วมากกว่าผลลัพธ์จากแหล่งที่มา 
เนื่องจากแหล่งที่มาของข้อมูลได้ทำการทดสอบโดยใช้กราฟิกการ์ดรุ่น Nvidia GeForce GTX TITAN X ซึ่งเป็นกราฟิกการ์ดที่มีประสิทธิภาพการทำงานดีน้อยกว่ากราฟิกการ์ดของผู้วิจัย
\subsection*{ตัวแปรควบคุม}
	\begin{enumerate}
		\setlength\itemsep{-0.25em}
		\item ชุดข้อมูล : ชุดข้อมูลที่ผู้วิจัยสร้างด้วยเครื่องมือกำกับคุณลักษณะ
		\item โมเดลปัญญาประดิษฐ์ : โมเดลปัญญาประดิษฐ์ที่ผู้วิจัยสร้างขึ้น
	\end{enumerate}
\subsection*{วิธีการทดลอง}
	\begin{enumerate}
		\setlength\itemsep{-0.25em}
		\item แบ่งชุดข้อมูลออกเป็น ชุดข้อมูลที่ใช้สำหรับการสร้างโมเดลปัญญาประดิษฐ์ และชุดข้อมูลสำหรับทดสอบที่มีคำกำกับเพื่อเป็นคำตอบ โดยคำกำกับจะประกอบไปด้วย : ชื่อของวิดีโอ เฟรม ตำแหน่งของกรอบสี่เหลี่ยม และรหัสของการกระทำ
		\item ทำการสร้างโมเดลปัญญาประดิษฐ์ด้วยชุดข้อมูลที่เตรียมไว้ และนำโมเดลปัญญาประดิษฐ์ทำนายผลลัพธ์กับชุดข้อมูลทดสอบ จากนั้นเก็บผลลัพธ์จากการทำนาย  ประกอบด้วย : ชื่อของวิดีโอ เฟรม ตำแหน่งของกรอบสี่เหลี่ยม รหัสของการกระทำ และค่าความมั่นใจ โดยรายละเอียดของชุดข้อมูลทดสอบมีดังนี้
	\begin{table}[!ht]
                \centering
                \begin{tabular}{|c|c|}
                    \hline
                    Label & Test set\\
                    \hline
                    Play phone  & 3,888 \\
                    Eat  & 3,000\\
                    Sit   & 5,591\\
                    Sleep  & 2,458\\
                    Stand  & 1,402\\
                    Walk  & 2,162\\
                    \hline
                \end{tabular}
                \caption{ตารางแสดงจำนวนภาพที่ใช้ในการทดลองนี้}
                \label{tab: ResNet50_datasetInfo}
	\end{table}
		\item มีการปรับเปลี่ยนการเตรียมข้อมูลเช่นการใช้ pretrain โมเดลหรือการปรับขนาดสเกลของชุดข้อมูลที่ใช้ในการสร้างโมเดลปัญญาประดิษฐ์
		\item เปรียบเทียบผลลัพธ์ของโมเดลปัญญาประดิษฐ์ เพื่อเปรียบเทียบประสิทธิภาพ
\end{enumerate}





