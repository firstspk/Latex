\subsection*{สิ่งที่ใช้ในการวัดผล}
	\begin{enumerate}
		\item ความเร็วต่อรูปภาพ (วินาที)
		\item ความแม่นยำ (PASCAL mAP)
	\end{enumerate}
\subsection*{สมมุติฐาน :}ผู้วิจัยได้ตั้งสมมุติฐานว่า ผลลัพธ์ของการทดลองจะมีความแม่นยำเทียบเท่ากับผลลัพธ์จากแหล่งที่มา แต่ความเร็วต่อรูปภาพจะมีความเร็วน้อยกว่าผลลัพธ์จากแหล่งที่มา เนื่องจาก แหล่งที่มาของข้อมูลได้ทำการทดสอบโดยใช้กราฟิกการ์ดรุ่น Nvidia GeForce GTX TITAN X card ซึ่งเป็นกราฟิกการ์ดที่มีประสิทธิภาพการทำงานดีกว่า กราฟิกการ์ดของผู้วิจัย จึงทำให้สามารถทดสอบด้วยความเร็วที่มากกว่า
\subsection*{ตัวแปรควบคุม}
	\begin{enumerate}
		\item ชุดข้อมูล : The validation split of AVA v2.1
		\item Machine learning model : Faster rcnn resnet101 ava v2.1
	\end{enumerate}
\subsection*{วิธีการทดลอง}
	\begin{enumerate}
		\item ดาว์นโหลดชุดข้อมูล The validation split of AVA v2.1
		\item แบ่งชุดข้อมูลออกเป็น ชุดข้อมูลสำหรับทดสอบ และ ชุดข้อมูลที่มีคำตอบ
			\begin{enumerate}
				\item ชุดข้อมูลสำหรับทดสอบ ประกอบด้วย : ชื่อของวิดีโอ
				\item ชุดข้อมูลที่มีคำตอบ ประกอบด้วย : ชื่อของวิดีโอ,เฟรม,ตำแหน่งของกรอบสี่เหลี่ยม,ไอดีของการกระทำ
			\end{enumerate}
		\item เรียกชื่อของวิดีโอจากชุดข้อมูลทดสอบ และนำ Machine learning model ทำนายผลลัพธ์ จากนั้นเก็บผลลัพธ์เป็น ชุดข้อมูลผลลัพธ์จากการทำนาย
			\begin{enumerate}
				\item ชุดข้อมูลผลลัพธ์จากการทำนาย ประกอบด้วย : ชื่อของวิดีโอ,เฟรม,ตำแหน่งของกรอบสี่เหลี่ยม,ไอดีของการกระทำ,ความมั่นใจ
			\end{enumerate}
		\item ประเมินผลการทำงานโดยเทียบระหว่างชุดผลลัพธ์จากการทำนาย และ ชุดข้อมูลที่มีคำตอบ ผ่านฟังก์ชั่นจากแหล่งที่มา		
		\item เปรียบเทียบผลลัพธ์จากแหล่งที่มา
\end{enumerate}
