% ************************** Thesis Chapter3 **********************************
\chapter{ระเบียบวิธีวิจัย}
ในการทําโครงการวิจัยแอพพลิเคชั่นสำหรับวิเคราะห์วิดีโอ(video analytics) จะมีการทำงานหลากหลายส่วนมาทำงานร่วมกัน ซึ่งทำให้จำเป็นจะต้องมีระเบียบวิจัยสำหรับอธิบายภาพรวม
\subsection*{โดยในระเบียบวิจัยนี้จะมีหัวข้อ และระเบียบวิธีวิจัยดังนี้}
\begin{itemize}\setlength\itemsep{-0.3em}
	\item แผนการดำเนินงาน
	\item เครื่องมือที่ใช้ในการดำเนินงานวิจัย
	\item ภาพรวมของแอพพลิเคชั่น
	\item รายละเอียดของโมเดล
\end{itemize}
\vspace{3mm}
\section{หน้าที่ความรับผิดชอบ} 
\paragraph*{ปฐมพงศ์ สินธุ์งาม}
สร้างและทดสอบโมเดล I3D และ ทำโปรแกรมในส่วน tracking
\paragraph*{ศุภกร เบญจวิกรัย}
รวบรวมฟังก์ชั่นต่างๆของแอพพลิเคชั่น และ ทำแอพพลิเคชั่นในส่วน selection , detection
\paragraph*{อุกฤษฎ์ เลิศวรรณาการ}
สร้างและทดสอบโมเดล Resnet-50 และ ทำโปรแกรมในส่วน person reid 

\vspace{3mm}
\section{แผนการดำเนินงาน}
โดยจากที่กล่าวไปตอนต้นในบทนำ
การดำเนินงานและการออกแบบการสร้างหุ่นยนต์ฮิวมานอยด์ UTHAI มีแผนการทำงานซึ่งแบ่งออกเป็นสามส่วนดังนี้
ส่วนแรกคือ ส่วนของฮาร์ดแวร์ที่เกี่ยวกับโครงสร้างทางกลของหุ่นยนต์ฮิวมานอยด์ เช่น ข้อต่อ ก้านต่อ ฝ่าเท้า
รวมไปถึงระบบอิเล็กทรอนิกส์ ตัวประมวลผลการควบคุม เซนเซอร์ ตัวขับเคลื่อนต่าง ๆ และส่วนที่สองคือ
ส่วนของซอฟท์แวร์ที่เกี่ยวข้องกับการติดต่อสื่อสารกันเบื้องต้น การควบคุมตัวขับเคลื่อนที่ข้อต่อ การอ่านค่าเซนเซอร์
และส่วนที่สาม คือระบบพื้นฐานสำหรับการนำไปศึกษาและพัฒนา โดยจะครอบคลุมไปถึงเอกสารวิธีการใช้งานในรูปแบบออนไลน์

ในการเริ่มทำงานวิจัยที่เกี่ยวกับฮิวมานอยด์นั้นสิ่งจำเป็นที่ต้องทำในอันดับแรกคือการศึกษาสิ่งที่เคยมีอยู่ หรืองานวิจัยมีนักวิจัยอื่นที่ทำเอาไว้แล้ว
จากนั้นศึกษาทำความเข้าใจใน ข้อดี-ข้อเสีย ของวิธีหรือกระบวนการต่างๆ เพื่อนำมาประยุกต์ใช้กับหุ่นยนต์ฮิวมานอยด์ภายในงานวิจัยนี้
ในการศึกษาโครงสร้างทางกลและระบบของหุ่นยนต์ฮิวมานอยด์ที่มีอยู่แล้วสิ่งที่ต้องดูและให้ความสนใจเป็นพิเศษคือ
วิธีการเชื่อมต่อกันระหว่างก้านต่อและข้อต่อ, ลักษณะโครงสร้างที่สามารถทำให้เกิดการเคลื่อนไหวเพียงพอต่อการเดิน,
ตำแหน่งที่ใช้ในติดตั้งเซนเซอร์หรือตัวรับรู้ต่างๆ, การเลือกใช้วัสดุให้เหมาะสม
และการทำงานของเซนเซอร์และตัวขับเคลื่อนที่จำเป็นต้องใช้ในการควบคุมการทำงานของหุ่นยนต์
เมื่อเราทำความศึกษาเสร็จเรียบร้อยแล้ว จึงเริ่มดำเนินงานวิจัยการพัฒนาโครงสร้างและระบบพื้นฐาน
ในส่วนนี้ทางกลจะเป็นการออกแบบโครงสร้างและจัดสร้างขึ้นมา ทางซอฟต์แวร์จะเป็นการสร้างแบบจำลองของหุ่นยนต์
ระบบจำลองที่สามารถจำลองการทำงานของหุ่นยนต์ได้ และติดต่อส่วนของทางกลกับทางโปรแกรม
สุดท้ายระบบพื้นฐานจะออกแบบโครงสร้างระบบพื้นฐานเพื่อทำให้เกิดการพัฒนาต่อได้อย่างมีประสิทธิภาพ
ในบทนี้ก็จะกล่าวถึงกระบวนการออกแบบและการดำเนินการตามแผนที่วางเอาไว้

\clearpage
\section{การออกแบบโครงสร้างของหุ่นยนต์}
\input{chapter3/hardware}

\clearpage
\section{การออกแบบโปรแกรมด้วย ROS}
\input{chapter3/software}

\clearpage
\section{การออกแบบระบบพื้นฐาน}
\input{chapter3/platform}

