% ************************** Thesis Chapter3 **********************************
\chapter{ระเบียบวิธีวิจัย}
ในการทําโครงการวิจัยแอพพลิเคชั่นสำหรับวิเคราะห์วิดีโอ(video analytics) จะมีการทำงานหลากหลายส่วนมาทำงานร่วมกัน ซึ่งทำให้จำเป็นจะต้องมีระเบียบวิจัยสำหรับอธิบายภาพรวม
\subsection*{โดยในระเบียบวิจัยนี้จะมีหัวข้อ และระเบียบวิธีวิจัยดังนี้}
\begin{itemize}\setlength\itemsep{-0.3em}
	\item แผนการดำเนินงาน
	\item เครื่องมือที่ใช้ในการดำเนินงานวิจัย
	\item ภาพรวมของแอพพลิเคชั่น
	\item รายละเอียดของโมเดล
\end{itemize}

\section{หน้าที่ความรับผิดชอบ} 
\paragraph*{ปฐมพงศ์ สินธุ์งาม}
สร้างและทดสอบโมเดลจดจำการกระทำมนุษย์ I3D และออกแบบพร้อมทั้งสร้างระบบ Tracker
\paragraph*{ศุภกร เบญจวิกรัย}
รวบรวมฟังก์ชั่นต่างๆของแอพพลิเคชั่น และออกแบบพร้อมทั้งสร้างระบบแอพพลิเคชั่นในส่วน Selection และ Detection
\paragraph*{อุกฤษฎ์ เลิศวรรณาการ}
สร้างและทดสอบโมเดลจดจำการกระทำมนุษย์ Resnet-50 และออกแบบพร้อมทั้งสร้างระบบ Person ReID 


\vspace{6mm}
\section{เครื่องมือที่ใช้ในงานวิจัย} 
\subsection*{Pycharm community รุ่น 2017.1.2} เป็นโปรแกรมไว้ใช้สำหรับเขียนและแก้ไขโค้ดซึ่งข้อดีของโปรแกรมนี้ คือ มี Feature ต่างๆเพื่ออำนวยความสะดวกในการเขียนโค้ด เช่น syntax highlighting , Auto-completion ฯลฯ  และสามารถนำมารันและทดสอบแอพพลิเคชั่นได้

\subsection*{Jupyter รุ่น 2017.1.2} เป็นโปรแกรมไว้ใช้สำหรับเขียนและแก้ไขโค้ด แต่เหมาะสำหรับรันโมเดล หรือ โค้ดที่รันเป็น section แยกกัน ซึ่งการรันเป็น section มีข้อดีคือ หากมีการแก้ไขโปรแกรม ไม่จำเป็นต้องรันใหม่ทุกส่วนในโปรแกรม

\subsection*{Qt Creator รุ่น 4.9.2 (Community)}
เป็นเครื่องมือสำหรับออกแบบ หน้าต่าง UI ของ pyqt และ เซฟเเป็นไฟล์ .ui ซึ่งมีข้อดีคือ เรียกใช้ง่าย มี  widget ที่สามารถใช้ได้หลากหลายเหมาะสำหรับการออกแบบโปรโตไทป์


\clearpage
\section{ภาษาที่ใช้ในการพัฒนาระบบ} 
	ใช้ภาษา python ในการพัฒนาเป็นหลัก เพราะ python เป็นภาษาที่ปัจจุบันมีการใช้กันอย่างแพร่ทำให้มี tools ที่อำนวยความสะดวกให้แก่การพัฒนา และยังเป็นภาษาที่สามารถเข้าใจได้ง่ายและไม่ซับซ้อนจนเกินไป โดยในการทำวิจัยครั้งนี้ได้เลือก python version 3.6.8 มาใช้ในการพัฒนา เนื่องจากเป็น version ที่รองรับการทำงานของ library tensorflow และ cuda version 9

\vspace{3mm}
\section{Library computer ที่ใช้ในการพัฒนาระบบและแอพพลิเคชั่น} 
\begin{enumerate}\setlength\itemsep{-0.3em}
	\item numpy		\quad\quad\quad version 1.16.4	\quad library ใช้สำหรับการคำนวณและ array
	\item pandas		\quad\quad\quad version 0.24.2	\quad library ใช้สำหรับการจัดการเกี่ยวกับ excel
	\item opencv		\quad\quad \quad version 4.1.0.25	\quad library ใช้สำหรับการจัดการข้อมูลที่เป็นภาพ
	\item pillow		\quad\quad\quad version 6.0.0		\quad library  ใช้สำหรับการจัดการข้อมูลที่เป็นภาพ
	\item torchsummary	\quad version 1.5.1		\quad library ใช้สำหรับการทำ 
	\item pytorch		\quad\quad\quad version 1.10.0	\quad library ใช้สำหรับการทำ AI
	\item torchvision		\quad\quad version 0.3.0	 	\quad library ใช้สำหรับการทำ AI
	\item scikit-learn	\quad\quad version 0.21.2	\quad library ใช้สำหรับการทำ AI
	\item scipy			\quad\quad\quad\quad version 1.3.0		\quad library ใช้สำหรับการทำ AI
	\item sklearn		\quad\quad\quad version 0.0		\quad library ใช้สำหรับการทำ AI
	\item pickleshare	\quad\quad version 0.7.5		\quad library ใช้สำหรับการทำ encoding โมเดล
	\item tqdm			\quad\quad\quad\quad version 4.32.1	\quad library ใช้สำหรับนับจำนวนการทำ loop
	\item pyqt5		\quad\quad\quad\quad version 5.9.2		\quad library ใช้สำหรับการทำแอพพลิเคชั่น
\end{enumerate}


\vspace{3mm}
\section{แผนการดำเนินงาน}
โดยจากที่กล่าวไปตอนต้นในบทนำ
การดำเนินงานและการออกแบบการสร้าง labeling tool และระบบวิเคราะห์การกระทำของมนุษย์ในวิดีโอ มีแผนการทำงานซึ่งถูกแบ่งออกเป็นสามส่วนดังนี้ 
ส่วนแรกคือ ส่วนของการศึกษาหาความเป็นไปได้ และเทคโนโลยีในปัจจุบันที่เกี่ยวกับการสร้างแอพพลิเคชั่น และการจดจำการกระทำของมนุษย์ด้วยปัญญาประดิษฐ์ เพื่อนำมาประยุกต์ใช้กับงานวิจัยนี้
ส่วนที่สองคือ ส่วนของการออกแบบและสร้างแอพพลิเคชั่นที่ใช้ในการสร้างชุดข้อมูลสำหรับการเทรนโมเดลจากวิดีโอ
ส่วนที่สามคือ ส่วนของการออกแบบและสร้างระบบแพลตฟอร์มวิเคราะห์การกระทำของมนุษย์ได้โดยมีข้อกำหนดตามที่กล่าวไว้ในบทนำ

ในการเริ่มทำงานวิจัยนี้นั้นสิ่งจำเป็นที่ต้องทำในอันดับแรกคือการศึกษาสิ่งที่เคยมีอยู่ หรืองานวิจัยอื่นที่ทำเอาไว้แล้ว
เพื่อศึกษาและทำความเข้าใจ ข้อดี-ข้อเสีย ของเทคนิคหรือกระบวนการต่างๆ เพื่อนำมาประยุกต์ใช้กับงานวิจัยนี้
ในการศึกษาเกี่ยวกับการออกแบบและการสร้างแอพพลิเคชั่นที่ใช้ในการสร้างชุดข้อมูลสำหรับการเทรนโมเดลจากวิดีโอ
ที่มีอยู่แล้วสิ่งที่ต้องให้ความสนใจคือฟังก์ชั่นการทำงาน การออกแบบและการจัดวางองค์ประกอบต่างๆในหน้าต่าง UI
และความสะดวกในการใช้งาน จากนั้นจึงเริ่มศึกษาเกี่ยวกับแพตฟอร์มที่ใช้ในการสร้างแอพพลิเคชั่น
ส่วนการศึกษาเกี่ยวกับการสร้างระบบวิเคราะห์การกระทำมนุษย์นั้น จะมุ่งความสนใจไปที่ชุดข้อมูลสำหรับการวิเคราะห์วิดีโอ
โมเดลสำหรับการวิเคราะห์วิดีโอ เทคนิคในการสร้างโมเดล เทคโนโลยีในการทำระบบวิเคราะห์วิดีโอ
เพื่อใช้ในการออกแบบและสร้างระบบวิเคราะห์การกระทำของมนุษย์ในวิดีโอให้มีประสิทธิภาพ
ในบทนี้ก็จะกล่าวถึงกระบวนการออกแบบและการดำเนินการตามแผนที่วางเอาไว้

\clearpage
\section{การออกแบบแอพพลิเคชั่น labeling tool}
การออกแบบเครื่องมือสำหรับกำกับข้อมูลด้วยปัญญาประดิษฐ์ ผู้วิจัยได้เลือกใช้ library PyQt และภาษาไพธอนในการพัฒนา
เนื่องจาก PyQt นั้นเป็น library ที่มีผู้พัฒนาใช้กันอย่างแพร่หลาย จึงสะดวกในการศึกษา หาข้อมูลในการสร้างหรือแก้ไข
อีกทั้งยังเป็น library ที่สามารถพัฒนาด้วยภาษาไพธอนได้ และใช้งานง่าย สามารถปรับปรุงแก้ไขได้สะดวก

\subsection{เครื่องมือสำหรับกำกับข้อมูลด้วยปัญญาประดิษฐ์}
เครื่องมือสำหรับกำกับข้อมูลด้วยปัญญาประดิษฐ์ แบ่งการทำงานออกเป็น 4 กระบวนการทำงาน คือ Select, Detect, Track และ Label
เพื่อช่วยแบ่งเบาภาระของผู้พัฒนาในการสร้างชุดข้อมูลสำหรับสร้างโมเดลจากข้อมูลประเภทวิดีโอ โดยกระบวนการ Select
จะต้องสามารถตัดวิดีโอส่วนที่ไม่มีมนุษย์อยู่ออกจากวิดีโอได้ จากนั้นกระบวนการ Detect จะต้องหาตำแหน่งของมนุษย์ภายในวิดีโอได้
แล้วใช้กระบวนการ Track ติดตามการเคลื่อนไหวตำแหน่งต่อไปของมนุษย์ในเฟรมถัดๆไป
และกระบวนการสุดท้าย คือ Label นั้นต้องสามารถทำนายการกระทำพื้นฐานของมนุษย์ได้ เช่น ยืน เดิน นั่ง กินข้าว หรือ นอน เป็นต้น 
โดยทุกส่วนการทำงานมนุษย์ต้องสามารถทำงานร่วมกับปัญญาประดิษฐ์ได้
ดังรูปที่ \ref{fig:labeling_system}

\begin{figure}[!ht]
    \centering
    \includegraphics[width=0.5\textwidth]{chapter3/images/3_6/labelingToolOverview.png}
    \caption{กระบวนการหลักของเครื่องมือสำหรับกำกับข้อมูลด้วยปัญญาประดิษฐ์}
    \label{fig:labeling_system}
\end{figure}
\clearpage

\subsection*{โดยแต่ละกระบวนการจะมีรายละเอียดดังนี้}
\subsection*{หน้าต่าง Select}
กระบวนการ Select จะต้องสามารถรับวิดีโอเข้ามา แล้วตัดวิดีโอในช่วงที่ไม่มนุษย์อยู่ในเฟรมออกได้อัตโนมัติด้วยปัญญาประดิษฐ์
แต่เนื่องจากการประมวลผลทุกเฟรมในวิดีโอนั้นจะทำให้เสียเวลามากเกินไป จึงใช้วิธีการเลือกตัวอย่างเฟรมด้วยอัตราคงที่ (สามารถกำหนดได้)
ซึ่งเรียกเฟรมเหล่านี้ว่า คีย์เฟรม จากนั้นใช้ปัญญาประดิษฐ์ประมวลผลคีย์เฟรม เพื่อลดระยะเวลาในการประมวลผลลง และมนุษย์จะต้องสามารถแก้ไขข้อผิดพลาดของปัญญาประดิษฐ์ได้ เพื่อเพิ่มคุณภาพของชุดข้อมูล จึงออกแบบหน้าต่างได้ดังรูปที่ \ref{fig:SelectDraft}

\begin{figure}[!ht]
    \centering
    \includegraphics[width=1\textwidth]{chapter3/images/3_6/SelectDraft.png}
    \caption{หน้าต่าง Select ของเครื่องมือสำหรับกำกับข้อมูลด้วยปัญญาประดิษฐ์}
    \label{fig:SelectDraft}
\end{figure}
\clearpage
\begin{figure}[!ht]
    \centering
    \includegraphics[width=1\textwidth]{chapter3/images/3_6/SelectDraft_point.png}
    \caption{ตำแหน่งของแต่ละวิดเจ็ตในหน้าต่าง Select}
    \label{fig:SelectDraft_point}
\end{figure}
โดยที่แต่ละวิดเจ็ตตามหมายเลขที่กำหนดตามรูปที่ \ref{fig:SelectDraft_point} มีรายละเอียดดังนี้
\begin{enumerate}
	\setlength\itemsep{-0.25em}
	\item หมายเลข 1 คือปุ่มสำหรับเลือกไฟล์วิดีโอที่ต้องการจากในคอมพิวเตอร์เข้ามาในเครื่องมือกำกับคุณลักษณะด้วยปัญญาประดิษฐ์
    	\item หมายเลข 2 คือปุ่มสำหรับสั่งให้ระบบทำการสุ่มคีย์เฟรมขึ้นมา แล้วใช้ปัญญาประดิษฐ์ประมวลผลเพื่อแยกว่าคีย์เฟรมไหนมีคนอยู่ และคีย์เฟรมไหนไม่มีคนอยู่แบบอัตโนมัติ
    	\item หมายเลข 3 คือแถบเลื่อนเพื่อกำหนดความถี่ในการสุ่มคีย์เฟรม โดยจะมีช่วงอยู่ที่สุ่มหยิบทุกๆ 1 เฟรมต่อวินาที จนถึง อัตราเฟรมต่อวินาทีสูงสุดของวิดีโอที่รับเข้ามา
	\item หมายเลข 4 คือกล่องสำหรับแสดงชื่อวิดีโอที่รับเข้ามาในเครื่องมือกำกับคุณลักษณะด้วยปัญญาประดิษฐ์เพื่อเลือกเข้ามาใช้ในการประมวลผล
	\item หมายเลข 5 คือกล่องสำหรับแสดงว่าคีย์เฟรมใดมีมนุษย์อยู่ในเฟรม โดยที่ผู้ใช้งานสามารถตรวจสอบความถูกต้องและแก้ไขข้อผิดพลาดของปัญญาประดิษฐ์ได้
	\item หมายเลข 6 คือกล่องสำหรับแสดงว่าคีย์เฟรมใดไม่มีมนุษย์อยู่ในเฟรม โดยที่ผู้ใช้งานสามารถตรวจสอบความถูกต้องและแก้ไขข้อผิดพลาดของปัญญาประดิษฐ์ได้
	\item หมายเลข 7 คือหน้าต่างสำหรับแสดงเฟรมที่เลือกจากหมายเลข 5 หมายเลข 6 หรือหมายเลข 8
	\item หมายเลข 8 คือแถบเลื่อนสำหรับเลื่อนดูคีย์เฟรมทั้งหมดที่ระบบสร้างขึ้น
	\item หมายเลข 9 คือปุ่มสำหรับไปกระบวนการต่อไปหลังจากระบบประมวลผลเสร็จแล้ว
\end{enumerate}
\clearpage

\subsection*{หน้าต่าง Detect}
กระบวนการ Delect จะต้องสามารถรับคีย์เฟรมจากกระบวนการ Select มาประมวลผลด้วยปัญญาประดิษฐ์เพื่อหาตำแหน่งของมนุษย์ที่อยู่ในคีย์เฟรม 
แล้วสร้างกรอบสี่เหลี่ยมครอบบริเวณดังกล่าวได้ในแบบอัตโนมัติ เพื่อแบ่งเบาภาระผู้ใช้ในการที่ต้องสร้างกรอบสี่เหลี่ยมครอบตำแหน่งของมนุษย์ด้วยตัวเอง
และผู้ใช้ต้องสามารถสร้างหรือลบกรอบสี่เหลี่ยมได้ด้วยตัวเองสำหรับแก้ไขความผิดพลาดของปัญญาประดิษฐ์ เพื่อเพิ่มคุณภาพของชุดข้อมูล
จึงออกแบบหน้าต่างได้ดังรูปที่ \ref{fig:DetectDraft}
\begin{figure}[!ht]
    \centering
    \includegraphics[width=1\textwidth]{chapter3/images/3_6/DetectDraft.png}
    \caption{หน้าต่าง Detect ของเครื่องมือสำหรับกำกับข้อมูลด้วยปัญญาประดิษฐ์}
    \label{fig:DetectDraft}
\end{figure}
\clearpage
\begin{figure}[!ht]
    \centering
    \includegraphics[width=1\textwidth]{chapter3/images/3_6/DetectDraft_point.png}
    \caption{ตำแหน่งของแต่ละวิดเจ็ตในหน้าต่าง Detect}
    \label{fig:DelectDraft_point}
\end{figure}
โดยที่แต่ละวิดเจ็ตตามหมายเลขที่กำหนดตามรูปที่ \ref{fig:DelectDraft_point} มีรายละเอียดดังนี้
\begin{enumerate}
	\setlength\itemsep{-0.25em}
    \item หมายเลข 1 คือช่องสำหรับกดเพื่อเปลี่ยนระบบจากสร้างกรอบสี่เหลี่ยมในแบบแก้ไขด้วยตนเองเป็นลบกรอบสี่เหลี่ยมแทน
    \item หมายเลข 2 คือช่องสำหรับเลือกว่าจะใช้ระบบการทำงานแบบใด ระหว่างแบบทำงานอัตโนมัติและแบบปรับแก้ไขด้วยตนเอง
    \item หมายเลข 3 คือปุ่มสำหรับสั่งให้ระบบทำการตรวจหาตำแหน่งของมนุษย์ในคีย์เฟรมทั้งหมดแล้วสร้างกรอบสี่เหลี่ยมขึ้นมาครอบบริเวณที่กำหนด
	\item หมายเลข 4 คือกล่องสำหรับแสดงคีย์เฟรมทั้งหมด
	\item หมายเลข 5 คือหน้าต่างสำหรับแสดงเฟรมที่เลือกจากหมายเลข 4 หรือหมายเลข 6
	\item หมายเลข 6 คือแถบเลื่อนสำหรับเลื่อนดูคีย์เฟรมทั้งหมดที่มี เพื่อตรวจสอบความถูกต้องของปัญญาประดิษฐ์
	\item หมายเลข 7 คือปุ่มสำหรับไปกระบวนการต่อไปหลังจากระบบประมวลผลเสร็จแล้ว
\end{enumerate}
\clearpage

\subsection*{หน้าต่าง Track}
เนื่องจากกระบวนการ Detect นั้นจะทำเฉพาะในคีย์เฟรมทำให้ในเฟรมอื่นๆนอกเหนือจากนั้นจะไม่มีกรอบสี่เหลี่ยปรากฎอยู่
ดังนั้นกระบวนการ Track จึงต้องสามารถติดตามการเคลื่อนไหวตำแหน่งต่อไปของมนุษย์ และสร้างกรอบสี่เหลี่ยมขึ้นมาบนเฟรมระหว่างคีย์เฟรมทั้งหมดได้โดยอัตโนมัติ
เพื่อสร้างข้อมูลตำแหน่งของมนุษย์ในเฟรมเหล่านั้น สุดท้ายนี้ผู้ใช้ต้องสามารถสร้างหรือลบกรอบสี่เหลี่ยมได้ด้วยตัวเองสำหรับแก้ไขความผิดพลาดของอัลกอริทึม
จึงออกแบบหน้าต่างได้ดังรูปที่ \ref{fig:TrackDraft}
\begin{figure}[!ht]
    \centering
    \includegraphics[width=1\textwidth]{chapter3/images/3_6/TrackDraft.png}
    \caption{หน้าต่าง Track ของเครื่องมือสำหรับกำกับข้อมูลด้วยปัญญาประดิษฐ์}
    \label{fig:TrackDraft}
\end{figure}
\clearpage
\begin{figure}[!ht]
    \centering
    \includegraphics[width=1\textwidth]{chapter3/images/3_6/TrackDraft_point.png}
    \caption{ตำแหน่งของแต่ละวิดเจ็ตในหน้าต่าง Track}
    \label{fig:TrackDraft_point}
\end{figure}
โดยที่แต่ละวิดเจ็ตตามหมายเลขที่กำหนดตามรูปที่ \ref{fig:TrackDraft_point} มีรายละเอียดดังนี้
\begin{enumerate}
	\setlength\itemsep{-0.25em}
    \item หมายเลข 1 คือช่องสำหรับกดเพื่อเปลี่ยนระบบจากสร้างกรอบสี่เหลี่ยมในแบบแก้ไขด้วยตนเองเป็นลบกรอบสี่เหลี่ยมแทน
    \item หมายเลข 2 คือช่องสำหรับเลือกว่าจะใช้ระบบแบบใด ระหว่างแบบอัตโนมัติและแบบแก้ไขด้วยตนเอง
    \item หมายเลข 3 คือปุ่มสำหรับสั่งให้ระบบทำการตรวจหาตำแหน่งของมนุษย์ในเฟรมระหว่างคีย์เฟรมทั้งหมดแล้วสร้างกรอบสี่เหลี่ยมขึ้นมาครอบบริเวณที่กำหนด
	\item หมายเลข 4 คือกล่องสำหรับแสดงกรอบสี่เหลี่ยมทั้งหมดที่อยู่ในเฟรม
	\item หมายเลข 5 คือหน้าต่างสำหรับแสดงเฟรมที่เลือกจากหมายเลข 6
	\item หมายเลข 6 คือแถบเลื่อนสำหรับเลื่อนดูเฟรมทั้งหมดที่มี เพื่อตรวจสอบความถูกต้องของอัลกอริทึม
	\item หมายเลข 7 คือปุ่มสำหรับไปกระบวนการต่อไปหลังจากระบบประมวลผลเสร็จแล้ว
\end{enumerate}
\clearpage

\subsection*{หน้าต่าง Label}
กระบวนการ Label นั้นต้องสามารถทำนายว่าการกระทำของมนุษย์ที่อยู่ในแต่ละเฟรมว่าคืออะไรได้โดยอัตโนมัติด้วยปัญญาประดิษฐ์
และผู้ใช้จะต้องสามารถแก้ไขข้อผิดพลาดของปัญญาประดิษฐ์ได้หากมีการทำนายที่ผิดพลาดเกิดขึ้น
หรือถ้าหากผู้ใช้ต้องการเพิ่มการกระทำที่ไม่ได้มีอยู่ในชุดการกระทำพื้นฐานที่มีอยู่แล้วของปัญญาประดิษฐ์ ผู้ใช้ก็สามารถเพิ่มการกระทำนั้นเข้ามาได้
จึงออกแบบหน้าต่างได้ดังรูปที่ \ref{fig:ActionLabelDraft}
\begin{figure}[!ht]
    \centering
    \includegraphics[width=1\textwidth]{chapter3/images/3_6/ActionLabelDraft.png}
    \caption{หน้าต่าง Label ของเครื่องมือสำหรับกำกับข้อมูลด้วยปัญญาประดิษฐ์}
    \label{fig:ActionLabelDraft}
\end{figure}
\clearpage
\begin{figure}[!ht]
    \centering
    \includegraphics[width=1\textwidth]{chapter3/images/3_6/ActionLabelDraft_point.png}
    \caption{ตำแหน่งของแต่ละวิดเจ็ตในหน้าต่าง Label}
    \label{fig:ActiobLabelDraft_point}
\end{figure}
โดยที่แต่ละวิดเจ็ตตามหมายเลขที่กำหนดตามรูปที่ \ref{fig:TrackDraft_point} มีรายละเอียดดังนี้
\begin{enumerate}
	\setlength\itemsep{-0.25em}
    \item หมายเลข 1 คือช่องสำหรับเลือกว่าจะใช้ระบบแบบใด ระหว่างแบบอัตโนมัติและแบบแก้ไขด้วยตนเอง
    \item หมายเลข 2 คือปุ่มสำหรับสั่งให้ระบบทำนายการกระทำของมนุษย์ในทุกๆเฟรม
    \item หมายเลข 3 คือกล่องสำหรับแสดงกรอบสี่เหลี่ยมทั้งหมดที่อยู่ในเฟรมที่เลือก
	\item หมายเลข 4 คือกล่องสำหรับแสดงการกระทำของมนุษย์แต่ละคนที่อยู่ในเฟรมที่เลือก โดยจะเรียงลำดับคู่กับกรอบสี่เหลี่ยมที่อยู่ในช่องหมายเลข 3
    \item หมายเลข 5 คือกล่องสำหรับแสดงชุดการกระทำที่ปัญญาประดิษฐ์มีอยู่แล้ว ซึ่งในการทำงานแบบแก้ไขด้วยตนเองนั้น จะสามารถค้นหาการกระทำที่มีอยู่แล้วได้ 
    และหากคำที่ใส่เข้ามานั้นไม่มีอยู่ในชุดการกระทำก็จะเป็นการเพิ่มการกระทำนั้นเข้ามาแทน
	\item หมายเลข 6 คือหน้าต่างสำหรับแสดงเฟรมที่เลือกจากหมายเลข 7
	\item หมายเลข 7 คือแถบเลื่อนสำหรับเลื่อนดูเฟรมทั้งหมดที่มี เพื่อตรวจสอบความถูกต้องของปัญญาประดิษฐ์
	\item หมายเลข 8 คือปุ่มสำหรับสร้างไฟล์ xml ของทุกๆเฟรมสำหรับใช้ในการสร้างโมเดลโดยรายละเอียดข้อมูลภายในไฟล์ xml จะอยู่ในหัวข้อ \ref{sec:XMLInfo}
\end{enumerate}
\clearpage

\subsection*{รายละเอียดข้อมูลภายในไฟล์ xml}
\label{sec:XMLInfo}
ไฟล์ xml นั้นเป็นรูปแบบที่นิยมใช้ในการเก็บข้อมูลสำหรับการสร้างโมเดลประเภทตรวจจับวัตถุ
โดยจะเก็บข้อมูลในรูปแบบของ PASCAL VOC ที่นิยมใช้ในการสร้างโมเดลด้วย Tensorflow โดยภายในไฟล์จะมีข้อมูลดังรูปที่ \ref{fig:XMLFormat}
\begin{figure}[!ht]
    \centering
    \includegraphics[width=1\textwidth]{chapter3/images/3_6/XMLFormat.png}
    \caption{ตัวอย่างข้อมูลภายในไฟล์ xml}
    \label{fig:XMLFormat}
\end{figure}
โดยข้อมูลส่วนสำคัญของรูปแบบนี้นั้นจะถูกใส่หมายเลขกำกับไว้ซึ่งแต่ละหมายเลขนั้นหมายถึง
\begin{enumerate}
	\setlength\itemsep{-0.25em}
    \item หมายเลข 1 คือชื่อโฟลเดอร์ที่เก็บไฟล์รูปภาพที่เกี่ยวข้องกับไฟล์ xml นี้อยู่
    \item หมายเลข 2 คือชื่อไฟล์ที่เกี่ยวข้องกับไฟล์ xml นี้
    \item หมายเลข 3 คือเส้นทางในคอมพิวเตอร์ (directory path) ของไฟล์รูปภาพที่เกี่ยวข้องกับไฟล์ xml นี้
    \item หมายเลข 4 คือขนาดและมิติของรูปภาพ ซึ่งจะประกอบด้วยความกว้าง (width) ความยาว (height) และจำนวนช่องสี (depth) 
    โดยที่จำนวนช่องสีที่มีความลึก 3 มักจะหมายถึงภาพสี RGB และจำนวนช่องสีที่มีความลึก 2 จะหมายถึงภาพขาวดำ (gray scale)
	\item หมายเลข 5 คือ label ของวัตถุหรืออย่างอื่น ที่อยู่ในกรอบสี่เหลี่ยมที่ถูกกำหนดไว้ในส่วนของหมายเลข 6
	\item หมายเลข 6 คือ กรอบสี่เหลี่ยมที่ครอบวัตถุที่สนใจ เช่นมนุษย์ เป็นต้น
\end{enumerate}

\clearpage
\section{การออกแบบระบบวิเคราะห์การกระทำของมนุษย์}
\input{chapter3/software}