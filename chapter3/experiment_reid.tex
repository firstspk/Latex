\subsection*{สิ่งที่ใช้ในการวัดผล}
	\begin{enumerate}
		\item ความแม่นยำสำหรับการระบุตัวตนของบุคคลภายในภาพ
	\end{enumerate}
\subsection*{สมมุติฐาน :}ผู้วิจัยได้ตั้งสมมุติฐานว่า ผลลัพธ์ของการทดลองของโมเดลปัญญาประดิษฐ์ ResNet50 ที่ผ่านการ train มาด้วยชุดข้อมูล Market1501 นั้นจะมีความแม่นยำมากที่สุดในการระบุตัวตนของบุคคลภายในภาพ เมื่อเทียบกับโมเดลปัญญาประดิษฐ์อื่นที่มาจากแหล่งข้อมูลเดียวกัน จึงนำโมเดลปัญญาประดิษฐ์นี้มาทดสอบสำหรับการใช้งานจริง
\subsection*{ตัวแปรควบคุม}
	\begin{enumerate}
		\item ชุดข้อมูล : ชุดข้อมูลที่ทางผู้วิจัยสร้างขึ้นสำหรับการทดสอบ
		\item โมเดลปัญญาประดิษฐ์ : YoLo-V3 320  สำหรับการหาตำแหน่งของบุคคล
	\end{enumerate}
\subsection*{วิธีการทดลอง}
	\begin{enumerate}
		\item ดาว์นโหลดโมเดลปัญญาประดิษฐ์ที่ผ่านการ train ด้วยชุดข้อมูลต่างได้แก่ Market1501 , DukeMTMCReID, CUHK03 และ MSMT17
		\item นำชุดข้อมูลที่ผู้วิจัยสร้างขึ้นมาผ่านโมเดลปัญญาประดิษฐ์ YoLo-V3 320 เพื่อหาตำแหน่งของบุคคล
		\item นำโมเดลปัญญาประดิษฐ์แต่ละอันมาทดสอบความแม่นยำสำหรับการระบุตัวตนของบุคคลภายในภาพ ด้วยตำแหน่งของบุคคลที่ได้มาจากก่อนหน้านี้
		\item ประเมินผลการทำงานโดยเทียบความแม่นยำสำหรับการระบุตัวตนของบุคคลภายในภาพของแต่ละโมเดลปัญญาประดิษฐ์ เพื่อหาโมเดลปัญญาประดิษฐ์ที่ได้ผลลัพท์ดีที่สุด
\end{enumerate}
