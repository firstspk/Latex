\subsection*{สิ่งที่ใช้ในการวัดผล}
	\begin{enumerate}
		\setlength\itemsep{-0.25em}
		\item ค่า AUC ที่ใช้สำหรับการระบุตัวตนของบุคคลภายในภาพ
	\end{enumerate}
\subsection*{สมมติฐาน}
ผู้วิจัยได้ตั้งสมมติฐานว่า ผลลัพธ์ของการทดลองการใช้งานจริงของโมเดลปัญญาประดิษฐ์ ResNet50 ที่สร้างด้วยชุดข้อมูล Market1501 
นั้นควรจะมีความแม่นยำในการระบุตัวตนของบุคคลภายในภาพมากที่สุดเมื่อเทียบกับโมเดลปัญญาประดิษฐ์ที่สร้างด้วยชุดข้อมูลอื่นๆ
เพราะเมื่อเทียบกับโมเดลปัญญาประดิษฐ์ที่ถูกสร้างด้วยชุดข้อมูลอื่นที่มาจากแหล่งข้อมูลเดียวกัน โมเดลปัญญาประดิษฐ์ ResNet50 ที่สร้างด้วยชุดข้อมูล Market1501 นั้นจะมีความแม่นยำสูงสุด
\subsection*{ตัวแปร}
	\begin{enumerate}
		\setlength\itemsep{-0.25em}
		\item โมเดลปัญญาประดิษฐ์ ซึ่งได้แก่
		\begin{enumerate}
			\setlength\itemsep{-0.25em}
			\item ResNet50 ที่ถูกสร้างด้วยชุดข้อมูล Market1501
			\item ResNet50 ที่ถูกสร้างด้วยชุดข้อมูล DukeMTMCReID
			\item ResNet50 ที่ถูกสร้างด้วยชุดข้อมูล CUHK03	
			\item ResNet50 ที่ถูกสร้างด้วยชุดข้อมูล MSMT17
		\end{enumerate}
	\end{enumerate}
\subsection*{ตัวแปรควบคุม}
	\begin{enumerate}
		\setlength\itemsep{-0.25em}
		\item ชุดข้อมูล : ชุดข้อมูลที่ทางผู้วิจัยสร้างขึ้นสำหรับการทดสอบ
		\item โมเดลปัญญาประดิษฐ์ : YOLO-V3 320  สำหรับการหาตำแหน่งของบุคคล
	\end{enumerate}
\subsection*{วิธีการทดลอง}
	\begin{enumerate}
		\setlength\itemsep{-0.25em}
		\item นำชุดข้อมูลที่ผู้วิจัยสร้างขึ้นมาผ่านโมเดลปัญญาประดิษฐ์ YOLO-V3 320 เพื่อหาตำแหน่งของบุคคล
		\item นำโมเดลปัญญาประดิษฐ์แต่ละอันมาทดสอบความแม่นยำสำหรับการระบุตัวตนของบุคคลภายในภาพ ด้วยตำแหน่งของบุคคลที่ได้มาจากขั้นตอนก่อนหน้านี้
		\item ประเมินผลการทำงานโดยเทียบค่า AUC สำหรับการระบุตัวตนของบุคคลภายในภาพของแต่ละโมเดลปัญญาประดิษฐ์ เพื่อหาโมเดลปัญญาประดิษฐ์ที่เหมาะสมกับชุดข้อมูลของผู้วิจัยมากที่สุด
\end{enumerate}
