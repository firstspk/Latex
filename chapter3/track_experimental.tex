\subsection*{สิ่งที่ใช้ในการวัดผล}
	\begin{enumerate}
		\item ความเร็วต่อวิดีโอ (วินาที)
		\item ความแม่นยำ โดยคำนึงถึงอัตราส่วนร่วมของกรอบที่เหลี่ยม
	\end{enumerate}
\subsection*{สมมุติฐาน}
ผู้วิจัยได้ตั้งสมมุติฐานว่า การใช้โมเดลปัญญาประดิษฐ์สำหรับตรวจจับวัตถุและสร้างกรอบสี่เหลี่ยมทุกๆ N เฟรม 
แล้วใช้ระบบทำนายตำแหน่งต่อไปของวัตถุในการสร้างกรอบสี่เหลี่ยมในเฟรมระหว่างนั้น จะทำให้ระบบสามารถทำงานได้เร็วขึ้น โดยที่ประสิทธิภาพจะลดลงเพียงเล็กน้อย
\subsection*{ตัวแปรควบคุม}
	\begin{enumerate}
		\item วิดีโอสาธารณะที่ไม่ติดลิขสิทธิ์ ความยาวประมาณ 10 - 30 วินาที หนึ่งวิดีโอ
		\item ใช้โมเดลปัญญาประดิษฐ์สำหรับตรวจจับตำแหน่งวัตถุ ResNet50 ในการสร้างชุดข้อมูลที่มีการกำกับตำแหน่งวัตถุไว้ (ground-truth) แล้วใช้มนุษย์ในการตรวจสอบความถูกต้อง
		เพื่อใช้เป็นคำตอบของการทำนาย
		\item โมเดลปัญญาประดิษฐ์สำหรับตรวจจับตำแหน่งที่ใช้ในการเปรียบเทียบ: YOLO-V3 320
		\item อัลกอริทึมสำหรับระบบทำนายตำแหน่งต่อไปของวัตถุ: dlib
		\item อัตราส่วนร่วมของกรอบที่เหลี่ยม: มีส่วนที่ทับกันมากกว่า 80\% ขึ้นไปจึงจะนับว่าผลการทำนายถูกต้อง
	\end{enumerate}
\subsection*{วิธีการทดลอง}
	\begin{enumerate}
		\item ใช้โมเดลปัญญาประดิษฐ์ YOLO-v3 320 ประมวลผลทุกเฟรมในวิดีโอ และเปรียบเทียบผลลัพธ์กับชุดข้อมูลที่ถูกกำกับตำแหน่งวัตถุไว้แล้ว เพื่อคำนวณหาความแม่นยำ
		\item ใช้โมเดลปัญญาประดิษฐ์ YOLO-v3 320 ประมวลผลทุกๆ N เฟรมในวิดีโอ แล้วใช้ระบบทำนายตำแหน่งต่อไปของวัตถุในการสร้างกรอบสี่เหลี่ยมในเฟรมระหว่างนั้น 
		และเปรียบเทียบผลลัพธ์กับชุดข้อมูลที่ถูกกำกับตำแหน่งวัตถุไว้แล้ว เพื่อคำนวณหาความแม่นยำ โดยที่ค่า N จะเท่ากับ 10 20 และ 25
		\item เปรียบเทียบความเร็วในการประมวลผล และความแม่นยำ
\end{enumerate}