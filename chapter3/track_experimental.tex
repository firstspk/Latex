\subsection*{สิ่งที่ใช้ในการวัดผล}
	\begin{enumerate}
		\item ความเร็วต่อวิดีโอ (วินาที)
		\item ความแม่นยำ (อัตราส่วนร่วมของกรอบที่เหลี่ยม หรือ Intersection of Union)
	\end{enumerate}
\subsection*{สมมุติฐาน}
ผู้วิจัยได้ตั้งสมมุติฐานว่า การใช้โมเดลปัญญาประดิษฐ์สำหรับตรวจจับวัตถุและสร้างกรอบสี่เหลี่ยมทุกๆ N เฟรม 
แล้วใช้ระบบทำนายตำแหน่งต่อไปของวัตถุในการสร้างกรอบสี่เหลี่ยมในเฟรมระหว่างนั้น จะทำให้ระบบสามารถทำงานได้เร็วขึ้น โดยที่ประสิทธิภาพจะลดลงเพียงเล็กน้อย
\subsection*{ตัวแปรควบคุม}
	\begin{enumerate}
		\item วิดีโอสาธารณะที่ไม่ติดลิขสิทธิ์ ความยาวประมาณ 120 - 180 วินาที หนึ่งวิดีโอ
		\item ใช้โมเดลปัญญาประดิษฐ์สำหรับตรวจจับตำแหน่งวัตถุ ResNet50 ในการสร้างชุดข้อมูลที่มีคำตอบ (ground-truth)
		\item โมเดลปัญญาประดิษฐ์สำหรับตรวจจับตำแหน่งที่ใช้ในการเปรียบเทียบ: YoLo-V3 320
		\item อัลกอริทึมสำหรับระบบทำนายตำแหน่งต่อไปของวัตถุ: dlib
	\end{enumerate}
\subsection*{วิธีการทดลอง}
	\begin{enumerate}
		\item แบ่งการทดลองออกเป็น 3 รูปแบบ ดังนี้
            \begin{enumerate}
                \item ใช้โมเดลปัญญาประดิษฐ์ YoLo-v3 320 ประมวลผลทุกเฟรมในวิดีโอ
                \item ใช้โมเดลปัญญาประดิษฐ์ YoLo-v3 320 ประมวลผลเพียงเฟรมแรกของวิดีโอ แล้วใช้ระบบทำนายตำแหน่งต่อไปของวัตถุในการสร้างกรอบสี่เหลี่ยมในเฟรมที่เหลือ
                \item ใช้โมเดลปัญญาประดิษฐ์ YoLo-v3 320 ประมวลผลทุกๆ N เฟรมในวิดีโอ แล้วใช้ระบบทำนายตำแหน่งต่อไปของวัตถุในการสร้างกรอบสี่เหลี่ยมในเฟรมระหว่างนั้น
            \end{enumerate}
		\item แบ่งชุดข้อมูลออกเป็น ชุดข้อมูลสำหรับทดสอบ และ ชุดข้อมูลที่มีคำตอบ
			\begin{enumerate}
				\item ชุดข้อมูลสำหรับทดสอบ ประกอบด้วย : ชื่อของวิดีโอ
				\item ชุดข้อมูลที่มีคำตอบ ประกอบด้วย : ชื่อของวิดีโอ,เฟรม,ตำแหน่งของกรอบสี่เหลี่ยม,ไอดีของการกระทำ
			\end{enumerate}
		\item เรียกชื่อของวิดีโอจากชุดข้อมูลทดสอบ และนำ Machine learning model ทำนายผลลัพธ์ จากนั้นเก็บผลลัพธ์เป็น ชุดข้อมูลผลลัพธ์จากการทำนาย
			\begin{enumerate}
				\item ชุดข้อมูลผลลัพธ์จากการทำนาย ประกอบด้วย : ชื่อของวิดีโอ,เฟรม,ตำแหน่งของกรอบสี่เหลี่ยม,ไอดีของการกระทำ,ความมั่นใจ
			\end{enumerate}
		\item ประเมินผลการทำงานโดยเทียบระหว่างชุดผลลัพธ์จากการทำนาย และ ชุดข้อมูลที่มีคำตอบ ผ่านฟังก์ชั่นจากแหล่งที่มา		
		\item เปรียบเทียบผลลัพธ์จากแหล่งที่มา
\end{enumerate}