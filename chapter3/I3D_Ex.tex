\subsection*{สิ่งที่ใช้ในการวัดผล}
	\begin{enumerate}
		\setlength\itemsep{-0.25em}
		\item PASCAL mAP
		\item Top@1 accuracy
		\item Top@3 accuracy
	\end{enumerate}
\subsection*{สมมติฐาน}
ผู้วิจัยได้ตั้งสมมติฐานว่า โมเดลปัญญาประดิษฐ์ I3D ที่ถูกสร้างด้วยชุดข้อมูลแบบ optical flow จะมีความแม่นยำสูงกว่าโมเดลปัญญาประดิษฐ์ที่ถูกสร้างด้วยชุดข้อมูลแบบปกติ
\subsection*{ตัวแปรควบคุม}
	\begin{enumerate}
		\setlength\itemsep{-0.25em}
		\item ชุดข้อมูล : ชุดข้อมูลที่ผู้วิจัยสร้างด้วยเครื่องมือกำกับคุณลักษณะ
		\item โมเดลปัญญาประดิษฐ์ : I3D
	\end{enumerate}
\subsection*{วิธีการทดลอง}
	\begin{enumerate}
		\setlength\itemsep{-0.25em}
		\item แบ่งชุดข้อมูลทั้งหมดในแต่ละการกระทำให้เป็นชุด ชุดละ 20 - 40 เฟรม 
        \item แบ่งชุดข้อมูลออกเป็น ชุดข้อมูลสำหรับสร้างโมเดลปัญญาประดิษฐ์ ชุดข้อมูลสำหรับทดสอบ และชุดข้อมูลที่มีคำกำกับเพื่อเป็นคำตอบ โดยจะมีรายละเอียดดังนี้
            \begin{table}[!ht]
                \centering
                \begin{tabular}{|c|c|c|c|}
                    \hline
                    Label & Train set & Validation set & Test set\\
                    \hline
                    Play phone & 705 & 191 & 96\\
                    Eat & 825 & 165 & 83\\
                    Sit & 775 & 206 & 104\\
                    Sleep & 299 & 96 & 48\\
                    Stand & 426 & 108 & 54\\
                    Walk & 261 & 75 & 38\\
                    \hline
                \end{tabular}
                \caption{ตารางแสดงจำนวนชุดของข้อมูลที่ใช้ในการทดลองนี้}
                \label{tab: I3D_datasetInfo}
            \end{table}
        \item สร้างโมเดลด้วยโครงสร้าง I3D และชุดข้อมูลที่ผู้วิจัยสร้างด้วยเครื่องมือกำกับคุณลักษณะทั้งแบบปกติและแบบ optical flow จากนั้นทดสอบและบันทึกผล 
		\item ปรับเปลี่ยนพารามิเตอร์บางตัวของโมเดลเพื่อเปรียบเทียบประสิทธิภาพ
		\item เปรียบเทียบผลลัพธ์ของโมเดล
\end{enumerate}
\clearpage