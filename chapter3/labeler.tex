การออกแบบเครื่องมือสำหรับกำกับข้อมูลด้วยปัญญาประดิษฐ์ ผู้วิจัยได้เลือกใช้ library PyQt และภาษาไพธอนในการพัฒนา
เนื่องจาก PyQt นั้นเป็น library ที่มีผู้พัฒนาใช้กันอย่างแพร่หลาย จึงสะดวกในการศึกษา หาข้อมูลในการสร้างหรือแก้ไข
อีกทั้งยังเป็น library ที่สามารถพัฒนาด้วยภาษาไพธอนได้ และใช้งานง่าย สามารถปรับปรุงแก้ไขได้สะดวก

\subsection{เครื่องมือสำหรับกำกับข้อมูลด้วยปัญญาประดิษฐ์}
เครื่องมือสำหรับกำกับข้อมูลด้วยปัญญาประดิษฐ์ แบ่งการทำงานออกเป็น 4 กระบวนการทำงาน คือ Select, Detect, Track และ Label
เพื่อช่วยแบ่งเบาภาระของผู้พัฒนาในการสร้างชุดข้อมูลสำหรับสร้างโมเดลจากข้อมูลประเภทวิดีโอ โดยกระบวนการ Select
จะต้องสามารถตัดวิดีโอส่วนที่ไม่มีมนุษย์อยู่ออกจากวิดีโอได้ จากนั้นกระบวนการ Detect จะต้องหาตำแหน่งของมนุษย์ภายในวิดีโอได้
แล้วใช้กระบวนการ Track ติดตามการเคลื่อนไหวตำแหน่งต่อไปของมนุษย์ในเฟรมถัดๆไป
และกระบวนการสุดท้าย คือ Label นั้นต้องสามารถทำนายการกระทำพื้นฐานของมนุษย์ได้ เช่น ยืน เดิน นั่ง กินข้าว หรือ นอน เป็นต้น 
โดยทุกส่วนการทำงานมนุษย์ต้องสามารถทำงานร่วมกับปัญญาประดิษฐ์ได้
ดังรูปที่ \ref{fig:labeling_system}

\begin{figure}[!ht]
    \centering
    \includegraphics[width=0.5\textwidth]{chapter3/images/3_6/labelingToolOverview.png}
    \caption{กระบวนการหลักของเครื่องมือสำหรับกำกับข้อมูลด้วยปัญญาประดิษฐ์}
    \label{fig:labeling_system}
\end{figure}
\clearpage

\subsection*{โดยแต่ละกระบวนการจะมีรายละเอียดดังนี้}
\subsection*{หน้าต่าง Select}
กระบวนการ Select จะต้องสามารถรับวิดีโอเข้ามา แล้วตัดวิดีโอในช่วงที่ไม่มนุษย์อยู่ในเฟรมออกได้อัตโนมัติด้วยปัญญาประดิษฐ์
แต่เนื่องจากการประมวลผลทุกเฟรมในวิดีโอนั้นจะทำให้เสียเวลามากเกินไป จึงใช้วิธีการเลือกตัวอย่างเฟรมด้วยอัตราคงที่ (สามารถกำหนดได้)
ซึ่งเรียกเฟรมเหล่านี้ว่า คีย์เฟรม จากนั้นใช้ปัญญาประดิษฐ์ประมวลผลคีย์เฟรม เพื่อลดระยะเวลาในการประมวลผลลง และมนุษย์จะต้องสามารถแก้ไขข้อผิดพลาดของปัญญาประดิษฐ์ได้ เพื่อเพิ่มคุณภาพของชุดข้อมูล จึงออกแบบหน้าต่างได้ดังรูปที่ \ref{fig:SelectDraft}

\begin{figure}[!ht]
    \centering
    \includegraphics[width=1\textwidth]{chapter3/images/3_6/SelectDraft.png}
    \caption{หน้าต่าง Select ของเครื่องมือสำหรับกำกับข้อมูลด้วยปัญญาประดิษฐ์}
    \label{fig:SelectDraft}
\end{figure}
\clearpage
\begin{figure}[!ht]
    \centering
    \includegraphics[width=1\textwidth]{chapter3/images/3_6/SelectDraft_point.png}
    \caption{ตำแหน่งของแต่ละวิดเจ็ตในหน้าต่าง Select}
    \label{fig:SelectDraft_point}
\end{figure}
โดยที่แต่ละวิดเจ็ตตามหมายเลขที่กำหนดตามรูปที่ \ref{fig:SelectDraft_point} มีรายละเอียดดังนี้
\begin{enumerate}
	\setlength\itemsep{-0.25em}
	\item หมายเลข 1 คือปุ่มสำหรับเลือกไฟล์วิดีโอที่ต้องการจากในคอมพิวเตอร์เข้ามาในเครื่องมือกำกับคุณลักษณะด้วยปัญญาประดิษฐ์
    	\item หมายเลข 2 คือปุ่มสำหรับสั่งให้ระบบทำการสุ่มคีย์เฟรมขึ้นมา แล้วใช้ปัญญาประดิษฐ์ประมวลผลเพื่อแยกว่าคีย์เฟรมไหนมีคนอยู่ และคีย์เฟรมไหนไม่มีคนอยู่แบบอัตโนมัติ
    	\item หมายเลข 3 คือแถบเลื่อนเพื่อกำหนดความถี่ในการสุ่มคีย์เฟรม โดยจะมีช่วงอยู่ที่สุ่มหยิบทุกๆ 1 เฟรมต่อวินาที จนถึง อัตราเฟรมต่อวินาทีสูงสุดของวิดีโอที่รับเข้ามา
	\item หมายเลข 4 คือกล่องสำหรับแสดงชื่อวิดีโอที่รับเข้ามาในเครื่องมือกำกับคุณลักษณะด้วยปัญญาประดิษฐ์เพื่อเลือกเข้ามาใช้ในการประมวลผล
	\item หมายเลข 5 คือกล่องสำหรับแสดงว่าคีย์เฟรมใดมีมนุษย์อยู่ในเฟรม โดยที่ผู้ใช้งานสามารถตรวจสอบความถูกต้องและแก้ไขข้อผิดพลาดของปัญญาประดิษฐ์ได้
	\item หมายเลข 6 คือกล่องสำหรับแสดงว่าคีย์เฟรมใดไม่มีมนุษย์อยู่ในเฟรม โดยที่ผู้ใช้งานสามารถตรวจสอบความถูกต้องและแก้ไขข้อผิดพลาดของปัญญาประดิษฐ์ได้
	\item หมายเลข 7 คือหน้าต่างสำหรับแสดงเฟรมที่เลือกจากหมายเลข 5 หมายเลข 6 หรือหมายเลข 8
	\item หมายเลข 8 คือแถบเลื่อนสำหรับเลื่อนดูคีย์เฟรมทั้งหมดที่ระบบสร้างขึ้น
	\item หมายเลข 9 คือปุ่มสำหรับไปกระบวนการต่อไปหลังจากระบบประมวลผลเสร็จแล้ว
\end{enumerate}
\clearpage

\subsection*{หน้าต่าง Detect}
กระบวนการ Delect จะต้องสามารถรับคีย์เฟรมจากกระบวนการ Select มาประมวลผลด้วยปัญญาประดิษฐ์เพื่อหาตำแหน่งของมนุษย์ที่อยู่ในคีย์เฟรม 
แล้วสร้างกรอบสี่เหลี่ยมครอบบริเวณดังกล่าวได้ในแบบอัตโนมัติ เพื่อแบ่งเบาภาระผู้ใช้ในการที่ต้องสร้างกรอบสี่เหลี่ยมครอบตำแหน่งของมนุษย์ด้วยตัวเอง
และผู้ใช้ต้องสามารถสร้างหรือลบกรอบสี่เหลี่ยมได้ด้วยตัวเองสำหรับแก้ไขความผิดพลาดของปัญญาประดิษฐ์ เพื่อเพิ่มคุณภาพของชุดข้อมูล
จึงออกแบบหน้าต่างได้ดังรูปที่ \ref{fig:DetectDraft}
\begin{figure}[!ht]
    \centering
    \includegraphics[width=1\textwidth]{chapter3/images/3_6/DetectDraft.png}
    \caption{หน้าต่าง Detect ของเครื่องมือสำหรับกำกับข้อมูลด้วยปัญญาประดิษฐ์}
    \label{fig:DetectDraft}
\end{figure}
\clearpage
\begin{figure}[!ht]
    \centering
    \includegraphics[width=1\textwidth]{chapter3/images/3_6/DetectDraft_point.png}
    \caption{ตำแหน่งของแต่ละวิดเจ็ตในหน้าต่าง Detect}
    \label{fig:DelectDraft_point}
\end{figure}
โดยที่แต่ละวิดเจ็ตตามหมายเลขที่กำหนดตามรูปที่ \ref{fig:DelectDraft_point} มีรายละเอียดดังนี้
\begin{enumerate}
	\setlength\itemsep{-0.25em}
    \item หมายเลข 1 คือช่องสำหรับกดเพื่อเปลี่ยนระบบจากสร้างกรอบสี่เหลี่ยมในแบบแก้ไขด้วยตนเองเป็นลบกรอบสี่เหลี่ยมแทน
    \item หมายเลข 2 คือช่องสำหรับเลือกว่าจะใช้ระบบการทำงานแบบใด ระหว่างแบบทำงานอัตโนมัติและแบบปรับแก้ไขด้วยตนเอง
    \item หมายเลข 3 คือปุ่มสำหรับสั่งให้ระบบทำการตรวจหาตำแหน่งของมนุษย์ในคีย์เฟรมทั้งหมดแล้วสร้างกรอบสี่เหลี่ยมขึ้นมาครอบบริเวณที่กำหนด
	\item หมายเลข 4 คือกล่องสำหรับแสดงคีย์เฟรมทั้งหมด
	\item หมายเลข 5 คือหน้าต่างสำหรับแสดงเฟรมที่เลือกจากหมายเลข 4 หรือหมายเลข 6
	\item หมายเลข 6 คือแถบเลื่อนสำหรับเลื่อนดูคีย์เฟรมทั้งหมดที่มี เพื่อตรวจสอบความถูกต้องของปัญญาประดิษฐ์
	\item หมายเลข 7 คือปุ่มสำหรับไปกระบวนการต่อไปหลังจากระบบประมวลผลเสร็จแล้ว
\end{enumerate}
\clearpage

\subsection*{หน้าต่าง Track}
เนื่องจากกระบวนการ Detect นั้นจะทำเฉพาะในคีย์เฟรมทำให้ในเฟรมอื่นๆนอกเหนือจากนั้นจะไม่มีกรอบสี่เหลี่ยปรากฎอยู่
ดังนั้นกระบวนการ Track จึงต้องสามารถติดตามการเคลื่อนไหวตำแหน่งต่อไปของมนุษย์ และสร้างกรอบสี่เหลี่ยมขึ้นมาบนเฟรมระหว่างคีย์เฟรมทั้งหมดได้โดยอัตโนมัติ
เพื่อสร้างข้อมูลตำแหน่งของมนุษย์ในเฟรมเหล่านั้น สุดท้ายนี้ผู้ใช้ต้องสามารถสร้างหรือลบกรอบสี่เหลี่ยมได้ด้วยตัวเองสำหรับแก้ไขความผิดพลาดของอัลกอริทึม
จึงออกแบบหน้าต่างได้ดังรูปที่ \ref{fig:TrackDraft}
\begin{figure}[!ht]
    \centering
    \includegraphics[width=1\textwidth]{chapter3/images/3_6/TrackDraft.png}
    \caption{หน้าต่าง Track ของเครื่องมือสำหรับกำกับข้อมูลด้วยปัญญาประดิษฐ์}
    \label{fig:TrackDraft}
\end{figure}
\clearpage
\begin{figure}[!ht]
    \centering
    \includegraphics[width=1\textwidth]{chapter3/images/3_6/TrackDraft_point.png}
    \caption{ตำแหน่งของแต่ละวิดเจ็ตในหน้าต่าง Track}
    \label{fig:TrackDraft_point}
\end{figure}
โดยที่แต่ละวิดเจ็ตตามหมายเลขที่กำหนดตามรูปที่ \ref{fig:TrackDraft_point} มีรายละเอียดดังนี้
\begin{enumerate}
	\setlength\itemsep{-0.25em}
    \item หมายเลข 1 คือช่องสำหรับกดเพื่อเปลี่ยนระบบจากสร้างกรอบสี่เหลี่ยมในแบบแก้ไขด้วยตนเองเป็นลบกรอบสี่เหลี่ยมแทน
    \item หมายเลข 2 คือช่องสำหรับเลือกว่าจะใช้ระบบแบบใด ระหว่างแบบอัตโนมัติและแบบแก้ไขด้วยตนเอง
    \item หมายเลข 3 คือปุ่มสำหรับสั่งให้ระบบทำการตรวจหาตำแหน่งของมนุษย์ในเฟรมระหว่างคีย์เฟรมทั้งหมดแล้วสร้างกรอบสี่เหลี่ยมขึ้นมาครอบบริเวณที่กำหนด
	\item หมายเลข 4 คือกล่องสำหรับแสดงกรอบสี่เหลี่ยมทั้งหมดที่อยู่ในเฟรม
	\item หมายเลข 5 คือหน้าต่างสำหรับแสดงเฟรมที่เลือกจากหมายเลข 6
	\item หมายเลข 6 คือแถบเลื่อนสำหรับเลื่อนดูเฟรมทั้งหมดที่มี เพื่อตรวจสอบความถูกต้องของอัลกอริทึม
	\item หมายเลข 7 คือปุ่มสำหรับไปกระบวนการต่อไปหลังจากระบบประมวลผลเสร็จแล้ว
\end{enumerate}
\clearpage

\subsection*{หน้าต่าง Label}
กระบวนการ Label นั้นต้องสามารถทำนายว่าการกระทำของมนุษย์ที่อยู่ในแต่ละเฟรมว่าคืออะไรได้โดยอัตโนมัติด้วยปัญญาประดิษฐ์
และผู้ใช้จะต้องสามารถแก้ไขข้อผิดพลาดของปัญญาประดิษฐ์ได้หากมีการทำนายที่ผิดพลาดเกิดขึ้น
หรือถ้าหากผู้ใช้ต้องการเพิ่มการกระทำที่ไม่ได้มีอยู่ในชุดการกระทำพื้นฐานที่มีอยู่แล้วของปัญญาประดิษฐ์ ผู้ใช้ก็สามารถเพิ่มการกระทำนั้นเข้ามาได้
จึงออกแบบหน้าต่างได้ดังรูปที่ \ref{fig:ActionLabelDraft}
\begin{figure}[!ht]
    \centering
    \includegraphics[width=1\textwidth]{chapter3/images/3_6/ActionLabelDraft.png}
    \caption{หน้าต่าง Label ของเครื่องมือสำหรับกำกับข้อมูลด้วยปัญญาประดิษฐ์}
    \label{fig:ActionLabelDraft}
\end{figure}
\clearpage
\begin{figure}[!ht]
    \centering
    \includegraphics[width=1\textwidth]{chapter3/images/3_6/ActionLabelDraft_point.png}
    \caption{ตำแหน่งของแต่ละวิดเจ็ตในหน้าต่าง Label}
    \label{fig:ActiobLabelDraft_point}
\end{figure}
โดยที่แต่ละวิดเจ็ตตามหมายเลขที่กำหนดตามรูปที่ \ref{fig:TrackDraft_point} มีรายละเอียดดังนี้
\begin{enumerate}
	\setlength\itemsep{-0.25em}
    \item หมายเลข 1 คือช่องสำหรับเลือกว่าจะใช้ระบบแบบใด ระหว่างแบบอัตโนมัติและแบบแก้ไขด้วยตนเอง
    \item หมายเลข 2 คือปุ่มสำหรับสั่งให้ระบบทำนายการกระทำของมนุษย์ในทุกๆเฟรม
    \item หมายเลข 3 คือกล่องสำหรับแสดงกรอบสี่เหลี่ยมทั้งหมดที่อยู่ในเฟรมที่เลือก
	\item หมายเลข 4 คือกล่องสำหรับแสดงการกระทำของมนุษย์แต่ละคนที่อยู่ในเฟรมที่เลือก โดยจะเรียงลำดับคู่กับกรอบสี่เหลี่ยมที่อยู่ในช่องหมายเลข 3
    \item หมายเลข 5 คือกล่องสำหรับแสดงชุดการกระทำที่ปัญญาประดิษฐ์มีอยู่แล้ว ซึ่งในการทำงานแบบแก้ไขด้วยตนเองนั้น จะสามารถค้นหาการกระทำที่มีอยู่แล้วได้ 
    และหากคำที่ใส่เข้ามานั้นไม่มีอยู่ในชุดการกระทำก็จะเป็นการเพิ่มการกระทำนั้นเข้ามาแทน
	\item หมายเลข 6 คือหน้าต่างสำหรับแสดงเฟรมที่เลือกจากหมายเลข 7
	\item หมายเลข 7 คือแถบเลื่อนสำหรับเลื่อนดูเฟรมทั้งหมดที่มี เพื่อตรวจสอบความถูกต้องของปัญญาประดิษฐ์
	\item หมายเลข 8 คือปุ่มสำหรับสร้างไฟล์ xml ของทุกๆเฟรมสำหรับใช้ในการสร้างโมเดลโดยรายละเอียดข้อมูลภายในไฟล์ xml จะอยู่ในหัวข้อ \ref{sec:XMLInfo}
\end{enumerate}
\clearpage

\subsection*{รายละเอียดข้อมูลภายในไฟล์ xml}
\label{sec:XMLInfo}
ไฟล์ xml นั้นเป็นรูปแบบที่นิยมใช้ในการเก็บข้อมูลสำหรับการสร้างโมเดลประเภทตรวจจับวัตถุ
โดยจะเก็บข้อมูลในรูปแบบของ PASCAL VOC ที่นิยมใช้ในการสร้างโมเดลด้วย Tensorflow โดยภายในไฟล์จะมีข้อมูลดังรูปที่ \ref{fig:XMLFormat}
\begin{figure}[!ht]
    \centering
    \includegraphics[width=1\textwidth]{chapter3/images/3_6/XMLFormat.png}
    \caption{ตัวอย่างข้อมูลภายในไฟล์ xml}
    \label{fig:XMLFormat}
\end{figure}
โดยข้อมูลส่วนสำคัญของรูปแบบนี้นั้นจะถูกใส่หมายเลขกำกับไว้ซึ่งแต่ละหมายเลขนั้นหมายถึง
\begin{enumerate}
	\setlength\itemsep{-0.25em}
    \item หมายเลข 1 คือชื่อโฟลเดอร์ที่เก็บไฟล์รูปภาพที่เกี่ยวข้องกับไฟล์ xml นี้อยู่
    \item หมายเลข 2 คือชื่อไฟล์ที่เกี่ยวข้องกับไฟล์ xml นี้
    \item หมายเลข 3 คือเส้นทางในคอมพิวเตอร์ (directory path) ของไฟล์รูปภาพที่เกี่ยวข้องกับไฟล์ xml นี้
    \item หมายเลข 4 คือขนาดและมิติของรูปภาพ ซึ่งจะประกอบด้วยความกว้าง (width) ความยาว (height) และจำนวนช่องสี (depth) 
    โดยที่จำนวนช่องสีที่มีความลึก 3 มักจะหมายถึงภาพสี RGB และจำนวนช่องสีที่มีความลึก 2 จะหมายถึงภาพขาวดำ (gray scale)
	\item หมายเลข 5 คือ label ของวัตถุหรืออย่างอื่น ที่อยู่ในกรอบสี่เหลี่ยมที่ถูกกำหนดไว้ในส่วนของหมายเลข 6
	\item หมายเลข 6 คือ กรอบสี่เหลี่ยมที่ครอบวัตถุที่สนใจ เช่นมนุษย์ เป็นต้น
\end{enumerate}