\section{จลนศาสตร์ของหุ่นยนต์}
จลนศาสตร์ เป็นการศึกษาเกี่ยวกับตำแหน่ง (position), ทิศทางการหมุน (orientation) และการเคลื่อนที่ 
ทั้งเชิงเส้น (translation) และเชิงมุม (rotation) โดยไม่ได้คำนึงถึงแรงที่ก่อให้เกิดการเคลื่อนที่ 
โดยพื้นฐานแล้วหุ่นยนต์เป็นระบบพลวัต (dynamic) ที่เกี่ยวข้องกับการเคลื่อนที่ 
ซึ่งต้องคำนวณถึงแรงที่ก่อให้เกิดการเคลื่อนที่ด้วย
\par
หุ่นยนต์ฮิวมานอยด์ตรงส่วนขา เป็นหุ่นยนต์แบบอนุกรมประกอบขึ้นจากการต่อกันของก้านต่อต่าง ๆด้วยข้อต่อ
ไล่เรียงลำดับจากส่วนสะโพกถึงส่วนปลายเท้า ในลักษณะโครงสร้างแบบโซ่เปิด
ขาของหุ่นยนต์ฮิวมานอยด์จะสามารถเคลื่อนที่ได้ด้วยการขับเคลื่อนของตัวขับ ซึ่งมักติดตั้งอยู่ที่ข้อต่อ 
ซึ่งทำให้ท่าทางของขาถูกกำหนดได้ด้วยค่าตัวแปรของข้อต่อ หากต้องการที่จะรู้ว่าขาของหุ่นยนต์จะเคลื่อนที่
อย่างไรในปริภูมิสามมิติ จำเป็นที่จะต้องรู้ความสัมพันธ์ของตำแหน่งและทิศทางการหมุนของขาหุ่นยนต์ 
และตัวแปรของข้อต่อ
\par
ดังนั้นท่าทางของขาจะถูกกำหนดด้วยค่าตัวแปรของข้อต่อ เพื่อความสะดวกในการใช้งานจึงมักจะ
เขียนให้อยู่ในรูปของเวกเตอร์ เรียกว่าเวกเตอร์ของข้อต่อ (joint vector)

$\dot{q}=[\theta_1 ... \theta_n ]^T$

ปริภูมิของเวกเตอร์ของข้อต่อทั้งหมด จะเรียกว่าปริภูมิของข้อต่อ (joint space)
\par
บางครั้งในทางปฏิบัติ อาจจะมีกลไกบางอย่างเพื่อเปลี่ยนแปลงการเคลื่อนที่จากตัวขับไปยังข้อต่อ 
เพื่อประโยชน์ต่าง ๆ เช่น ทดแรง ทดรอบ ลดมวล นั่นทำให้ ตัวแปรข้อต่อไม่ใช่ตัวแปรที่เกิดจากตัวขับโดยตรง 
จึงมีการเขียนค่าของตัวแปรตัวขับให้อยู่ในรูปของเวกเตอร์ เรียกว่าเวกเตอร์ของตัวขับ (actuator vector)

$\dot{q}=[\theta_1 ... \theta_n ]^T$

ปริภูมิของเวกเตอร์ของตัวขับทั้งหมด จะเรียกว่าเวกเตอร์ของตัวขับ (actuator space)
\par
โดยทั่วไปแล้วการอธิบายส่วนปลายขาของหุ่นยนต์ซึ่งมีทั้งตำแหน่งและทิศทางการหมุน 
นิยมอธิบายด้วยเวกเตอร์ตำแหน่ง และมุมออยเลอร์ โดยเราสามารถเขียนให้อยู่ในรูปเวกเตอร์รวมได้

$\dot{q}=[\theta_1 ... \theta_n ]^T$

ปริภูมิของตำแหน่งและการหมุนนี้จะเรียกว่าปริภูมิของการทำงาน (task space)

จลนศาสตร์ของหุ่นยนต์จะแบ่งออกเป็น 2 ประเภท คือ
\begin{enumerate}[label=\thesection.\arabic*, leftmargin=1.5cm]
	\item จลนศาสตร์ไปข้างหน้า (Forward Kinematics)
จลนศาสตร์ไปข้างหน้าเป็นการวิเคราะห์หาฟังก์ชันของตำแหน่งและทิศทางการหมุน (task space) ในพจน์ที่มีตัวแปรเป็น ค่าของข้อต่อ (joint space)
    \item จลนศาสตร์ผกผัน (Inverse Kinematics)
จลนศาสตร์ผกผันเป็นการวิเคราะห์หาฟังก์ชันของค่าของข้อต่อ (joint space) ในพจน์ที่มีตัวแปรเป็นค่าของตำแหน่งและทิศทางการหมุน (task space)
\end{enumerate}

รูปที่ 2.4 แสดงความสัมพันธ์ระหว่างปริภูมิตัวขับ ปริภูมิข้อต่อ และปริภูมิการทำงาน โดยจลนศาสตร์ไปข้างหน้า


รูปที่ 2.5 แสดงความสัมพันธ์ระหว่างปริภูมิตัวขับ ปริภูมิข้อต่อ และปริภูมิการทำงาน โดยจลนศาสตร์ผกผัน