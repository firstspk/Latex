\subsection{ข้อมูลรายละเอียดประกอบการทดสอบ}
ชื่อวิดีโอ: Photographer beach photography

ความยาววิดีโอ: 15 วินาที

จำนวนเฟรมทั้งหมด: 374 เฟรม

อัตราเฟรมต่อวินาที: 24.9 เฟรมต่อวินาที

ความละเอียดของวิดีโอ: 1920 \texttimes 1080

ความละเอียดของวิดีโอที่ใช้ในการประมวลผลจริง: 1280 \texttimes 720

ขอบเขตอัตราส่วนร่วมของกรอบที่เหลี่ยมที่จะนับว่าการทำนายถูกต้อง: 80\% ขึ้นไป

\subsection{ทดสอบประสิทธิภาพ และความเร็วในการประมวลผล}
\begin{table}[!ht]
    \centering
    \begin{tabular}{|c|c|c|c|c|}
        \hline
        วิธีการทดสอบ & \multicolumn{2}{c|}{ความแม่นยำ (\%)} & \multicolumn{2}{c|}{ความเร็วในการประมวลผล (วินาที)}\\
        \hline
        \begin{tabular}{@{}l@{}}ใช้โมเดลปัญญาประดิษฐ์ YOLO-v3 320 \\ ประมวลผลทุกเฟรมในวิดีโอ\end{tabular} & 95 & - & 452 & -\\
        \hline
        \begin{tabular}{@{}l@{}}ใช้โมเดลปัญญาประดิษฐ์ YOLO-v3 320 \\ ประมวลผลทุกๆ N เฟรมในวิดีโอ \\แล้วใช้ระบบทำนายตำแหน่งต่อไปของ\\วัตถุในเฟรมระหว่างนั้น\end{tabular} & \multicolumn{1}{c}{} & \multicolumn{1}{c}{} & \multicolumn{1}{c}{} & \multicolumn{1}{c|}{}\\ 
        \hline    
        \multicolumn{1}{|l|}{N = 10} & 85 & -10 & 69 & -383\\
        \multicolumn{1}{|l|}{N = 20} & 80 & -15 & 41 & -411\\
        \multicolumn{1}{|l|}{N = 25} & 75 & -20 & 35 & -417\\
        \hline
    \end{tabular}
    \caption{ผลการทดสอบประสิทธิภาพของการตรวจจับกรอบสี่เหลี่ยมภายในวิดีโอ}
    \label{tab:trackEx}
\end{table}
จากตารางที่ \ref{tab:trackEx} ผู้วิจัยได้ทำการทดสอบความแม่นยำและความเร็วในการประมวลผลของการใช้โมเดลปัญญาประดิษฐ์ YOLO-v3 320 ประมวลผลทุกเฟรม 
แม้จะตั้งขอบเขตอัตราส่วนร่วมของกรอบที่เหลี่ยมที่จะนับว่าการทำนายถูกต้องสูงถึง 80\% แต่ความแม่นยำยังสูงถึง 95\% ใช้เวลาในการประมวลผล 452 วินาที 
เฉลี่ยเฟรมละ 1.2 วินาที ซึ่งถือเป็นความแม่นยำที่สูงมากเมื่อเทียบกับเวลาที่ใช้ในการประมวล

ต่อมาเป็นการทดสอบโดยใช้โมเดลปัญญาประดิษฐ์ประมวลผลเฉพาะบางเฟรมทุกๆช่วงหนึ่ง แล้วใช้ระบบทำนายตำแหน่งต่อไปของวัตถุในการสร้างกรอบสี่เหลี่ยมในเฟรมระหว่างนั้น เพื่อเพิ่มความเร็วในการประมวลผล 
โดยระยะที่ใช้ในการทดสอบคือ ทุกๆ 10 เฟรม 20 เฟรม และ 25 เฟรม ซึ่งจากผลการทดสอบนั้นพบว่าวิธีการนี้มีความแม่นแปรผกผันกับจำนวนเฟรมที่ใช้ในการประมวลผล (จำนวนเฟรมมากขึ้นจะทำให้ความแม่นยำต่ำลง) 
และความเร็วในการประมวลผลนั้นจะแปรผันตรงกับจำนวนเฟรมที่ใช้ในการประมวลผล (จำนวนเฟรมมากขึ้นจะทำให้ประมวลผลเร็วขึ้น) โดยที่การใช้ระยะประมวลผลเป็น 10 เฟรมนั้นใช้เวลาในการประมวลผลเพียง 69 วินาที 
น้อยกว่าการใช้โมเดลปัญญาประดิษฐ์ YOLO-v3 320 ประมวลผลทุกเฟรมถึง 383 วินาที ซึ่งเร็วกว่าถึง 6.5 เท่า และความแม่นยำลดลงมาเหลือ 85\% น้อยกว่าอยู่เพียง 10\% เท่านั้น ถือเป็นความแม่นยำที่สูงเมื่อเทียบกันด้วยระยะเวลาในการประมวลผล
ในขณะที่การใช้ระยะประมวลผล 20 เฟรมนั้นจะประมวลผลเร็วกว่าการใช้โมเดลปัญญาประดิษฐ์ YOLO-v3 320 ประมวลผลทุกเฟรมถึง 11 เท่า และมีความแม่นยำต่ำกว่า 15\%
และเมื่อใช้ระยะประมวลผล 25 เฟรมจะเร็วกว่าประมาณ 13 เท่า และความแม่นยำต่ำลงถึง 20\% 