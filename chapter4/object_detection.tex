\subsection{ข้อมูลรายละเอียดประกอบการทดสอบ}
จำนวนเฟรมทั้งหมด: 20 เฟรม

จำนวนมนุษย์ที่อยู่ในเฟรม : 0-5 คน

ความละเอียดรูปภาพ : 1280\texttimes 720 พิกเซล

ขอบเขตอัตราส่วนร่วมของกรอบที่เหลี่ยมที่จะนับว่าการทำนายถูกต้อง: 50\% ขึ้นไป


\subsection{ผลทดสอบประสิทธิภาพการทำงานของโมเดลปัญญาประดิษฐ์สำหรับการทำการตรวจจับภาพบุคคล}

ข้อมูลความแม่นยำของโมเดลปัญญาประดิษฐ์เมื่อทดสอบด้วยชุดข้อมูลของผู้วิจัย
\begin{table}[!ht]
	\centering
	\begin{tabular}{|c|c|c|}
			\hline 
			{โมเดลปัญญาประดิษฐ์}&{ความเร็วต่อรูปภาพ (มิลลิวินาที)}&{ความแม่นยำ (0.5 IOU)}			\\
			\hline
			SSD Mobilenet v1 ppn	 					& 63.82 		& 37.03			\\
			YOLO-v3 320							& 65.00		& 64.91		\\
			YOLO-v3 tiny							& 17.21		& 44.44			\\
			YOLO-v3 spp							& 65.40		& 70.30			\\	
			Faster RCNN inception v2					& 981.21		& 42.59		\\
		\hline
	\end{tabular}
	\caption{ข้อมูลผลการทำงานของโมเดลปัญญาประดิษฐ์สำหรับการทำการตรวจจับภาพบุคคล}
    \label{tab:origina_detectEx}
\end{table}
\\
จากตารางที่ \ref{tab:trackEx} ผู้วิจัยได้ทำการทดสอบความแม่นยำและความเร็วในการประมวลผลของโมเดลปัญญาประดิษฐ์สำหรับการทำการตรวจจับภาพบุคคล 
พบว่าโมเดลปัญญาประดิษฐ์ที่มีความแม่นยำมากที่สุดคือ YOLO-v3 spp และโมเดลปัญญาประดิษฐ์ที่มีความเร็วในการทำนายต่อรูปภาพเร็วที่สุดคือ YOLO-v3 tiny จากผลลัพธ์การทดลองดังกล่าว 
ทุกโมเดลปัญญาประดิษฐ์ยกเว้น Faster RCNN มีความเร็วในการประมวลผลต่อรูปภาพที่ผู้วิจัยสามารถรับได้ (ไม่เกิน 1 วินาที) 
ดังนั้นผู้วิจัยจึงเลือกโมเดลปัญญาประดิษฐ์ที่จะใช้จากความแม่นยำมากที่สุด คือ YOLO-v3 spp