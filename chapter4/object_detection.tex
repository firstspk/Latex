\subsection{ข้อมูลรายละเอียดประกอบการทดสอบ}
จำนวนเฟรมทั้งหมด: 20 เฟรม

จำนวนมนุษย์ที่อยู่ในเฟรม : 0-5 คน

ความละเอียดรูปภาพ : 1280 \texttimes 720  พิกเซล

ขอบเขตอัตราส่วนร่วมของกรอบที่เหลี่ยมที่จะนับว่าการทำนายถูกต้อง: 50\% ขึ้นไป

\subsubsection*{ข้อมูลความเที่ยงตรงของโมเดลปัญญาประดิษฐ์}
\begin{table}[!ht]
    \centering
	\begin{tabular}{|c|c|c|}
			\hline
			{}&{ความเร็วต่อรูปภาพ(มิลลิวินาที)}&{ความเที่ยงตรง (0.5 IOU mAP)}			\\
			\hline
			SSD Mobilenet v1 ppn	 		& 26				& 20														\\
			YOLOv3-320				& 22				& 51.5				\\	
			YOLOv3-tiny				& 4.5				& 33.1				\\
			YOLOv3-spp				& 50				& 60.6				\\	
			Faster rcnn inceptrionv2		& 58				& 28		\\
		\hline
	\end{tabular}
	\caption{ข้อมูลผลการทำงานของโมเดลปัญญาประดิษฐ์สำหรับการทำการตรวจจับภาพบุคคล อ้างอิงข้อมูลจากเว็บไซต์ของ yolo}
    	\label{tab:origina_detectEx}
\end{table}


\subsection{ทดสอบประสิทธิภาพการทำงานของโมเดลปัญญาประดิษฐ์สำหรับการทำการตรวจจับภาพบุคคล}
\subsubsection*{ข้อมูลความแม่นยำของโมเดลปัญญาประดิษฐ์เมื่อทดสอบด้วยชุดข้อมูลของผู้วิจัย}
\begin{table}[!ht]
	\centering
	\begin{tabular}{|c|c|c|}
			\hline 
			{}&{ความเร็วต่อรูปภาพ(มิลลิวินาที)}&{ความแม่นยำ (0.5 IOU)}			\\
			\hline
			SSD Mobilenet v1 ppn	 					& 63 			& 37			\\
			YOLOv3-320							& 65			& 64.9		\\
			YOLOv3-tiny							& 17			& 44.4			\\
			YOLOv3-spp							& 65.4			& 70.3			\\	
			Faster rcnn inceptrionv2					& 981		& 42.5		\\
		\hline
	\end{tabular}
	\caption{ข้อมูลผลการทำงานของโมเดลปัญญาประดิษฐ์สำหรับการทำการตรวจจับภาพบุคคลหลังจากทำการทดลอง}
    	\label{tab:origina_detectEx}
\end{table}
