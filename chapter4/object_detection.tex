\subsection{ข้อมูลรายละเอียดประกอบการทดสอบ}
จำนวนเฟรมทั้งหมด: 20 เฟรม

จำนวนมนุษย์ที่อยู่ในเฟรม : 0-5 คน

ความละเอียดรูปภาพ : 1280 \texttimes 720  พิกเซล

ขอบเขตอัตราส่วนร่วมของกรอบที่เหลี่ยมที่จะนับว่าการทำนายถูกต้อง: 50\% ขึ้นไป


\subsection{ผลทดสอบประสิทธิภาพการทำงานของโมเดลปัญญาประดิษฐ์สำหรับการทำการตรวจจับภาพบุคคล}
%ข้อมูลผลการทำงานของโมเดลปัญญาประดิษฐ์สำหรับการทำการตรวจจับภาพบุคคล อ้างอิงข้อมูลจากเว็บไซต์ของ YOLO
%\begin{table}[!ht]
%	\begin{tabular}{|c|c|c|}
%		\hline
%		{}&{ความเร็วต่อรูปภาพ (มิลลิวินาที)}&{ความแม่นยำ (0.5 IoU mAP)}			\\
%		\hline
%		SSD Mobilenet v1 ppn	 		& 26				& 20														\\
%		YOLO-v3 320				& 22				& 51.5				\\	
%		YOLO-v3 tiny				& 4.5				& 33.1				\\
%		YOLO-v3 spp				& 50				& 60.6				\\	
%		Faster RCNN inceptrion v2		& 58				& 28		\\
%	\hline
%	\end{tabular}
%	\caption{ข้อมูลผลการทำงานของโมเดลปัญญาประดิษฐ์สำหรับการทำการตรวจจับภาพบุคคล อ้างอิงข้อมูลจากเว็บไซต์ของ YOLO}
%\end{table}

ข้อมูลความแม่นยำของโมเดลปัญญาประดิษฐ์เมื่อทดสอบด้วยชุดข้อมูลของผู้วิจัย
\begin{table}[!ht]
	\centering
	\begin{tabular}{|c|c|c|}
			\hline 
			{}&{ความเร็วต่อรูปภาพ(มิลลิวินาที)}&{ความแม่นยำ (0.5 IOU)}			\\
			\hline
			SSD Mobilenet v1 ppn	 					& 63.82 		& 37.03			\\
			YOLOv3-320							& 65.00		& 64.91		\\
			YOLOv3-tiny							& 17.21		& 44.44			\\
			YOLOv3-spp							& 65.40		& 70.30			\\	
			Faster rcnn inceptrionv2					& 981.21		& 42.59		\\
		\hline
	\end{tabular}
	\caption{ข้อมูลผลการทำงานของโมเดลปัญญาประดิษฐ์สำหรับการทำการตรวจจับภาพบุคคล}
    \label{tab:origina_detectEx}
\end{table}
\\
จากตารางที่ \ref{tab:trackEx} ผู้วิจัยได้ทำการทดสอบความแม่นยำและความเร็วในการประมวลผลของโมเดลปัญญาประดิษฐ์สำหรับการทำการตรวจจับภาพบุคคล พบว่าโมเดลปัญญาประดิษฐ์ที่มีความแม่นยำมากที่สุดคือ YOLOv3-spp และ โมเดลปัญญาประดิษฐ์ที่มีความเร็วในการทำนายต่อรูปภาพเร็วที่สุดคือ YOLO-tinys