\subsection{ทดสอบประสิทธิภาพการทำงานของโมเดลปัญญาประดิษฐ์ I3D สร้างด้วยชุดข้อมูลที่ผู้วิจัยสร้างขึ้น โดยใช้ชุดข้อมูลที่ผู้วิจัยสร้างขึ้นในการทดสอบ}
คุณลักษณะที่ใช้ในการสร้างโมเดลปัญญาประดิษฐ์ I3D ที่ผู้วิจัยได้พัฒนาเป็นชุดของเฟรมที่เป็นภาพสีปกติ (RGB) และชุดของเฟรมที่เป็น optical flow (OF) โดยใช้ PASCAL mAP, Top@1 และ Top@3
ในการวัดผลความแม่นยำของแต่ละโมเดล ซึ่งมีรายละเอียดและพารามิเตอร์ดังนี้
\begin{enumerate}
	\item โมเดลที่ 1
	\begin{enumerate}
		\item Learning rate: 0.001
		\item Dropout: 0.36
		\item Optimizer: Momentum
		\item Optimizer parameter:
		\begin{enumerate}
			\item Momentum: 0.8
		\end{enumerate}
	\end{enumerate}
	\item โมเดลที่ 2
	\begin{enumerate}
		\item Learning rate: 0.001
		\item Dropout: 0.36
		\item Optimizer: Adam
		\item Optimizer parameters:
		\begin{enumerate}
			\item $\beta_1$: 0.9
			\item $\beta_2$: 0.999
			\item $\epsilon$: $10^{-8}$
		\end{enumerate}
	\end{enumerate}
\end{enumerate}

\clearpage
\subsubsection{การทดสอบประสิทธิภาพของโมเดลปัญญาประดิษฐ์ I3D ที่ใช้ชุดข้อมูลที่เป็นแบบภาพสีปกติ}
\begin{table}[!ht]
	\centering
	\begin{tabular}{|c|c|c|c|}
			\hline
			{โมเดล}	&	{PASCAL mAP}	&	{Top@1}	&	{Top@3}\\
			\hline
			RGB โมเดลที่ 1	& 0.564	& 0.482	& 0.641	\\
			RGB โมเดลที่ 2	& 0.356	& 0.265	& 0.487	\\
			\hline
	\end{tabular}
\caption{ผลการทดสอบความแม่นยำของโมเดลปัญญาประดิษฐ์ที่ผู้วิจัยสร้างขึ้นโดยใช้ชุดข้อมูลที่ผู้วิจัยสร้างขึ้นแบบภาพสีปกติ}
\label{tab:I3D_RGB_performance}
\end{table}

\subsubsection{การทดสอบประสิทธิภาพของโมเดลปัญญาประดิษฐ์ I3D ที่ใช้ชุดข้อมูลที่เป็นแบบ optical flow}
\begin{table}[!ht]
	\centering
	\begin{tabular}{|c|c|c|c|}
			\hline
			{โมเดล}	&	{PASCAL mAP}	&	{Top@1}	&	{Top@3}\\
			\hline
			Optical flow โมเดลที่ 1	& 0.748	& 0.737	& 0.908	\\
			Optical flow โมเดลที่ 2	& 0.777	& 0.759	& 0.959	\\
			\hline
	\end{tabular}
\caption{ผลการทดสอบความแม่นยำของโมเดลปัญญาประดิษฐ์ที่ผู้วิจัยสร้างขึ้นโดยใช้ชุดข้อมูลที่ผู้วิจัยสร้างขึ้นแบบ optical flow}
\label{tab:I3D_OF_performance}
\end{table}


\subsubsection{ตารางแสดงการเปรียบเทียบค่า average precision (AP) ของทุกการกระทำของแต่ละโมเดล}
\label{sec:I3D_AP}
\begin{table}[!ht]
	\centering
	\begin{tabular}{|c|c|c|c|c|}
			\hline
			{Label} & {RGB โมเดลที่ 1} & {RGB โมเดลที่ 2} & {OF โมเดลที่ 1} & {OF โมเดลที่ 1}\\
			\hline
			Play phone  & 0.239 & 0.011 & 0.552 & 0.599	\\
			Eat			& 0.282	& 0.058	& 0.787	& 0.839	\\
			Sit		 	& 0.450 & 0.113 & 0.795 & 0.799	\\
			Sleep		& 0.800	& 0.655	& 0.704	& 0.628	\\
			Stand		& 0.865	& 0.822	& 0.731	& 0.797	\\
			Walk		& 0.748	& 0.476	& 0.921	& 1.000	\\
			\hline
	\end{tabular}
\caption{ตารางเปรียบเทียบค่า AP ของทุกการกระทำของแต่ละโมเดล}
\label{tab:I3D_RGB_OF_AP}
\end{table}
จะเห็นได้ว่าโมเดลที่ถูกสร้างจากชุดข้อมูลแบบ optical flow นั้นมีความแม่นยำสูงกว่าโมเดลที่ใช้ข้อมูลภาพสีแบบปกติในการสร้าง แต่ถ้าหากพิจารณาจากค่า average precision (AP) 
ตามตารางที่ \ref{tab:I3D_RGB_OF_AP} โมเดลที่ผ่านการสร้างด้วยชุดข้อมูลแบบภาพสีปกตินั้นจะมีประสิทธิภาพสูงเมื่อเป็นการกระทำที่แทบจะไม่มีการเคลื่อนไหวคือ นอนและยืน 
ในขณะที่โมเดลที่ผ่านการสร้างด้วยชุดข้อมูลแบบ optical flow นั้นมีความแม่นยำสูงกว่ามากในการกระทำที่มีการเคลื่อนไหว จึงสามารถกล่าวได้ว่าโมเดลแบบ optical flow 
นั้นเหมาะสำหรับใช้ในการจำแนกการกระทำที่มีการเคลื่อนไหว และโมเดลแบบภาพสีปกติเหมาะสำหรับใช้ในการจำแนกการกระทำที่แทบจะไม่มีการเคลื่อนไหว

\subsubsection{การทดสอบประสิทธิภาพของโมเดลปัญญาประดิษฐ์ I3D ที่ใช้ชุดข้อมูลทั้งสองแบบ (RGB + optical flow)}
จากผลการวิเคราะห์ในหัวข้อที่ \ref{sec:I3D_AP} ทำให้ผู้วิจัยสนใจนำโมเดลทั้งสองแบบมาใช้ร่วมกันในการจำแนกการกระทำ โดยจะใช้การรวมผลลัพธ์ความน่าจะเป็นที่ถูกคูณด้วยอัตราส่วนความน่าเชื่อถือ
หมายความว่าในการกระทำ $l$ โมเดล $M_1$ ทำนายผลออกมาว่ามีความเป็นไปได้ว่าชุดข้อมูลนี้จะเป็นการกระทำดังกล่าว $P_l^{M_1}$ 
และโมเดล $M_2$ ทำนายผลออกมาว่ามีความเป็นไปได้ว่าชุดข้อมูลนี้จะเป็นการกระทำดังกล่าว $P_l^{M_2}$ สมมติว่าให้อัตราส่วนความน่าเชื่อถือของโมเดล $M_1$ เป็น $W_1$
และของโมเดล $M_2$ เป็น $W_2$ (โดยที่ $W_1 + W_2 = 1$) ทำให้สามารถเขียนสมการความเป็นไปได้ว่าชุดข้อมูลนี้จะเป็นการกระทำ $l$ ได้ดังนี้
\begin{equation}
	P_l = \frac{W_1 P_l^{M_1} + W_2 P_l^{M_2}}{2}
\end{equation}
ซึ่งผลลัพธ์จากการทดลองเป็นดังตารางที่ \ref{tab:I3D_RGB_OF_performance} และผลลัพธ์ค่า average precision เป็นดังตารางที่ \ref{tab:I3D_RGB_OF_AP}
\begin{table}[!ht]
	\centering
	\begin{tabular}{|c|c|c|c|}
			\hline
			{โมเดล (อัตราส่วนความน่าเชื่อถือ)}	&	{PASCAL mAP}	&	{Top@1}	&	{Top@3}\\
			\hline
			RGB โมเดลที่ 1 + OF โมเดลที่ 1 (50:50)	& 0.765	& 0.740	& 0.903	\\
			RGB โมเดลที่ 1 + OF โมเดลที่ 2 (50:50)	& 0.823	& 0.806	& 0.945	\\
			RGB โมเดลที่ 2 + OF โมเดลที่ 1 (50:50)	& 0.706	& 0.679	& 0.865	\\
			RGB โมเดลที่ 2 + OF โมเดลที่ 2 (50:50)	& 0.804	& 0.780	& 0.931	\\
			\hline
	\end{tabular}
\caption{ผลการทดสอบความแม่นยำเมื่อนำโมเดลแบบภาพสีปกติ และ optical flow มาใช้ร่วมกัน}
\label{tab:I3D_RGB_OF_performance}
\end{table}
จากผลลัพธ์ในตารางที่ \ref{tab:I3D_RGB_OF_performance} จะเห็นว่าเมื่อนำโมเดลทั้งสองแบบมาใช้งานร่วมกันด้วยอัตราส่วนความน่าเชื่อถือ 50:50 ทำให้ประสิทธิภาพโดยรวมสูงขึ้น
ซึ่งหากลองพิจารณาจากประสิทธิภาพในแต่ละการกระทำ ด้วยค่า average precision จะได้ผลลัพธ์ดังตารางที่ \ref{tab:I3D_RGB_w_OF_AP} 
จากตารางจะเห็นว่าประสิทธิภาพในการจำแนกการกระทำนั้นมีความครอบคลุมมากกว่าการใช้โมเดลแบบเดียวในการทำนายผล แต่การเล่นโทรศัพท์ (play phone) ยังคงมีประสิทธิภาพที่ต่ำมาก
ทั้งนี้ผู้วิจัยได้ตรวจสอบแล้วพบว่าการทำนายที่ผิดส่วนมากจะเป็นการนั่ง (sit) ผู้วิจัยจึงคาดว่าเกิดจากการที่การเคลื่อนไหวของการเล่นโทรศัพท์นั้นมีความใกล้เคียงกับการนั่งมาก
จึงทำให้การโมเดลสามารถแยกความแตกต่างออกได้ยาก
\begin{table}[!ht]
	\centering
	\begin{tabular}{|c|c|c|c|c|}
		\hline
		{Label} & {RGB\textsubscript{1}+OF\textsubscript{1}} & {RGB\textsubscript{1}+OF\textsubscript{2}} & {RGB\textsubscript{2}+OF\textsubscript{1}} & 
		{RGB\textsubscript{2}+OF\textsubscript{2}}\\
		\hline
		Play phone  & 0.544	& 0.581	& 0.405	& 0.574	\\
		Eat			& 0.785	& 0.836	& 0.827	& 0.847	\\
		Sit		 	& 0.705	& 0.791	& 0.685	& 0.772	\\
		Sleep		& 0.852	& 0.963	& 0.825	& 0.981	\\
		Stand		& 0.883	& 0.881	& 0.840	& 0.809	\\
		Walk		& 0.824	& 0.884	& 0.653	& 0.841	\\
		\hline
	\end{tabular}
\caption{ตารางเปรียบเทียบค่า AP ของทุกการกระทำของแต่ละโมเดล}
\label{tab:I3D_RGB_w_OF_AP}
\end{table}
