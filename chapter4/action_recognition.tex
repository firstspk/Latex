\subsection{ทดสอบประสิทธิภาพการทำงานของโมเดลปัญญาประดิษฐ์ที่เคยถูกสร้างผ่าน AVA เทียบผลลัพธ์กับแหล่งอ้างอิง ได้ผลการทดลองดังตารางต่อไปนี้}
\begin{table}[!ht]
	\centering
	\begin{tabular}{|c|c|c|}
			\hline
			{}&{ความเร็วต่อรูปภาพ(มิลลิวินาที)}&{ความแม่นยำ (PASCAL mAP)}			\\
			\hline
			แหล่งอ้างอิง	 					& 93.00		& 11.00				\\
			ผลการทดสอบของผู้วิจัย				& 85.35  		& 6.8	0				\\
			\hline
	\end{tabular}
\caption{ผลการทดสอบความแม่นยำของโมเดลปัญญาประดิษฐ์เทียบผลลัพธ์กับแหล่งอ้างอิง}
\label{tab: Compare PASCAL mAP with source}
\end{table}
ความเร็วของต่อรูปภาพทางผู้วิจัยได้ใช้กราฟิกการ์ด GTX 2080 Ti ในการทดสอบซึ่งจะให้ความเร็วอยู่ที่ 0.085 วินาที ซึ่งทางแหล่งอ้างอิงนั้นใช้กราฟิกการ์ด Nvidia GeForce GTX TITAN X 
ในส่วนของค่าความแม่นยำที่ไม่เท่ากัน คาดว่าจะเป็นเพราะการประมวลผลของกราฟิกการ์ดของรุ่นที่ต่างกันและสเปคของเครื่องคอมพิวเตอร์จึงทำให้ค่า mAP ที่ออกมาไม่เท่ากัน

\subsection{ผลการทดสอบประสิทธิภาพการทำงานของโมเดลปัญญาประดิษฐ์ที่เคยถูกสร้างผ่าน AVA และ ใช้ชุดข้อมูลที่ผู้วิจัยสร้างขึ้น ในการทดสอบและเทียบผลลัพธ์กับแหล่งอ้างอิง}
\begin{table}[!ht]
	\centering
	\begin{tabular}{|c|c|c|}
			\hline
			{}&{ความเร็วต่อรูปภาพ(มิลลิวินาที)}&{ความแม่นยำ (PASCAL mAP)}			\\
			\hline
			แหล่งอ้างอิง	 					& 93.00			& 11.00			\\
			ผลการทดสอบของผู้วิจัย				& 63.11			& 15.18			\\
			\hline
	\end{tabular}
\caption{ผลการทดสอบความแม่นยำของโมเดลปัญญาประดิษฐ์ เมื่อใช้กับชุดข้อมูลที่ผู้วิจัยสร้างขึ้น}
\label{tab: Compare PASCAL mAP with dataset created by the researcher}
\end{table}
ความเร็วของต่อรูปภาพทางผู้วิจัยได้ใช้กราฟิกการ์ด Tesla V100-SXM2 ในการทดสอบซึ่งจะให้ความเร็วอยู่ที่ 0.063 วินาที ซึ่งทางแหล่งอ้างอิงนั้นใช้กราฟิกการ์ด Nvidia GeForce GTX TITAN X ซึ่งการนำโมเดลปัญญาประดิษฐ์ AVA มาใช้ในการทดสอบกับชุดข้อมูลทดสอบที่ทางผู้วิจัยสร้างขึ้น ซึ่งผลลัพธ์ที่ได้ออกมานั้นสูงขึ้นหากเทียบการทดลองจากแหล่งอ้างอิง ทำให้การตั่งสมมติฐานที่ตั่งไว้ตอนแรกนั้นไม่ถูกต้อง ถึงแม้ว่าชุดข้อมูลที่ทางผู้วิจัยได้สร้างขึ้นจะมีการตัดหมวดหมู่ที่ไม่ได้ใช่ออกไป จึงทำให้ผู้วิจัยสรุปได้ว่าการตัดหมวดหมู่ของชุดข้อมูลออกไป นั้นไม่ได้ส่งผลต่อประสิทธิภาพการทำงานของโมเดลปัญญาประดิษฐ์ AVA

\subsection{ทดสอบประสิทธิภาพการทำงานของโมเดลปัญญาประดิษฐ์ ResNet50 ที่เคยถูกสร้างผ่านชุดข้อมูลสำหรับการสร้างโมเดลปัญญาประดิษฐ์ที่ผู้วิจัยสร้างขึ้น และใช้ชุดข้อมูลทดสอบที่ผู้วิจัยสร้างขึ้น ในการทดสอบและเทียบผลลัพธ์การทดสอบก่อนหน้า}
ในส่วนนี้จะเป็นการทดสอบโดยใช้โครงสร้างโมเดลปัญญาประดิษฐ์เป็น ResNet50 โดยจะมีการใช้ชุดข้อมูลที่ผู้วิจัยสร้างขึ้นซึ่งเป็นชุดข้อมูลทดสอบเดียวกับที่ทางผู้วิจัยใช้ในการทดสอบโมเดลปัญญาประดิษฐ์ AVA มาใช้ในการทำสอบครั้งนี้ด้วย โดยชุดข้อมูลที่นำมาใช้สำหรับการสร้างโมเดลปัญญาประดิษฐ์จะมีอยู่ 2 ชุดข้อมูล ซึ่งได้แก่ goggle dataset v1 และ goggle dataset v2 ชุดข้อมูลทั้งสองอันนี้จะแตกต่างกันตรงที่ goggle dataset v2 นั้นเป็นชุดข้อมูลเกิดจากการที่ทางผู้วิจัยได้เข้าไปลบในส่วนที่มีการกำกับข้อมูลภาพผิดและมีการสุ่มตัวอย่างของข้อมูลออกมาเพื่อลดโอกาสที่จะเกิด overfitting ของข้อมูล โดยจำนวนชุดข้อมูลของ goggle dataset v1 ที่ใช้สำหรับการสร้างโมเดลจะมี 213,061 ภาพ และในส่วนของ goggle dataset v2 จะมีจำนวนชุดข้อมูล 120,177 ภาพ
\begin{table}[!ht]
	\centering
	\begin{tabular}{|c|c|c|}
			\hline
			{ชุดข้อมูล}&{ความเร็วต่อรูปภาพ(มิลลิวินาที)}&{ความแม่นยำ (PASCAL mAP)}			\\
			\hline
			goggle dataset v1 ResNet50			& 2.78			& 27.92				\\
			goggle dataset v2 ResNet50			& 2.52			& 29.35				\\
			\hline
	\end{tabular}
\caption{ผลการทดสอบความแม่นยำของโมเดลปัญญาประดิษฐ์ที่ผู้วิจัยสร้างขึ้น ใช้กับชุดข้อมูลที่ผู้วิจัยสร้างขึ้น}
\label{tab: Test PASCAL mAP of dataset created by the researcher}
\end{table}
จากตาราง \ref{tab: Test PASCAL mAP of dataset created by the researcher} จะเป็นการทดสอบโมเดลปัญญาประดิษฐ์ที่สร้างจากชุดข้อมูล goggle dataset v1 และ goggle dataset v2 โดยมีการตั้งค่าตัวแปรต่าง ๆ ดังนี้ ขนาดของรูปภาพจะอยู่ที่ 224x224 พิกเซล pooling ของ ResNet50 ใช้ average pooling และ activation function ใช้ softmax และใ้ช้ epoch 50 มีการใช้ค่า batch size เท่ากับ 16 โดยใช้ optimize เป็น stochastic gradient descent (sgd) จากการทดลองจะเป็นได้ว่าความเร็วของต่อรูปภาพทางผู้วิจัยได้ใช้กราฟิกการ์ด Tesla V100-SXM2 แต่ความเร็วนั้นเร็วกว่าตอนที่ทดสอบด้วยโมเดลปัญญาประดิษฐ์ AVA เป็นอย่างมาก เนื่องจากโมเดลปัญญาประดิษฐ์ของ AVA นั้นจะมีการทำในส่วนของการหากรอบสี่เหลี่ยมรอบตัวมนุษย์ด้วย ในขณะที่โมเดลปัญญาประดิษฐ์ของผู้วิจัยนั้นจะไม่มีหากรอบสี่เหลี่ยมรอบตัวมนุษย์ เพราะจะมีการนำโมเดลปัญญาปรดิษฐ์ของ YOLO-v3 spp มาหากรอบสี่เหลี่ยมรอบตัวมนุษย์ตั้งแต่แรกแล้ว ในส่วนของค่า mAP ที่ได้ออกมานั้น goggle dataset v2 มีค่า mAP มากกว่าแต่ต่างกันไม่มาก อาจจะเป็นไปได้ว่าเพราะจำนวนข้อมูลที่น้อยกว่า goggle dataset v1 ต่อมาผู้วิจัยได้นำชุดข้อมูลของ goggle dataset v2 มาทำการสร้างโมเดลประดิษฐ์อีกรอบ โดยรอบนี่จะมีการนำ weight ของ ImageNet มาใช้ร่วมกันในการสร้างโมเดลปัญญาประดิษฐ์

\begin{table}[!ht]
	\centering
	\begin{tabular}{|c|c|c|}
			\hline
			{}&{ความเร็วต่อรูปภาพ(มิลลิวินาที)}&{ความแม่นยำ (PASCAL mAP)}			\\
			\hline
			ResNet50-ImageNet			& 2.51			& 45.39				\\
			\hline
	\end{tabular}
\caption{ผลการทดสอบความแม่นยำของโมเดลปัญญาประดิษฐ์ที่ผู้วิจัยสร้างขึ้นโดยใช้ weight จาก ImageNet ใช้กับชุดข้อมูลที่ผู้วิจัยสร้างขึ้น}
\label{tab: Test PASCAL mAP of dataset created by the researcher and pretrain weight imagenet}
\end{table}
จากตาราง \ref{tab: Test PASCAL mAP of dataset created by the researcher and pretrain weight imagenet} เมื่อทางผู้วิจัยได้นำ weight ของ ImageNet มาช่วยในการสร้างโมเดลทำให้ประสิทธิภาพของโมเดลที่ได้ออกมานั้นสูงขึ้นมากเมื่อเทียบกับการทดสอบโมเดลปัญญาประดิษฐ์ของผู้วิจัยก่อนหน้านี้ ต่อมาจะนำโมเดลปัญญาประดิษฐ์ในรอบนี้มาสร้างใหม่อีกครั้งโดยใช้ชุดข้อมูล goggle dataset v2 และจะเพิ่มในส่วนของการทำ scaling ข้อมูลก่อนได้แก่การทำ normalization , centering และการทำ standardize

\begin{table}[!ht]
	\centering
	\begin{tabular}{|c|c|c|}
			\hline
			{}&{ความเร็วต่อรูปภาพ(มิลลิวินาที)}&{ความแม่นยำ (PASCAL mAP)}			\\
			\hline
			ResNet50-ImageNet	 Normalization				& 2.51			& 44.93				\\
			ResNet50-ImageNet	 Centering Featurewise		& 2.40			& 45.67				\\
			ResNet50-ImageNet	 Centering Samplewise		& 2.49			& 46.61				\\
			ResNet50-ImageNet	 Standardize Featurewise		& 2.48			& 40.65				\\
			ResNet50-ImageNet	 Standardize Samplewise		& 2.49			& 43.21				\\
			\hline
	\end{tabular}
\caption{ผลการทดสอบความแม่นยำของโมเดลปัญญาประดิษฐ์ที่ผู้วิจัยสร้างขึ้นโดยใช้ weight จาก ImageNet และการทำ scaling ใช้กับชุดข้อมูลที่ผู้วิจัยสร้างขึ้น}
\label{tab: Test PASCAL mAP of dataset created by the researcher with pretrain weight imagenet and scaling}
\end{table}

จากตาราง \ref{tab: Test PASCAL mAP of dataset created by the researcher with pretrain weight imagenet and scaling} จะเป็นการทดลอง scaling ด้วยรูปแบบต่าง อย่างการทำ normalization จะเป็นทำให้ค่าในพิกเซลอยู่ในช่วง 0 ถึง 1 การทำ centering คือการลบค่าในพิกเซลด้วยค่าเฉลี่ยของพิกเซล โดยจะแบ่งออกเป็น 2 แบบได้แก่ featurewise และ samplewise โดย featurewise จะเป็นการหาค่าเฉลี่ยพิกเซลจากทุกรูปในชุดข้อมูลแล้วนำมาลบออกในแต่ละพักเซลของรูป ส่วนของ samplewise จะไม่มีการไปยุ่งเกี่ยวกับรูปอื่น คือจะหาค่าเฉลี่ยของพิกเซลของรูปนั้น ๆ และนำค่าพิกเซลในรูปนั้น ๆ ลบออกด้วยค่าเฉลี่ย ต่อมาการทำ standardize คือการหารค่าในพิกเซลด้วยค่า standard deviation ของพิกเซลในรูป ซึ่งจะแบ่งออกเป็น 2 แบบ ได้แก่ featurewise และ samplewise เหมือนกับ centering โดย featurewise จะหาค่า standard deviation ของทุกพิกเซลในชุดข้อมูลแล้วนำมาหารในแต่ละพิกเซลของรูป ส่วนของ samplewise จะเป็นการหาค่า standard deviation ของรูปนั้น ๆ มาหารกับทุกพิกเซลในรูปนั้น ๆ จากการทำลองจะทำให้เห็นว่าโมเดลปัญญาประดิษฐ์ที่มีประสิทธิภาพสูงที่สุดคือโมเดลปัญญาประดิษฐ์ ResNet50-ImageNet	 Centering Samplewise


