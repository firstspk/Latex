% ************************** Thesis Abbreviations **************************
\begin{abbreviations}
    \noindent
    \begin{tabular*}{\textwidth}{@{}p{0.3\textwidth}p{0.8\textwidth}@{}}
	Accuracy     &   ความถูกต้อง\\
        Action classification  &  การจำแนกการกระทำ\\
        AI assisted labeling tool  &   เครื่องมือกำกับคุณลักษณะด้วยปัญญาประดิษฐ์\\
	Algorithm   &     อัลกอริทึม\\
        Amazon mechanical turk (AMT) & ตลาดแรงงาน\\
	Architecture  &  โครงสร้างของโมเดลปัญญาประดิษฐ์\\
	Artificial intelligence (AI) &           ปัญญาประดิษฐ์\\
        Auditory  &      เสียง\\
	Blacklist    &    บัญชีดำ\\
        Binary feature vector & เวกเตอร์คุณลักษณะฐานสอง\\
	Bounding box  &  กรอบสี่เหลี่ยม\\
        Class        &    หมวดหมู่\\
        Dataset   &      ชุดข้อมูล\\
	Directory path &   เส้นทางในคอมพิวเตอร์\\
        Extract feature & การสกัดคุณลักษณะ \\
        Feature     &    คุณลักษณะ\\
        Feature map  &  ผังคุณลักษณะ\\
        Feature vector  &  เวกเตอร์คุณลักษณะ\\
        Filter & ตัวกรอง\\
        Fourier transform &  การแปลงฟูรีเยร์\\
        Framerate per second (FPS) & อัตราเฟรมต่อวินาที \\
        Functional requirements & ความต้องการเชิงการใช้งาน\\
        Gradient    &    เกรเดียน\\
        Gray scale     &   ภาพขาวดำ\\
        Ground-truth &   คำตอบของชุดข้อมูล\\        
	Ground-truth tubes & กรอบสี่เหลี่ยมจริงในช่วงของเฟรมที่ต่อกัน\\
        Height     &   ความสูง\\
	Human working memory & ช่วงเวลาที่มนุษย์เข้าใจเหตุการณ์ที่เกิดขึ้น\\
	Hungarian distance & อัลกอริทึมสำหรับการหาข้อเสนอที่ดีที่สุด\\
        Image processing   & การประมวลผลภาพ\\
        Image understanding & การทำความเข้าใจภาพด้วยปัญญาประดิษฐ์\\
        Interaction with object  &  ปฏิสัมพันธ์กับวัตถุ\\
        Interaction with people & ปฏิสัมพันธ์กับบุคคล\\
	Intersection over union (IoU) & อัตราส่วนร่วมของกรอบสี่เหลี่ยม\\
	Inverse fourier transform & การแปลงฟูรีเยร์ผกผัน\\
	Kernel & เคอร์เนล\\
	Keyframe & คีย์เฟรม\\
	Label & คำกำกับคุณลักษณะของภาพ\\
	Labeling tool & เครื่องมือกำกับคุณลักษณะ\\
	Layer & ชั้นการทำงานของโมเดลปัญญาประดิษฐ์\\
	Linked detection tubes & กรอบสี่เหลี่ยมที่ทำนายขึ้นมาในช่วงที่เฟรมต่อกัน\\
	Machine learning model & โมเดลปัญญาประดิษฐ์\\
	Mode auto & การทำงานแบบอัตโนมัติ\\
	Mode manual & การทำงานแบบทำด้วยตัวเอง\\
	Neural network & โครงข่ายประสาทเทียม\\
	Non-functional requirements & ความต้องการเชิงวิศวกรรม\\
	Object detection & ระบบตรวจจับวัตถุ\\
	Object tracker & ระบบติดตามการเคลื่อนไหวของวัตถุ\\
	Open source & ชิ้นงานสาธารณะ\\
	Person detection & ระบบตรวจจับมนุษย์\\
	Person embedding & ใช้โครงขายประสาทเทียมในการหาคุณลักษณะ/ฟีเจอร์ขนส่งและใช้เมทริกซ์ในการหาความสัมพันธ์ของแต่ละคน\\
	Person re-identification & การระบุตัวตนของบุคคล\\
	Person tracker & ระบบติดตามการเคลื่อนไหวของมนุษย์\\
	Pixel & พิกเซล\\
	Pose & ท่าทาง\\
	Predict & การทำนาย\\
	Region of interest (ROI) & พื้นที่ที่สนใจ\\
	Spatial & ภาพนิ่ง\\
	Temporal & การเคลื่อนไหว\\
	Test set & ชุดข้อมูลสำหรับการทดสอบ\\
        Threshold & เกณฑ์สำหรับการตัดสินหรือแบ่ง\\
	Train set & ชุดข้อมูลสำหรับสร้างโมเดลปัญญาประดิษฐ์\\
	User interface & หน้าต่างของติดต่อกับผู้ใช้\\
	Valid set & ชุดข้อมูลสำหรับตรวจคำตอบ\\
	Vector & เวกเตอร์\\
	Video analytic & การประมวลผลวิดีโอ\\
	Video theme & สาระสำคัญของวิดีโอ\\
	Video understanding & การทำความเข้าใจวิดีโอด้วยปัญญาประดิษฐ์\\
	widget & วิดเจ็ต \\
	Width & ความกว้าง\\
	Whitelist & บัญชีขาว\\
    \end{tabular*}
\end{abbreviations}