\chapter{บทนำ}
%%%%%%%%%%%%%%%%%%%%%%%%%%%%%%%%%%%%%%%%%%%%%%%%%%%%%%%%%%%%%%%%%%%%%%%%%%%%%%%
\section{ที่มาและความสำคัญ}
บริษัท เพอเซ็ปทรา ดำเนินธุรกิจเกี่ยวกับด้าน artificial intelligence service โดยลูกค้านั้นมีความต้องการที่จะให้ทางบริษัทสร้างปัญญาประดิษฐ์(artificial intelligence) เพื่อนำไปใช้งานหรือแก้ปัญหาที่ต่างกันออกไป ทำให้การสร้างปัญญาประดิษฐ์ (artificial intelligence) เพื่อตอบสนองความต้องการของลูกค้าเหล่านั้นต้องมีชุดข้อมูลที่เหมาะสมกับปัญหานั้นๆ เช่น ร้านขายของแห่งหนึ่งต้องการรู้ว่าในแต่ละวันมีลูกค้าเดินเข้าร้านกี่คน เป็นผู้ชายกี่คน เป็นผู้หญิงกี่คน เป็นต้น ซึ่งการจะได้ข้อมูลที่เหมาะสมกับงานนั้น ต้องใช้มนุษย์ในการสร้างขึ้นมาโดยการเก็บข้อมูลวิดีโอ และสร้าง label สำหรับใช้ในการสร้างโมเดล machine learning ด้วยตัวเอง ถ้าหากมีวิดีโอเป็นจำนวนมาก การที่จะใช้มนุษย์ในการสร้าง label นั้นอาจจะต้องใช้มนุษย์เป็นจำนวนมาก หรือ ก่อให้เกิดภาระแก่มนุษย์ อีกทั้งการสร้าง label 

นั้นเป็นงานที่ลำบาก และน่าเบื่อ
ทางคณะผู้วิจัยจึงมีความต้องการที่จะออกแบบ และพัฒนา video analytics platform ที่มีเครื่องมือในการสร้าง label สำหรับวิดีโอ เพื่อช่วยแบ่งเบาภาระของผู้พัฒนาในการสร้าง label เพื่อนำไปสร้างโมเดล machine learning สำหรับใช้แก้ปัญหาที่ลูกค้าต้องการ โดยโครงการสหกิจนี้เน้นศึกษาการวิเคราะห์และจดจำการกระทำของมนุษย์จากภาพเคลื่อนไหวเป็นหลัก

%%%%%%%%%%%%%%%%%%%%%%%%%%%%%%%%%%%%%%%%%%%%%%%%%%%%%%%%%%%%%%%%%%%%%%%%%%%%%%%
\section{วัตถุประสงค์}
\begin{enumerate}
	\setlength\itemsep{-0.25em}
	\item เพื่อออกแบบ และ สร้างระบบที่สามารถตรวจจับมนุษย์ และจดจำการกระทำพื้นฐานของมนุษย์ภายในสำนักงาน ประกอบด้วย ยืน นั่ง ใช้คอมพิวเตอร์ เล่นโทรศัพท์ เดิน กินข้าว โดยใช้ปัญญาประดิษฐ์มาประมวลผลกับวิดีโอ
	\item เพื่อพัฒนาเครื่องมือในการทำ video labeling ในการสร้างข้อมูลที่ใช้สร้างโมเดลจากวิดีโอ ให้สามารถทำได้ง่าย และ มีประสิทธิภาพที่สูงกว่าเครื่องมือตัวอื่นในปัจจุบัน
\end{enumerate}

%%%%%%%%%%%%%%%%%%%%%%%%%%%%%%%%%%%%%%%%%%%%%%%%%%%%%%%%%%%%%%%%%%%%%%%%%%%%%%%

\section{ประโยชน์ที่คาดว่าจะได้รับ}
\begin{enumerate}
	\setlength\itemsep{-0.25em}
	\item พัฒนาเครื่องมือในการทำ labeling โดยมี artificial intelligence เข้ามาช่วย ที่สามารถสร้าง label ที่สามารถนำไปใช้สร้างโมเดล machine learning ได้
	\item พัฒนาต้นแบบของ video analytics platform ที่สามารถรับวิดีโอเข้ามาในระบบแล้วสร้างรายงานเกี่ยวกับกิจกรรมของมนุษย์ในวิดีโอได้
	\item สร้างและทดสอบโมเดลสำหรับทำ action recognition อย่างน้อย 2 โมเดล
\end{enumerate}
\clearpage

%%%%%%%%%%%%%%%%%%%%%%%%%%%%%%%%%%%%%%%%%%%%%%%%%%%%%%%%%%%%%%%%%%%%%%%%%%%%%%%
\section{ขอบเขตการดำเนินงาน}
\begin{enumerate}
	\setlength\itemsep{-0.25em}
	\item Labeling tool สามารถตัดวิดิโอเฉพาะในช่วงเวลาที่มีมนุษย์อยู่ได้อัตโนมัติ
	\item Labeling tool สามารถระบุตำแหน่งได้ว่ามนุษย์แต่ละคนในวิดีโออยู่ตรงส่วนใดของวิดีโอและสามารถระบุการกระทำของมนุษย์ในวิดีโอได้ ประกอบด้วยกระทำได้แก่ ยืน นั่ง ใช้คอมพิวเตอร์ เล่นโทรศัพท์ เดิน กินข้าว 
	\item Label ผลลัพธ์ที่ได้จาก labeling tool ต้องสามารถนำไปใช้ในการสร้างโมเดลต่อได้
	\item พัฒนา Labeling tool ด้วยภาษา Python
	\item พัฒนา Labeling tool ที่สามารถให้มนุษย์ทำงานแก้ไขได้ เมื่อระบบอัตโนมัติทำงานผิดพลาด
	\item สร้างโมเดลสำหรับการทำ action recognition อย่างน้อย 2 โมเดลที่สามารถระบุการกระทำของมนุษย์ตามที่กำหนดไว้ได้ เพื่อนำไปใช้ใน video analytics platform
	\item Video analytics platform ต้องสามารถนำวิดีโอมาวิเคราะห์ข้อมูลการกระทำและตำแหน่งของมนุษย์แต่ละคนได้ แล้วนำข้อมูลเหล่านั้นไปสร้างรายงานออกมาได้
	\item ความละเอียดอย่างต่ำของวิดีโอต้องมากกว่า 640 x 480 (ยาว x สูง)
	\item วิดีโอจะต้องมีเฟรมเรท (fps) อย่างต่ำ 24 fps
\end{enumerate}

%%%%%%%%%%%%%%%%%%%%%%%%%%%%%%%%%%%%%%%%%%%%%%%%%%%%%%%%%%%%%%%%%%%%%%%%%%%%%%%
\section{ภาพรวมของระบบและขั้นตอนการดำเนินงาน}
งานวิจัยนี้การดำเนินงานวิจัยถูกแบ่งออกเป็นสองส่วน คือ ส่วนที่หนึ่งส่วนเครื่องมือสำหรับการเตรียมชุดข้อมูล (dataset) เป็นส่วนที่ทำเครื่องมือสำหรับช่วยผู้พัฒนาในการสร้างชุดข้อมูล และส่วนที่สองนำชุดข้อมูลไปสร้างโมเดล
\subsection*{ศึกษาค้นคว้าเอกสารและงานวิจัยที่เกี่ยวข้อง}
\begin{enumerate}\setlength\itemsep{-0.25em}
	\item ศึกษาเกี่ยวกับการวิเคราะห์ผลวิดิโอ (video analytics)
	\item ศึกษาเกี่ยวกับชุดข้อมูลสำหรับการวิเคราะห์ผลวิดิโอ
	\item ศึกษาเกี่ยวกับโมเดลใช้ในการวิเคราะห์ผลวิดิโอ
	\item ศึกษาเครื่องมือที่ใช้สำหรับช่วยสร้างชุดข้อมูล
\end{enumerate}
\subsection*{ส่วนเครื่องมือสำหรับการเตรียมชุดข้อมูล (dataset)}
\begin{enumerate}\setlength\itemsep{-0.25em}
	\item ออกแบบหน้าต่างของแอพพลิเคชั่น
	\item สร้างระบบของแอพพลิเคชั่น
	\item ทดสอบการทำงานของแอพพลิเคชั่น
\end{enumerate}
\subsection*{2) ส่วนนำชุดข้อมูลไปสร้างโมเดล}
\begin{enumerate}\setlength\itemsep{-0.25em}
	\item สร้างชุดข้อมูลสำหรับสร้างโมเดล
	\item สร้างโมเดลสำหรับการทำนายการกระทำของมนุษย์
	\item ทดสอบการทำงานของโมเดล
\end{enumerate}
\clearpage
%%%%%%%%%%%%%%%%%%%%%%%%%%%%%%%%%%%%%%%%%%%%%%%%%%%%%%%%%%%%%%%%%%%%%%%%%%%%%%%
\section{Gantt chart}แผนการทำงานของคณะวิจัยจะแบ่งออกเป็น 3 ช่วง ซึ่งในแต่ละช่วงจะมี task ดังนี้
\begin{enumerate}
	\item Phase 1 :  ศึกษาข้อมูลเกี่ยวกับการทำวิเคราะห์วิดีโอ(video analytics) และเครื่องมือที่ใช้ในการทำโครงการวิจัย
		\begin{enumerate}\setlength\itemsep{-0.25em}
			\item task 1 	ศึกษา และ ทดลองใช้เทคนิคที่เกี่ยวข้องกับการวิเคราะห์วิดีโอ(video analytics)
			\item task2	รวบรวมชุดข้อมูลสำหรับการทำ human action recognition
		\end{enumerate}
	\item Phase 2 :  สร้างและออกแบบแอพพลิเคชั่นโดยจะประกอบด้วยงานในส่วนหลักๆ คือ
		\begin{enumerate}\setlength\itemsep{-0.25em}
			\item task3	โมเดลสำหรับวิเคราะห์ผลการกระทำของมนุษย์
			\item task4	หน้าต่าง UI
		\end{enumerate}
	\item Phase 3 : ปรับปรุงและแก้ไขปัญหา ทำให้แอพพลเคชั่นสามารถนำไปใช้งานได้จริง
\end{enumerate}

%%%%%%%%%%%%%%%%%%%%%%%%%%%%%%%%%%%%%%%%%%%%%%%%%%%%%%%%%%%%%%%%%%%%%%%%%%%%%%%
\section{Milestones}จะมีการกำหนด milestones ทั้งหมด 3 ครั้ง คือ
\begin{enumerate}
	\item Milestone  1  วันที่ 12  มิถุนายน  2562
		\begin{enumerate}\setlength\itemsep{-0.25em}
			\item ศึกษางานวิจัยที่เกี่ยวข้องกับการวิเคราะห์วิดีโอ(video analytics) ต่าง ๆ  ได้แก่ Youtube-8M , Moment in time , AVA
		\end{enumerate}
	\item Milestone  2  วันที่ 14  สิงหาคม 2562
		\begin{enumerate}\setlength\itemsep{-0.25em}
			\item สร้างหน้าต่าง UI ครบสมบูรณ์
			\item ความคืบหน้าของโมเดลมากกว่า 50 \%-symbol
			\item ฟังก์ชั่นการทำงานของแอพพลิเคชั่นสามารถทำงานได้มากกว่า 50 \%-symbol
		\end{enumerate}
	\item Milestone  3  วันที่ 15  ตุลาคม 2562
		\begin{enumerate}\setlength\itemsep{-0.25em}
			\item นาย ปฐมพงศ์ สินธุ์งาม \\หน้าที่รับผิดชอบ สร้างและทดสอบโมเดล I3D และ ทำโปรแกรมในส่วน Tracking
			\item นาย ศุภกร เบญจวิกรัย \\หน้าที่รับผิดชอบ รวบรวมฟังก์ชั่นต่างๆของแอพพลิเคชั่น และ ทำแอพพลิคชั่นในส่วน Select, Detection
			\item นาย อุกฤษฎ์ เลิศวรรณาการ \\หน้าที่รับผิดชอบ สร้างและทดสอบโมเดล Resnet-50และทำโปรแกรมในส่วน Person reid 
		\end{enumerate}
\end{enumerate}



