\chapter{บทนำ}
%%%%%%%%%%%%%%%%%%%%%%%%%%%%%%%%%%%%%%%%%%%%%%%%%%%%%%%%%%%%%%%%%%%%%%%%%%%%%%%
\section{ที่มาและความสำคัญ}
บริษัท เพอเซ็ปทรา ดำเนินธุรกิจเกี่ยวกับการให้บริการและคำปรึกษาเกี่ยวกับปัญญาประดิษฐ์ (artificial intelligence)
เนื่องจากปัจจุบันนั้นความสามารถและประสิทธิภาพของปัญญาประดิษฐ์มีความก้าวหน้าขึ้นจนสามารถก้าวข้ามความสามารถของมนุษย์ในงานหลายๆประเภท
ทำให้ลูกค้านั้นมีความต้องการที่จะให้ทางบริษัทสร้างปัญญาประดิษฐ์เพื่อนำไปใช้งานหรือแก้ปัญหาที่ต่างกันออกไป เช่น ใช้ปัญญาประดิษฐ์มาช่วยประมวลผลภาพจากกล้องวงจรปิด เพื่อหาบุคคลที่มีท่าทางน่าสงสัย เป็นต้น
ซึ่งการจะสร้างปัญญาประดิษฐ์ที่เหมาะสมกับการแก้ปัญหาเหล่านั้น จำเป็นต้องมีชุดข้อมูล (dataset) ที่เหมาะสม บางครั้งอาจต้องใช้มนุษย์ในการสร้างขึ้นมาโดยการเก็บข้อมูลวิดีโอ 
และลงมือสร้างชุดข้อมูลจากวิดีโอที่ได้ด้วยตัวเอง หนึ่งในปัจจัยสำคัญในการพัฒนาโมเดลปัญญาประดิษฐ์ (mahcine lerning model) ให้มีประสิทธิภาพสูงคือจำนวนข้อมูล
ซึ่งหากมีจำนวนวิดีโอเป็นจำนวนมาก การใช้มนุษย์ในการสร้างชุดข้อมูลนั้นอาจจะต้องใช้มนุษย์เป็นจำนวนมาก และใช้เวลานาน 

ทางคณะผู้วิจัยจึงมีความต้องการที่จะออกแบบและสร้างต้นแบบของเครื่องมือกำกับคุณลักษณะด้วยปัญญาประดิษฐ์ (AI-assisted video labeling tool) สำหรับสร้างชุดข้อมูลจากวิดีโอ 
เพื่อช่วยแบ่งเบาภาระของผู้พัฒนาในการสร้างชุดข้อมูลสำหรับการพัฒนาโมเดลปัญญาประดิษฐ์ในการแก้ปัญหาที่ลูกค้าต้องการ 
โดยโครงการสหกิจนี้จะศึกษาเกี่ยวกับการวิเคราะห์และจำแนกการกระทำของมนุษย์ (action classification) ภายในสำนักงานจากวิดีโอเป็นหลัก

%%%%%%%%%%%%%%%%%%%%%%%%%%%%%%%%%%%%%%%%%%%%%%%%%%%%%%%%%%%%%%%%%%%%%%%%%%%%%%%
\section{วัตถุประสงค์}
\begin{enumerate}
	\setlength\itemsep{-0.25em}
	\item เพื่อสร้างต้นแบบของเครื่องมือกำกับคุณลักษณะด้วยปัญญาประดิษฐ์ ที่มนุษย์และปัญญาประดิษฐ์ทำงานร่วมกันเพื่อสร้างชุดข้อมูลในการนำไปพัฒนาปัญญาประดิษฐ์อื่นๆที่เหมาะสมกับปัญหาที่ต้องการ
	\item เพื่อออกแบบและสร้างต้นแบบของระบบประมวลผลวิดีโอ (video analytics) ที่สามารถตรวจจับมนุษย์และจำแนกการกระทำพื้นฐานของมนุษย์ภายในสำนักงาน ประกอบด้วย ยืน นั่ง เดิน เล่นโทรศัพท์ กินข้าว นอน โดยใช้ปัญญาประดิษฐ์
	\item เพื่อสร้างเครื่องมือที่สามารถสร้างชุดข้อมูลสำหรับการจำแนกการกระทำของมนุษย์ให้สามารถใช้งานได้ง่าย สะดวกสบายมากขึ้น และมีประสิทธิภาพที่สูงกว่าเครื่องมือที่เปิดให้ใช้งานสาธารณะ (open source) ตัวอื่นในปัจจุบัน
\end{enumerate}

%%%%%%%%%%%%%%%%%%%%%%%%%%%%%%%%%%%%%%%%%%%%%%%%%%%%%%%%%%%%%%%%%%%%%%%%%%%%%%%

\section{ประโยชน์ที่คาดว่าจะได้รับ}
\begin{enumerate}
	\setlength\itemsep{-0.25em}
	\item เพิ่มความสะดวกในการสร้างชุดข้อมูลสำหรับพัฒนาโมเดลปัญญาประดิษฐ์จากวิดีโอ
	\item ต้นแบบระบบประมวลผลวิดีโอที่สามารถจำแนกการกระทำของมนุษย์
\end{enumerate}
\clearpage

%%%%%%%%%%%%%%%%%%%%%%%%%%%%%%%%%%%%%%%%%%%%%%%%%%%%%%%%%%%%%%%%%%%%%%%%%%%%%%%
\section{ขอบเขตการดำเนินงาน}
\begin{enumerate}
	\setlength\itemsep{-0.25em}
	\item สร้างต้นแบบของเครื่องมือกำกับคุณลักษณะด้วยปัญญาประดิษฐ์ โดยระบบจะประกอบไปด้วยสี่ส่วนดังนี้
	\begin{enumerate}
		\setlength\itemsep{-0.25em}
		\item หน้าต่างของติดต่อกับผู้ใช้ (user interface)
		\item ระบบตรวจจับมนุษย์ในภาพ (person detection)
		\item ระบบติดตามการเคลื่อนไหวของมนุษย์ในวิดีโอ (person tracker)
		\item ระบบจำแนกการกระทำของมนุษย์ ซึ่งประกอบไปด้วย ยืน นั่ง เดิน เล่นโทรศัพท์ กินข้าว นอน
	\end{enumerate}
	\item ทดสอบโมเดลปัญญาประดิษฐ์สำหรับจำแนกการกระทำของมนุษย์กับชุดข้อมูลที่ได้จากเครื่องมือกำกับคุณลักษณะด้วยปัญญาประดิษฐ์ เพื่อทดสอบว่าชุดข้อมูลที่ได้สามารถใช้งานจริงได้หรือไม่
	\item พัฒนาโมเดลปัญญาประดิษฐ์สำหรับจำแนกการกระทำของมนุษย์ภายในสำนักงานอย่างน้อย 2 โมเดล
\end{enumerate}

%%%%%%%%%%%%%%%%%%%%%%%%%%%%%%%%%%%%%%%%%%%%%%%%%%%%%%%%%%%%%%%%%%%%%%%%%%%%%%%
\section{ขั้นตอนการดำเนินงาน}
การดำเนินงานวิจัยถูกแบ่งออกเป็นสามส่วน โดยส่วนแรกคือการศึกษาเทคโนโลยีในปัจจุบันเพื่อหาความเป็นไปได้และกำหนดขอบเขตของงาน 
ส่วนที่สองคือออกแบบและสร้างเครื่องมือกำกับคุณลักษณะด้วยปัญญาประดิษฐ์ เพื่อช่วยผู้พัฒนาในการสร้างชุดข้อมูล 
และส่วนที่สุดท้ายคือการนำชุดข้อมูลที่ได้จากการใช้เครื่องมือกำกับคุณลักษณะด้วยปัญญาประดิษฐ์ไปพัฒนาโมเดลปัญญาประดิษฐ์สำหรับการจำแนกการกระทำของมนุษย์ภายในสำนักงาน
\subsection*{ศึกษาค้นคว้าเอกสารและงานวิจัยที่เกี่ยวข้อง}
\begin{enumerate}\setlength\itemsep{-0.25em}
	\item ศึกษาเกี่ยวกับการประมวลผลวิดีโอ
	\item ศึกษาเกี่ยวกับชุดข้อมูลสำหรับการประมวลผลวิดีโอ
	\item ศึกษาเกี่ยวกับโมเดลปัญญาประดิษฐ์ที่ใช้ในการประมวลผลวิดีโอ
	\item ศึกษาเครื่องมือที่ใช้ในการช่วยสร้างชุดข้อมูลจากวิดีโอ (labeling tool)
\end{enumerate}
\subsection*{เครื่องมือกำกับคุณลักษณะด้วยปัญญาประดิษฐ์}
\begin{enumerate}\setlength\itemsep{-0.25em}
	\item ออกแบบและสร้างหน้าต่างของเครื่องมือกำกับคุณลักษณะด้วยปัญญาประดิษฐ์
	\item ออกแบบและสร้างระบบของเครื่องมือกำกับคุณลักษณะด้วยปัญญาประดิษฐ์
	\item ทดสอบและปรับปรุงการทำงานของเครื่องมือกำกับคุณลักษณะด้วยปัญญาประดิษฐ์
\end{enumerate}
\subsection*{โมเดลปัญญาประดิษฐ์สำหรับการจำแนกการกระทำของมนุษย์ภายในสำนักงาน}
\begin{enumerate}\setlength\itemsep{-0.25em}
	\item สร้างชุดข้อมูลสำหรับสร้างโมเดลปัญญาประดิษฐ์จากเครื่องมือกำกับคุณลักษณะด้วยปัญญาประดิษฐ์
	\item สร้างโมเดลปัญญาประดิษฐ์สำหรับการจำแนกการกระทำของมนุษย์ภายในสำนักงาน
	\item ทดสอบโมเดลปัญญาประดิษฐ์สำหรับการจำแนกการกระทำของมนุษย์ภายในสำนักงาน
\end{enumerate}
\clearpage
\section*{แผนการดำเนินงาน}
\begin{figure}[!ht]
	\centering
	\includegraphics[width=0.9\textwidth]{chapter1/gantt_chart.jpg}
	\caption{แผนการดำเนินงาน}
	\label{tab:ganttchart}
\end{figure}
\subsection*{รายละเอียดแผนดำเนินงาน}
\begin{enumerate}\setlength\itemsep{-0.25em}
	\item ช่วงการทำงานที่ 1 : ศึกษาและรวบรวมข้อมูล
		\begin{enumerate}
			\item 3 มิถุนายน - 15 กรกฎาคม 2562\\ศึกษาข้อมูลเกี่ยวกับการวิเคราะห์วิดีโอด้วยปัญญาประดิษฐ์และการสร้างชุดข้อมูลสำหรับจำแนกการกระทำของมนุษย์		
		\end{enumerate}	
	\item ช่วงการทำงานที่ 2 : ออกแบบและพัฒนาเครื่องมือสำหรับกำกับคุณลักษณะด้วยปัญญาประดิษฐ์
		\begin{enumerate}
			\item 16 กรกฎาคม - 1 สิงหาคม 2562\\ออกแบบและสร้างหน้าต่างการทำงานของเครื่องมือกำกับคุณลักษณะด้วยปัญญาประดิษฐ์
			\item 16 กรกฎาคม - 15 สิงหาคม 2562\\ออกแบบและสร้างระบบการทำงานของเครื่องมือกำกับคุณลักษณะด้วยปัญญาประดิษฐ์
			\item 1 สิงหาคม - 15 กันยายน 2562\\รวมระบบเข้ากับหน้าต่างการทำงานของเครื่องมือกำกับคุณลักษณะด้วยปัญญาประดิษฐ์
			\item 16 สิงหาคม - 15 ตุลาคม 2562\\ปรับปรุงระบบและหน้าต่างการทำงานของเครื่องมือกำกับคุณลักษณะด้วยปัญญาประดิษฐ์
		\end{enumerate}
	\item ช่วงการทำงานที่ 3 : สร้างและทดสอบโมเดลปัญญาประดิษฐ์สำหรับจำแนกการกระทำของมนุษย์
		\begin{enumerate}
			\item 16 ตุลาคม - 11 พฤศจิกายน 2562\\สร้างชุดข้อมูลสำหรับโมเดลปัญญาประดิษฐ์จากเครื่องมือกำกับคุณลักษณะด้วยปัญญาประดิษฐ์
			\item 21 ตุลาคม - 22 พฤศจิกายน 2562\\สร้างและทดสอบโมเดลปัญญาประดิษฐ์สำหรับจำแนกการกระทำของมนุษย์กับชุดข้อมูลที่ได้จากเครื่องมือกำกับคุณลักษณะด้วยปัญญาประดิษฐ์		
		\end{enumerate}	
\end{enumerate}
\clearpage