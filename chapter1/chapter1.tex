\chapter{บทนำ}
%%%%%%%%%%%%%%%%%%%%%%%%%%%%%%%%%%%%%%%%%%%%%%%%%%%%%%%%%%%%%%%%%%%%%%%%%%%%%%%
\section{ที่มาและความสำคัญ}
บริษัท เพอเซ็ปทรา ดำเนินธุรกิจเกี่ยวกับการให้บริการและคำปรึกษาเกี่ยวกับปัญญาประดิษฐ์(artificial intelligence)
โดยลูกค้านั้นมีความต้องการที่จะให้ทางบริษัทสร้างปัญญาประดิษฐ์เพื่อนำไปใช้งานหรือแก้ปัญหาที่ต่างกันออกไป 
ทำให้การสร้างปัญญาประดิษฐ์ ซึ่งการที่จะตอบสนองความต้องการของลูกค้าเหล่านั้น 
จำเป็นต้องมีชุดข้อมูล(dataset)ที่เหมาะสมกับปัญหานั้นๆ เช่น ร้านขายของแห่งหนึ่งต้องการรู้ว่าในแต่ละวันมีลูกค้าเดินเข้าร้านกี่คน เป็นผู้ชายกี่คน เป็นผู้หญิงกี่คน เป็นต้น
ซึ่งการจะสร้างปัญญาประดิษฐ์ที่เหมาะสมกับการแก้ปัญหานี้ ก็จำเป็นต้องมีชุดข้อมูลที่เหมาะสม บางครั้งอาจต้องใช้มนุษย์ในการสร้างขึ้นมาโดยการเก็บข้อมูลวิดีโอ 
และลงมือสร้างชุดข้อมูลจากวิดีโอที่ได้ด้วยตัวเอง ซึ่งหากมีวิดีโอเป็นจำนวนมาก การใช้มนุษย์ในการสร้างชุดข้อมูลนั้นอาจจะต้องใช้มนุษย์เป็นจำนวนมาก 
และก่อให้เกิดภาระแก่มนุษย์ อีกทั้งยังเป็นงานที่ลำบาก และน่าเบื่อ

ทางคณะผู้วิจัยจึงมีความต้องการที่จะออกแบบและสร้างแอพพลิเคชั่น labeling tool สำหรับสร้างชุดข้อมูลจากวิดีโอ เพื่อช่วยแบ่งเบาภาระของผู้พัฒนาในการสร้างชุดข้อมูล 
เพื่อนำชุดข้อมูลเหล่านี้ไปสร้างโมเดลปัญญาประดิษฐ์สำหรับใช้แก้ปัญหาที่ลูกค้าต้องการ โดยโครงการสหกิจนี้เน้นศึกษาเกี่ยวกับการวิเคราะห์และจดจำการกระทำของมนุษย์ภายในสำนักงานจากภาพเคลื่อนไหวเป็นหลัก

%%%%%%%%%%%%%%%%%%%%%%%%%%%%%%%%%%%%%%%%%%%%%%%%%%%%%%%%%%%%%%%%%%%%%%%%%%%%%%%
\section{วัตถุประสงค์}
\begin{enumerate}
	\setlength\itemsep{-0.25em}
	\item เพื่อออกแบบ และสร้างระบบที่สามารถตรวจจับมนุษย์ และจดจำการกระทำพื้นฐานของมนุษย์ภายในสำนักงาน 
	ประกอบด้วย ยืน นั่ง เดิน เล่นโทรศัพท์ กินข้าว พูดคุย นอน โดยใช้ปัญญาประดิษฐ์ประมวลผล
	\item เพื่อสร้างเครื่องมือที่สามารถการสร้างชุดข้อมูลสำหรับการสร้างโมเดลปัญญาประดิษฐ์ในการจดจำการกระทำของมนุษย์ ให้สามารถใช้งานได้ง่าย 
	สะดวกสบายมากขึ้น และมีประสิทธิภาพที่สูงกว่าเครื่องมือตัวอื่นในปัจจุบัน
\end{enumerate}

%%%%%%%%%%%%%%%%%%%%%%%%%%%%%%%%%%%%%%%%%%%%%%%%%%%%%%%%%%%%%%%%%%%%%%%%%%%%%%%

\section{ประโยชน์ที่คาดว่าจะได้รับ}
\begin{enumerate}
	\setlength\itemsep{-0.25em}
	\item ออกแบบและสร้างแอพพลิเคชั่น labeling tool โดยมีปัญญาประดิษฐ์เข้ามาช่วย ที่สามารถสร้าง label สำหรับนำไปใช้สร้างโมเดลปัญญาประดิษฐ์ได้
	\item ออกแบบและสร้างต้นแบบของแพลตฟอร์มวิเคราะห์การกระทำมนุษย์ โดยงานวิจัยนี้จะมุ่งเน้นการประมวลผลวิดีโอแล้วสร้างรายงานเกี่ยวกับกิจกรรมของมนุษย์ภายในสำนักในวิดีโอได้
	\item สร้างและทดสอบโมเดลปัญญาประดิษฐ์สำหรับการจดจำการกระทำของมนุษย์อย่างน้อย 2 โมเดล
\end{enumerate}
\clearpage

%%%%%%%%%%%%%%%%%%%%%%%%%%%%%%%%%%%%%%%%%%%%%%%%%%%%%%%%%%%%%%%%%%%%%%%%%%%%%%%
\section{ขอบเขตการดำเนินงาน}
\begin{enumerate}
	\setlength\itemsep{-0.25em}
	\item Labeling tool ต้องสามารถตัดวิดิโอช่วงเวลาที่ไม่มีมนุษย์อยู่ออกได้อัตโนมัติโดยใช้ปัญญาประดิษฐ์
	\item Labeling tool สามารถระบุตำแหน่งมนุษย์แต่ละคนในวิดีโอได้และสามารถระบุการกระทำของมนุษย์ในวิดีโอได้ 
	โดยการกระทำที่กำหนดจะประกอบไปด้วยกระทำได้แก่ ยืน นั่ง ใช้คอมพิวเตอร์ เล่นโทรศัพท์ เดิน กินข้าว 
	\item ชุดข้อมูลผลลัพธ์ที่ได้จาก labeling tool ต้องสามารถนำไปใช้ในการสร้างโมเดลปัญญาประดิษฐ์ต่อได้
	\item สร้างแอพพลิเคชั่น labeling tool ด้วยภาษาไพธอน(Python)
	\item ออกแบบและสร้างแอพลิเคชั่น labeling tool ที่มนุษย์สามารถทำงานร่วมกับปัญญาประดิษฐ์ได้
	\item สร้างโมเดลปัญญาประดิษฐ์สำหรับการจดจำการกระทำของมนุษย์อย่างน้อย 2 โมเดล ที่สามารถระบุการกระทำของมนุษย์ตามที่กำหนดไว้ได้ในข้อ 2. 
	เพื่อนำไปใช้ในการสร้างระบบวิเคราะห์การกระทำมนุษย์ภายในสำนักงาน
	\item ระบบวิเคราะห์การกระทำมนุษย์ต้องสามารถนำวิดีโอมาวิเคราะห์ข้อมูลการกระทำและตำแหน่งของมนุษย์แต่ละคน แล้วนำข้อมูลเหล่านั้นไปสร้างรายงานออกมาได้
	\item ความละเอียดอย่างต่ำของวิดีโอต้องมากกว่า 640 x 480 (กว้าง x สูง)
	\item วิดีโอจะต้องมีเฟรมเรทต่อวินาที(fps) อย่างต่ำ 10 fps
\end{enumerate}

%%%%%%%%%%%%%%%%%%%%%%%%%%%%%%%%%%%%%%%%%%%%%%%%%%%%%%%%%%%%%%%%%%%%%%%%%%%%%%%
\section{ภาพรวมของระบบและขั้นตอนการดำเนินงาน}
งานวิจัยนี้การดำเนินงานวิจัยถูกแบ่งออกเป็นสองส่วน คือ ส่วนที่หนึ่งส่วนเครื่องมือสำหรับการเตรียมชุดข้อมูล (dataset) เป็นส่วนที่ทำเครื่องมือสำหรับช่วยผู้พัฒนาในการสร้างชุดข้อมูล และส่วนที่สองนำชุดข้อมูลไปสร้างโมเดล
\subsection*{ศึกษาค้นคว้าเอกสารและงานวิจัยที่เกี่ยวข้อง}
\begin{enumerate}\setlength\itemsep{-0.25em}
	\item ศึกษาเกี่ยวกับการวิเคราะห์ผลวิดิโอ (video analytics)
	\item ศึกษาเกี่ยวกับชุดข้อมูลสำหรับการวิเคราะห์ผลวิดิโอ
	\item ศึกษาเกี่ยวกับโมเดลใช้ในการวิเคราะห์ผลวิดิโอ
	\item ศึกษาเครื่องมือที่ใช้สำหรับช่วยสร้างชุดข้อมูล
\end{enumerate}
\subsection*{ส่วนเครื่องมือสำหรับการเตรียมชุดข้อมูล (dataset)}
\begin{enumerate}\setlength\itemsep{-0.25em}
	\item ออกแบบหน้าต่างของแอพพลิเคชั่น
	\item สร้างระบบของแอพพลิเคชั่น
	\item ทดสอบการทำงานของแอพพลิเคชั่น
\end{enumerate}
\subsection*{ส่วนนำชุดข้อมูลไปสร้างโมเดล}
\begin{enumerate}\setlength\itemsep{-0.25em}
	\item สร้างชุดข้อมูลสำหรับสร้างโมเดล
	\item สร้างโมเดลสำหรับการทำนายการกระทำของมนุษย์
	\item ทดสอบการทำงานของโมเดล
\end{enumerate}
\clearpage
%%%%%%%%%%%%%%%%%%%%%%%%%%%%%%%%%%%%%%%%%%%%%%%%%%%%%%%%%%%%%%%%%%%%%%%%%%%%%%%
\section{Gantt chart}แผนการทำงานของคณะวิจัยจะแบ่งออกเป็น 3 ช่วง ซึ่งในแต่ละช่วงจะมี task ดังนี้
\begin{enumerate}
	\item Phase 1 :  ศึกษาข้อมูลเกี่ยวกับการทำวิเคราะห์วิดีโอ(video analytics) และเครื่องมือที่ใช้ในการทำโครงการวิจัย
		\begin{enumerate}\setlength\itemsep{-0.25em}
			\item task 1 	ศึกษา และ ทดลองใช้เทคนิคที่เกี่ยวข้องกับการวิเคราะห์วิดีโอ(video analytics)
			\item task2	รวบรวมชุดข้อมูลสำหรับการทำ human action recognition
		\end{enumerate}
	\item Phase 2 :  สร้างและออกแบบแอพพลิเคชั่นโดยจะประกอบด้วยงานในส่วนหลักๆ คือ
		\begin{enumerate}\setlength\itemsep{-0.25em}
			\item task3	โมเดลสำหรับวิเคราะห์ผลการกระทำของมนุษย์
			\item task4	หน้าต่าง UI
		\end{enumerate}
	\item Phase 3 : ปรับปรุงและแก้ไขปัญหา ทำให้แอพพลเคชั่นสามารถนำไปใช้งานได้จริง
\end{enumerate}

%%%%%%%%%%%%%%%%%%%%%%%%%%%%%%%%%%%%%%%%%%%%%%%%%%%%%%%%%%%%%%%%%%%%%%%%%%%%%%%
\section{Milestones}จะมีการกำหนด milestones ทั้งหมด 3 ครั้ง คือ
\begin{enumerate}
	\item Milestone  1  วันที่ 12  มิถุนายน  2562
		\begin{enumerate}\setlength\itemsep{-0.25em}
			\item ศึกษางานวิจัยที่เกี่ยวข้องกับการวิเคราะห์วิดีโอ(video analytics) ต่าง ๆ  ได้แก่ Youtube-8M , Moment in time , AVA
		\end{enumerate}
	\item Milestone  2  วันที่ 14  สิงหาคม 2562
		\begin{enumerate}\setlength\itemsep{-0.25em}
			\item สร้างหน้าต่าง UI ครบสมบูรณ์
			\item ความคืบหน้าของโมเดลมากกว่า 50 \%-symbol
			\item ฟังก์ชั่นการทำงานของแอพพลิเคชั่นสามารถทำงานได้มากกว่า 50 \%-symbol
		\end{enumerate}
	\item Milestone  3  วันที่ 15  ตุลาคม 2562
		\begin{enumerate}\setlength\itemsep{-0.25em}
			\item แอพพลิเคชั่นสามารถทำงานได้ครบกระบวนการ
			\item ได้โมเดลสามารถทำนายการกระทำของมนุษย์ได้
		\end{enumerate}
\end{enumerate}



