%%%%%%%%%%%%%%%%%%%%%%%%%%%%%%%%%%%%%%%%%%%%%%%%%%%%%%%%%%%%%%%%%%%%%%%%%%%%%%%
\section{Gantt chart}แผนการทำงานของคณะวิจัยจะแบ่งออกเป็น 3 ช่วง ซึ่งในแต่ละช่วงจะมี task ดังนี้
\begin{enumerate}
	\item Phase 1 :  ศึกษาข้อมูลเกี่ยวกับการทำวิเคราะห์วิดีโอ(video analytics) และเครื่องมือที่ใช้ในการทำโครงการวิจัย
		\begin{enumerate}\setlength\itemsep{-0.25em}
			\item task 1 	ศึกษา และ ทดลองใช้เทคนิคที่เกี่ยวข้องกับการวิเคราะห์วิดีโอ(video analytics)
			\item task2	รวบรวมชุดข้อมูลสำหรับการทำ human action recognition
		\end{enumerate}
	\item Phase 2 :  สร้างและออกแบบแอพพลิเคชั่นโดยจะประกอบด้วยงานในส่วนหลักๆ คือ
		\begin{enumerate}\setlength\itemsep{-0.25em}
			\item task3	โมเดลสำหรับวิเคราะห์ผลการกระทำของมนุษย์
			\item task4	หน้าต่าง UI
		\end{enumerate}
	\item Phase 3 : ปรับปรุงและแก้ไขปัญหา ทำให้แอพพลเคชั่นสามารถนำไปใช้งานได้จริง
\end{enumerate}

%%%%%%%%%%%%%%%%%%%%%%%%%%%%%%%%%%%%%%%%%%%%%%%%%%%%%%%%%%%%%%%%%%%%%%%%%%%%%%%
\section{Milestones}จะมีการกำหนด milestones ทั้งหมด 3 ครั้ง คือ
\begin{enumerate}
	\item Milestone  1  วันที่ 12  มิถุนายน  2562
		\begin{enumerate}\setlength\itemsep{-0.25em}
			\item ศึกษางานวิจัยที่เกี่ยวข้องกับการวิเคราะห์วิดีโอ(video analytics) ต่าง ๆ  ได้แก่ Youtube-8M , Moment in time , AVA
		\end{enumerate}
	\item Milestone  2  วันที่ 14  สิงหาคม 2562
		\begin{enumerate}\setlength\itemsep{-0.25em}
			\item สร้างหน้าต่าง UI ครบสมบูรณ์
			\item ความคืบหน้าของโมเดลมากกว่า 50 \%-symbol
			\item ฟังก์ชั่นการทำงานของแอพพลิเคชั่นสามารถทำงานได้มากกว่า 50 \%-symbol
		\end{enumerate}
	\item Milestone  3  วันที่ 15  ตุลาคม 2562
		\begin{enumerate}\setlength\itemsep{-0.25em}
			\item แอพพลิเคชั่นสามารถทำงานได้ครบกระบวนการ
			\item ได้โมเดลสามารถทำนายการกระทำของมนุษย์ได้
		\end{enumerate}
\end{enumerate}