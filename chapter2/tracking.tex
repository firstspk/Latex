การติดตามการเคลื่อนไหวของวัตถุ\textsuperscript{\cite{danelljan2014accurate}} คือระบบที่ใช้สำหรับการติดตามการเคลื่อนไหวของวัตถุที่สนใจที่อยู่ในรูปภาพ 
โดยใช้การคำนวณทางคณิตศาสตร์ และการประมวลผลภาพ (image processing) ทำให้การประมวลผลนั้นเร็วกว่าการใช้โมเดลปัญญาประดิษฐ์ ซึ่งอัลกอริทึมติดตามการเคลื่อนไหวที่นิยมใช้มีสองอัลกอริทึม
คือ correlation filter และ kalman filter ซึ่งหลักการของทั้งสองอัลกอริทึมนั้นจะแตกต่างกันโดยที่ correlation filter นั้นจะใช้พิกเซลของวัตถุในการคำนวณตำแหน่งถัดไปของวัตถุ 
และ kalman filter จะใช้ข้อมูลการเคลื่อนไหวในการคำนวณตำแหน่งถัดไปของวัตถุ ซึ่งจากการศึกษาในบทความ "Object Tracking using Correlation,
Kalman Filterand Fast Means Shift Algorithms"\textsuperscript{\cite{ali2006object}} kalman filter มีประสิทธิภาพที่สูงนั้นจะขึ้นอยู่กับข้อมูลที่ได้จากการวัด (measurement)
และความซับซ้อนในการเคลื่อนไหวของวัตถุ ในขณะที่ correlation นั้นมีประสิทธิภาพที่ด้อยกว่า kalman filter เพียงเล็กน้อยและสามารถติดตามการเคลื่อนไหวที่ซับซ้อนของวัตถุได้ดีกว่า 
(การเคลื่อนไหวที่ซับซ้อนหมายถึง การเคลื่อนไหวที่เกิดการเปลี่ยนทิศทางฉับพลันบ่อย) ผู้วิจัยจึงตัดสินใจเลือกใช้ correlation filter ในงานครั้งนี้
\begin{figure}[!ht]
	\centering
	\includegraphics[width=1\textwidth]{chapter2/images/track-concept.png}
		\caption[แนวคิดของระบบติดตามการเคลื่อนไหวของวัตถุ]{แนวคิดของระบบติดตามการเคลื่อนไหวของวัตถุ\textsuperscript{\cite{correlation_filter}}}
    	\label{fig:Track_concept}
\end{figure}

จากรูปที่ \ref{fig:Track_concept} เป็นหลักการในการติดตามการเคลื่อนไหวของวัตถุแบบ correlation filter โดยการนำรูปมาผ่านกระบวนการแปลงฟูรีเยร์ (fourier transform)
และนำมาคูณกับ correlation filter ซึ่งเป็นตัวกรองที่ใช้สำหรับการหาความสัมพันธ์กับวัตถุในภาพ จากนั้นทำการแปลงฟูรีเยร์ผกผัน (inverse fourier transform) 
เพื่อตรวจสอบว่าวัตถุในภาพนั้นอยู่ที่ตำแหน่งใด โดยมีการคำนวณเริ่มจากการหา correlation filter ที่ดีที่สุดโดยใช้วิธีลดผลรวมของข้อผิดพลาดกำลังสองให้น้อยที่สุดดังนี้

\begin{equation}
\epsilon = \left \| \sum_{l = 1}^{d} h^{l} \star f^{l} - g \right \|^2 + \lambda \sum_{l = 1}^{d}\left \| h^{l} \right \|^2
\end{equation}
โดยที่
\begin{conditions}
 \epsilon     	&   ค่าความคลาดเคลื่อน 							\\
 d      		&  จำนวนมิติของผังคุณลักษณะของภาพ  \\   
 h 			&  correlation filter								\\
\star 			&  circular correlation							\\
 f			&  พื้นที่สี่เหลี่ยมของวัตถุที่สนใจที่ได้จากการทำผังคุณลักษณะ	\\
 g			&  ผลลัพธ์ correlation ที่ต้องการของ f					\\
 \lambda   		&  regularization term
\end{conditions}

เมื่อพิจารณาจากรูปภาพเดียวในกรณีที่เวลา ($t$) เท่ากับ 1 จะสามารถจัดรูปสมการด้านบนได้ดังนี้ 

\begin{equation}
H^{l} = \frac{\bar{G}F^{l}}{\sum_{k=1}^{d}\bar{F^{k}}F^{k} + \lambda}
\end{equation}
\begin{equation}
H_{t}^{l} = \frac{A_{t}^{l}}{B_{t}}					
\end{equation}					
\begin{equation}
A_{t}^{l} = (1-\eta )A_{t-1}^{l} + \eta \bar{G_{t}}F_{t}^{l}
\end{equation}
\begin{equation}
B_{t} = (1-\eta )B_{t-1} + \eta \sum_{k=1}^{d}\bar{F_{t}^{k}}F_{t}^{k}
\end{equation}
\clearpage
โดยที่
\begin{conditions}
 H 		     	&   correlation filter								\\
 \eta      		&  อัตราการเรียนรู้						 		\\   
 \bar{G} 		&  g ที่ผ่านการทำ complex conjugation					\\
 F			&  พื้นที่สี่เหลี่ยมของวัตถุที่สนใจที่ได้จากการทำผังคุณลักษณะ	\\
 \bar{F}		&   f ที่ผ่านการทำ complex conjugation					\\
 t 	  		&  เวลา
\end{conditions}
จากสมการที่ได้มาจะสามารถทำให้หาตำแหน่งต่อไปของวัตถุที่สนใจได้ด้วยสมการต่อไปนี้
\begin{equation}
y = F^{-1}\left \{ \frac{\sum_{l = 1}^{d} \bar{A^{l}}Z^{l}}{B + \lambda} \right \}
\end{equation}
โดยที่
\begin{conditions}
 y 		     	&   correlation score										\\
 F^{-1}    		&  การแปลงฟูรีเยร์ผกผันแบบไม่ต่อเนื่อง (inverse discrete fourier transform)						\\   	
 \bar{A}^{l} 	&  $A^{l}$ ที่ผ่านการทำ complex conjugation				\\
 Z	 		&  พื้นที่สี่เหลี่ยมของวัตถุที่สนใจที่ได้จากการหาผังคุณลักษณะของภาพใหม่	
\end{conditions}
โดยค่าของ $y$ ที่ได้ออกมาจะทำให้รู้ถึงตำแหน่งของวัตถุที่สนใจได้ ณ ตำแหน่งที่ $y$ มีค่าสูงสุด