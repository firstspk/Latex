% ************************** Thesis Chapter2 **********************************
\clearpage
\chapter{ทฤษฎี/การวิจัยที่เกี่ยวข้อง}
การวิเคราะห์วิดีโอในปัจจุบันนั้นมีวิธีและเทคนิคมากมาย ผู้วิจัยจึงต้องศึกษาองค์ความรู้และงานวิจัยที่เกี่ยวข้องกับวัตถุประสงค์ของงาน 
เพื่อศึกษาและใช้เป็นแนวทางในการประยุกต์สำหรับสร้างเครื่องมือสำหรับกำกับข้อมูลด้วยปัญญาประดิษฐ์ และโมเดลปัญญาประดิษฐ์สำหรับการจำแนกการกระทำของมนุษย์ 
ซึ่งหัวข้อที่ผู้วิจัยได้ไปศึกษามา มีดังต่อไปนี้
\begin{enumerate}
	\setlength\itemsep{-0.25em}
	\item การวิเคราะห์ผลวิดีโอ
	\begin{enumerate}	
		\item การตรวจจับวัตถุ (object detection)
		\item การทำนายตำแหน่งถัดไปของวัตถุ (object tracker)
		\item การระบุตัวตนของบุคคล (person re-identification)
		\item การจำแนกการกระทำ
	\end{enumerate}
	\setlength\itemsep{-0.25em}
	\item เครื่องมือสำหรับการวิเคราะห์ผลวิดีโอ
	\begin{enumerate}	
		\item โมเดลปัญญาประดิษฐ์สำหรับจำแนกการกระทำมนุษย์
		\item เครื่องมือกำกับคุณลักษณะ (labeling tool)
	\end{enumerate}
	\item ทฤษฎีที่เกี่ยวข้อง
	\begin{enumerate}	
		\item Optical flow
		\item IoU
	\end{enumerate}
\end{enumerate}

\section{การวิเคราะห์ผลวิดีโอ}
ในส่วนของงานวิจัยนี้สิ่งที่สนใจ คือ ข้อมูลการกระทำของมนุษย์แต่ละคนภายในวิดีโอ เพื่อที่จะได้ผลลัพธ์ที่มีประสิทธิภาพออกมาเป็นข้อมูลของสิ่งที่สนใจ เช่น จำนวนคนที่เดินผ่านกล้อง 
หรือทิศทางการเดินของคนในวิดีโอ จึงจำเป็นต้องใช้การประมวลผลวิดีโอเพื่อที่จะสกัดสิ่งที่สนใจออกมาจากวิดีโอ ซึ่งการประมวลผลวิดีโอมีหลากหลายกระบวนการ 
โดยในแต่ละกระบวนการจะมีจุดประสงค์ของการทำและผลลัพธ์หลังการประมวลผลที่แตกต่างกัน ในหัวข้อนี้จะมาอธิบายถึงกระบวนการในการประมวลผลของวิดีโอและผลลัพธ์ของกระบวนการนั้น
\subsection{การตรวจจับวัตถุ}
การตรวจจับวัตถุนั้นเป็นหนึ่งในกระบวนการวิเคราะห์ผลของวิดีโอ กล่าวคือกระบวนการที่ผู้วิจัยจะต้องทำการระบุสิ่งที่สนใจว่า คืออะไร อยู่ที่ตำแหน่งใด 	การตรวจจับวัตถุถูกค้นพบเมื่อนานมาแล้ว และในปัจจุบันนั้นสามารถทำได้หลากหลายวิธี โดยภายในบทความนี้จะสรุปใจความสำคัญของวิธีการต่างในการตรวจจับวัตถุ เช่น Sliding Window , Brute Force Search , R-CNN , Fast-RCNN , Faster-RCNN , YOLO , SSD 
\begin{enumerate}
		\item Sliding Window วิธีการที่เปรียบเสมือนมีหน้าต่าง (kernel) ค่อยๆเลื่อนไปยังแต่ละพิกเซลบนรูป ซึ่งก่อนการเลื่อนของหน้าต่างแต่ละครั้ง จะนำส่วนของรูปภาพที่ถูกหน้าต่างทับอยู่ไปทำนายว่าใช่วัตถุที่เราต้องการหรือไม่ จากนั้นจึงค่อยเลื่อนถัดไป โดยจะทำกระบวนการแบบนี้จนครบทั้งรูปภาพ
	\item Brute Force Search ถูกสร้างขึ้นมาเพื่อแก้ปัญหาขนาดของหน้าต่างไม่ตรงกับขนาดของวัตถุที่อยู่ในภาพทำให้มีโอกาสที่จะไม่พบวัตถุ โดยหลักการของวิธีการนี้ คือ การย่อ-ขยาย รูปภาพและนำเข้าในหลายๆอัตราส่วน ตั้งแต่ 0.1 เท่า จนถึง 2 เท่า แต่ข้อเสียของวิธีการนี้คือ มีการคำนวณพื้นที่ซ้ำๆ และ ใช้เวลานาน
	\item R-CNN ใช้อัลกอริทึ่ม Selective search เข้ามาช่วยในการเสนอพื้นที่ที่น่าจะมีวัตถุอยู่ทดแทนการค้นหาทุกๆตำแหน่ง จากนั้นก็นำรูปภาพในส่วนพื้นที่นั้นไปทำนายว่าวัตถุนั้นคืออะไร กรณีที่มีพื้นที่ที่อยู่ใกล้ๆวัตถุถูกเสนอเข้ามาเป็นจำนวนมากด้วย เราจะใช้ Non-Maximum Suppression (NMS) หรือการเลือกพื้นที่ที่ถูกทับซ้อนมากที่สุดในบริเวณนั้น
	\item Fast-RCNN จากวิธีการ R-CNN แต่ละพื้นที่จะถูกนำไปสกัดคุณลักษณะ และ ทำนายผลทีละพื้นที่่ ทำให้เสียเวลา โดย Faster-RCNN จะมีส่วนที่คล้ายกับ R-CNN ในส่วนการทำ Selective search หาพื้นที่ที่น่าจะมีวัตถุเหมือนเดิม แต่ Faster-RCNN จะนำรูปภาพทั้งรูปภาพไปสกัดคุณลักษณะ หลังจากที่ได้คุณลักษณะแล้ว นำพิกัดของพื้นที่ที่น่าจะมีวัตถุ บนรูปภาพที่ถูกสกัดคุณลักษณะแล้วของ ไปผ่าน ROI Pooling (การลดขนาดข้อมูลให้มีขนาดคงที่เพื่อเป็นอินพุทให้กับโมเดลในการทำนายผล)
	\item Faster-RCNN พัฒนาจาก Fast-RCNN โดยวิธีของ Faster-RCNN จะรวมในส่วนของ Selective search และ การทำงานอื่นๆให้อยู่ในโครงข่ายเดียวกัน สรุปคือการทำงานของโครงข่ายของ Faster-RCNN จะมีการทำงาน 3 อย่างหลักคือ 1) สกัดคุณลักษณะ 2) การเสนอส่วนที่น่าจะมีวัตถุอยู่ในรูปภาพ 3) หลังจากได้รูปภาพจากการสกัดคุณลักษณะ นำพิกัดของพื้นที่ที่น่าจะมีวัตถุ บนรูปภาพที่ถูกสกัดคุณลักษณะแล้วของ ไปผ่าน ROI Pooling
	\item YOLO เป็นวิธีการที่ใช้โครงข่ายประสาทแบบคอนโวลูชั่นเพียงตัวเดียวทำนายรูปภาพทั้งรูป โดยโครงข่ายจะแบ่งรูปภาพออกเป็นพื้นที่ และ ใช้ Fully-connected (เป็นโครงข่ายประสาทเทียมที่นำเอาคุณลักษณะมาทำนายผล) ทำนายตำแหน่งของกรอบสี่เหลี่ยมและหมวดหมู่ของกรอบสี่เหลี่ยมในแต่ละพื้นที่ไปพร้อมกัน 
	\item SSD ใช้โครงข่ายประสาทเทียมตัวเดียวเหมือนกับ YOLO แต่การออกแบบโครงสร้างแตกต่างกัน SSD จะใช้ VGG-16(เป็นโมเดล CNN ชนิดหนึ่ง) ในการสกัดคุณลักษณะ และ ใช้ Convolution layer ต่อกันหลายๆชั้นเพื่อลดมิติและความละเอียดทำให้ตรวจจับวัตถุในหลายๆขนาด ซึ่งในแต่ละชั้นจะได้ผลออกมาเป็น Convolution filter จากนั้นจะนำ Convolution filter ไปทำนายผลต่อ
\end{enumerate}

 



%ซึ่งในปัจจุบันการทำการตรวจจับวัตถุมักนำปัญญาประดิษฐ์มาใช้เนื่องจากมีความแม่นยำ และ ช่วยแบ่งเบาภาระของผู้วิจัยในการระบุสิ่งที่สุดใจภายในเฟรม ซึ่งโครงสร้างโมเดลปัญญาประดิษฐ์ของการตรวจจับวัตถุที่ผู้วิจัยสนใจมีดังนี้

\subsubsection*{YOLO}
\begin{figure}[!ht]
    \centering
    \includegraphics[width=0.7\textwidth]{chapter2/images/yolo.jpg}
    \caption{กระบวนการทำงานของโครงสร้างโมเดลปัญญาประดิษฐ์ของ YOLO}
    \label{fig:yolo}
\end{figure}

โครงสร้างโมเดลปัญญาประดิษฐ์ของ YOLO เป็นโครงสร้างที่มีความเร็วมาก มีความเร็วในการประมวลผลถึง 45 เฟรมต่อวิ ทำให้สามารถประมวลผลแบบเรียลไทม์ได้ นอกจากนั้นยังมีความแม่นยำ mAP มากกว่าโมเดลสำหรับตรวจจับวัตถุอื่นๆถึง 2 เท่า ซึ่งเหตุผลที่โครงสร้างโมเดลปัญญาประดิษฐ์ของ YOLO เร็วกว่าโมเดลปัญญาประดิษฐ์ตัวอื่นๆ เนื่องจาก มีแนวคิดที่ต่างออกไป คือ สำหรับการตรวจจับวัตถุในวิธีการก่อนหน้าจะใช้วิธีทำนายกรอบสี่เหลี่ยมก่อน แล้วจึงค่อยนำกรอบสี่เหลี่ยมไปทำนายว่าเป็นหมวดหมู่อะไร ซึ่ง YOLO มีวิธีการที่ต่างออกไป คือ ทำนายตำแหน่งของกรอบสี่เหลี่ยมและทำนายว่ากรอบสี่เหลี่ยมนั้นเป็นหมวดหมู่อะไรพร้อมกัน โดยใช้โครงข่ายประสาทแบบคอนโวลูชั่น ด้วยแนวคิดนี้จึงเป็นที่มาของชื่อ YOLO หรือ you only look once การมองแค่เพียงครั้งเดียว ซึ่งโครงสร้างโมเดลปัญญาประดิษฐ์ของ YOLO ที่ถูกใช้ในงานวิจัยนี้ประกอบไปด้วย 1) YOLOv3-tiny 2)YOLOv3 3) YOLOv3-spp	ซึ่งทั้ง 3 โครงสร้างจะมีความแตกต่างของโครงสร้างดังนี้
\begin{enumerate}
	\setlength\itemsep{-0.25em}
	\item YOLOv3-tiny ใช้ Max-Pooling layers ในขั้นตอนของการลดจำนวนข้อมูลตัวอย่าง
	\item YOLOv3 ใช้ Convolutional layers ในขั้นตอนของการลดจำนวนข้อมูลตัวอย่าง
	\item YOLOv3-spp ใช้ Convolutional layers+ฟีเจอร์ที่ดีที่สุดของ Max-Pooling layers ในขั้นตอนของการลดจำนวนข้อมูลตัวอย่าง
\end{enumerate}
%ข้อมูลผลการทำงานของโมเดลปัญญาประดิษฐ์สำหรับการทำการตรวจจับภาพบุคคล อ้างอิงข้อมูลจากเว็บไซต์ของ YOLO
%\begin{table}[!ht]
%	\begin{tabular}{|c|c|c|}
%		\hline
%		{}&{ความเร็วต่อรูปภาพ (มิลลิวินาที)}&{ความแม่นยำ (0.5 IoU mAP)}			\\
%		\hline
%		SSD Mobilenet v1 ppn	 		& 26				& 20														\\
%		YOLO-v3 320				& 22				& 51.5				\\	
%		YOLO-v3 tiny				& 4.5				& 33.1				\\
%		YOLO-v3 spp				& 50				& 60.6				\\	
%		Faster RCNN inceptrion v2		& 58				& 28		\\
%	\hline
%	\end{tabular}
%	\caption{ข้อมูลผลการทำงานของโมเดลปัญญาประดิษฐ์สำหรับการทำการตรวจจับภาพบุคคล อ้างอิงข้อมูลจากเว็บไซต์ของ YOLO}
%\end{table}



\subsection{ระบบติดตามการเคลื่อนไหวของวัตถุ}
การติดตามการเคลื่อนไหวของวัตถุ\textsuperscript{\cite{danelljan2014accurate}} คือระบบที่ใช้สำหรับการติดตามการเคลื่อนไหวของวัตถุที่สนใจที่อยู่ในรูปภาพ 
โดยใช้การคำนวณทางคณิตศาสตร์ และการประมวลผลภาพ (image processing) ทำให้การประมวลผลนั้นเร็วกว่าการใช้โมเดลปัญญาประดิษฐ์ ซึ่งอัลกอริทึมติดตามการเคลื่อนไหวที่นิยมใช้มีสองอัลกอริทึม
คือ correlation filter และ kalman filter ซึ่งหลักการของทั้งสองอัลกอริทึมนั้นจะแตกต่างกันโดยที่ correlation filter นั้นจะใช้พิกเซลของวัตถุในการคำนวณตำแหน่งถัดไปของวัตถุ 
และ kalman filter จะใช้ข้อมูลการเคลื่อนไหวในการคำนวณตำแหน่งถัดไปของวัตถุ ซึ่งจากการศึกษาในบทความ "Object Tracking using Correlation,
Kalman Filterand Fast Means Shift Algorithms"\textsuperscript{\cite{ali2006object}} kalman filter มีประสิทธิภาพที่สูงนั้นจะขึ้นอยู่กับข้อมูลที่ได้จากการวัด (measurement)
และความซับซ้อนในการเคลื่อนไหวของวัตถุ ในขณะที่ correlation นั้นมีประสิทธิภาพที่ด้อยกว่า kalman filter เพียงเล็กน้อยและสามารถติดตามการเคลื่อนไหวที่ซับซ้อนของวัตถุได้ดีกว่า 
(การเคลื่อนไหวที่ซับซ้อนหมายถึง การเคลื่อนไหวที่เกิดการเปลี่ยนทิศทางฉับพลันบ่อย) ผู้วิจัยจึงตัดสินใจเลือกใช้ correlation filter ในงานครั้งนี้
\begin{figure}[!ht]
	\centering
	\includegraphics[width=1\textwidth]{chapter2/images/track-concept.png}
		\caption[แนวคิดของระบบติดตามการเคลื่อนไหวของวัตถุ]{แนวคิดของระบบติดตามการเคลื่อนไหวของวัตถุ\textsuperscript{\cite{correlation_filter}}}
    	\label{fig:Track_concept}
\end{figure}

จากรูปที่ \ref{fig:Track_concept} เป็นหลักการในการติดตามการเคลื่อนไหวของวัตถุแบบ correlation filter โดยการนำรูปมาผ่านกระบวนการแปลงฟูรีเยร์ (fourier transform)
และนำมาคูณกับ correlation filter ซึ่งเป็นตัวกรองที่ใช้สำหรับการหาความสัมพันธ์กับวัตถุในภาพ จากนั้นทำการแปลงฟูรีเยร์ผกผัน (inverse fourier transform) 
เพื่อตรวจสอบว่าวัตถุในภาพนั้นอยู่ที่ตำแหน่งใด โดยมีการคำนวณเริ่มจากการหา correlation filter ที่ดีที่สุดโดยใช้วิธีลดผลรวมของข้อผิดพลาดกำลังสองให้น้อยที่สุดดังนี้

\begin{equation}
\epsilon = \left \| \sum_{l = 1}^{d} h^{l} \star f^{l} - g \right \|^2 + \lambda \sum_{l = 1}^{d}\left \| h^{l} \right \|^2
\end{equation}
โดยที่
\begin{conditions}
 \epsilon     	&   ค่าความคลาดเคลื่อน 							\\
 d      		&  จำนวนมิติของผังคุณลักษณะของภาพ  \\   
 h 			&  correlation filter								\\
\star 			&  circular correlation							\\
 f			&  พื้นที่สี่เหลี่ยมของวัตถุที่สนใจที่ได้จากการทำผังคุณลักษณะ	\\
 g			&  ผลลัพธ์ correlation ที่ต้องการของ f					\\
 \lambda   		&  regularization term
\end{conditions}

เมื่อพิจารณาจากรูปภาพเดียวในกรณีที่เวลา ($t$) เท่ากับ 1 จะสามารถจัดรูปสมการด้านบนได้ดังนี้ 

\begin{equation}
H^{l} = \frac{\bar{G}F^{l}}{\sum_{k=1}^{d}\bar{F^{k}}F^{k} + \lambda}
\end{equation}
\begin{equation}
H_{t}^{l} = \frac{A_{t}^{l}}{B_{t}}					
\end{equation}					
\begin{equation}
A_{t}^{l} = (1-\eta )A_{t-1}^{l} + \eta \bar{G_{t}}F_{t}^{l}
\end{equation}
\begin{equation}
B_{t} = (1-\eta )B_{t-1} + \eta \sum_{k=1}^{d}\bar{F_{t}^{k}}F_{t}^{k}
\end{equation}
\clearpage
โดยที่
\begin{conditions}
 H 		     	&   correlation filter								\\
 \eta      		&  อัตราการเรียนรู้						 		\\   
 \bar{G} 		&  g ที่ผ่านการทำ complex conjugation					\\
 F			&  พื้นที่สี่เหลี่ยมของวัตถุที่สนใจที่ได้จากการทำผังคุณลักษณะ	\\
 \bar{F}		&   f ที่ผ่านการทำ complex conjugation					\\
 t 	  		&  เวลา
\end{conditions}
จากสมการที่ได้มาจะสามารถทำให้หาตำแหน่งต่อไปของวัตถุที่สนใจได้ด้วยสมการต่อไปนี้
\begin{equation}
y = F^{-1}\left \{ \frac{\sum_{l = 1}^{d} \bar{A^{l}}Z^{l}}{B + \lambda} \right \}
\end{equation}
โดยที่
\begin{conditions}
 y 		     	&   correlation score										\\
 F^{-1}    		&  การแปลงฟูรีเยร์ผกผันแบบไม่ต่อเนื่อง (inverse discrete fourier transform)						\\   	
 \bar{A}^{l} 	&  $A^{l}$ ที่ผ่านการทำ complex conjugation				\\
 Z	 		&  พื้นที่สี่เหลี่ยมของวัตถุที่สนใจที่ได้จากการหาผังคุณลักษณะของภาพใหม่	
\end{conditions}
โดยค่าของ $y$ ที่ได้ออกมาจะทำให้รู้ถึงตำแหน่งของวัตถุที่สนใจได้ ณ ตำแหน่งที่ $y$ มีค่าสูงสุด

\subsection{ระบบระบุตัวตนของบุคคล}
การระบุตัวตนของบุคคล คือการระบุตัวตนของบุคคลภายในวิดีโอหรือระหว่างรูปภาพ สามารถนำมาประยุกต์ใช้ในด้านของการรักษาความปลอดภัย การตามหาบุคคล หรือการตรวจสอบการกระทำของบุคคลนั้นในวิดีโอได้
\par
การทำ การระบุตัวตนของบุคคล นั้นเป็นปัญหาที่ท้าทาย เนื่องจากคุณลักษณะทั่วไปของบุคคลในรูปภาพไม่เพียงพอต่อการระบุบุคคลภายในภาพว่าเป็นบุคคลคนเดียวกันได้ ซึ่งวิธีการที่ใช้สำหรับในการทำ การระบุตัวตนของบุคคล คือวิธีการที่เรียกว่า Dynamically Matching Local Information (DMLI) ที่สามารถจัดลายละเอียดข้อมูลของภาพที่เหมือนกันได้ และได้ประสิทธิภาพที่สูงออกมา
\par
โดยเริ่มต้นของการทำ การระบุตัวตนของบุคคล จะแบ่งภาพออกเป็นทั้งหมด 8 ส่วนและใช่คุณลักษณะของภาพมาทำ normalize ซึ่งจะช่วยในการลดความซ้ำซ้อนของข้อมูล ต่อมาข้อมูลที่ทำการ normalize มาใช้เปรียบเทียบความแตกต่างของคุณลักษณะของรูป หลังจากนั้นหาค่าเฉลี่ยของความแตกต่างออกมา ถ้าค่าที่ออกมาใกล้เคียง 0 จะพูดได้ว่าบุคคลในรูปนั้นเป็นบุคคลเดียวกัน
\clearpage
\subsection{ระบบจำแนกการกระทำของมนุษย์}
การจำแนกการกระทำเป็นกระบวนการในการทำนายการกระทำของมนุษย์หรือสิ่งที่สนใจอื่นๆที่เกิดการกระทำขึ้นภายในวิดีโอ 
โดยในหัวข้อนี้จะกล่าวถึงตั้งแต่ขั้นตอนการได้มาซึ่งชุดข้อมูลมีกระบวนการอย่างไร การนำโมเดลปัญญาประดิษฐ์มาใช้ในการจำแนกการกระทำ และการวัดผลของโมเดลปัญญาประดิษฐ์
โดยชุดข้อมูลที่ผู้วิจัยได้เลือกนำมาศึกษาจากชุดข้อมูลที่ถูกเป็นที่กล่าวถึงในปัจจุบัน และมีขนาดของชุดข้อมูลที่ใหญ่

จากบทความข้างต้นชุดข้อมูลที่เราได้เลือกนำมาใช้ได้แก่ YouTube-8M ,AVA ,Moment in Time โดยแต่ละชุดข้อมูลจะมีความแตกต่างกันในหลายๆด้าน 
แต่จะมีสิ่งที่เหมือนกัน คือ เป็นชุดข้อมูลสำหรับการวิเคราะห์ผลวิดีโอที่มีการสนใจการกระทำของมนุษย์ โดยในบทความนี้จะกล่าวถึงความแตกต่างในด้านต่างๆ 
เช่น เป้าหมายของแต่ละชุดข้อมูล ,วิธีการเก็บข้อมูลสำหรับชุดข้อมูล ,วิธีการสร้างคำกำกับ และรายละเอียดของชุดข้อมูล จากนั้นจะสรุปข้อมูลของแต่ละชุดข้อมูล

%%%%%%%%%%%%%%%%%%%%%%%%%%%%%%%%%%%%%%%%%%%%%%%%%%%%%%%%%%%%%%%%%%%%%%%%%%%%%
\subsubsection*{YouTube-8M} 
\begin{enumerate}
	\item {ชุดข้อมูล}
	\begin{enumerate}
		\setlength\itemsep{-0.25em}
		\item เป้าหมายของชุดข้อมูล : ใช้ทำนายธีมของวิดีโอ
		\item จำนวนของวิดีโอ : 8,264,650 วิดีโอ
		\item ความยาวเฉลี่ยของแต่ละวิดีโอ : 229.6 วินาที
		\item จำนวนของหมวดหมู่ของคำกำกับ : 4800 หมวดหมู่
		\item กฏในการรวบรวมข้อมูลดังนี้
		\begin{enumerate}
			\setlength\itemsep{-0.25em}
			\item ทุกๆหัวข้อต้องเป็นรูปธรรม
			\item ในแต่ละหัวข้อต้องมีจำนวนวิดีโอไม่น้อยกว่า 200 วิดีโอ
			\item ความยาวของวิดีโอต้องอยู่ระหว่าง 120 - 500 วินาที
		\end{enumerate}
		หลังจากได้กฏในการรวบรวมข้อมูลแล้ว ขั้นตอนต่อไปคือการสร้างคำศัพท์ที่ใช้ในการค้นหาข้อมูลวิดีโอจากใน YouTube 
		\item ขั้นตอนในการสร้างคำศัพท์มีดังนี้
		\begin{enumerate}
			\setlength\itemsep{-0.25em}
			\item กำหนดรายการที่อนุญาตหัวข้อที่เป็นรูปธรรมมา 25 ชนิด เช่น เกมส์ เป็นต้น
			\item กำหนดบัญชีดำหัวข้อที่คิดว่าไม่เป็นรูปธรรมไว้ เช่น software เป็นต้น
			\item รวบรวมหัวข้อที่มีอยู่ในรายการที่อนุญาตอย่างน้อย 1 หัวข้อ และต้องไม่มีอยู่ในบัญชีดำซึ่งจะทำให้ได้หัวข้อที่ต้องการมาประมาณ 50,000 หัวข้อ
			\item จากนั้นใช้ผู้ประเมินจำนวน 3 คน ในการคัดหัวข้อที่คิดว่าเป็นรูปธรรม และสามารถจดจำหรือเข้าใจได้ง่ายโดยไม่ต้องเชี่ยวชาญในด้านนั้นๆ 
			ซึ่งผู้ประเมิน ก็จะมีคำถามว่า “มันยากขนาดไหนถึงจะระบุได้ว่ามีหัวข้อดังกล่าวอยู่ในรูปหรือวิดีโอ โดยใช้เพียงแค่การมองเท่านั้น?” โดยแบ่งเป็นระดับดังนี้
			\begin{enumerate}
				\setlength\itemsep{-0.25em}
				\item บุคคลทั่วไปสามารถเข้าใจได้
				\item บุคคลทั่วไปที่ผ่านการอ่านบทความที่เกี่ยวข้องมาแล้วสามารถเข้าใจได้
				\item ต้องเชี่ยญในด้านใดซักด้านจึงจะเข้าใจได้
				\item เป็นไปไม่ได้ ถ้าไม่มีความรู้ที่ไม่ได้เป็นรูปธรรม
				\item ไม่เป็นรูปธรรม
			\end{enumerate}
			\item หลังจากคำถามข้างบนและการให้คะแนน จะทำการเก็บไว้เฉพาะหัวข้อที่มีคะแนนเฉลี่ยมากที่สุดอยู่ที่ประมาณ 2.5 คะแนนหรือต่ำกว่าเท่านั้น
			\item ทำให้สุดท้ายเหลือเพียงประมาณ 10,000 หัวข้อที่สามารถใช้ได้
			\item หลังจากได้หัวข้อที่คิดว่าเป็นรูปธรรมแล้วก็นำไปค้นหาและรวบรวมด้วย YouTube annotation system โดยมีขั้นตอนดังนี้										
			\begin{enumerate}
				\setlength\itemsep{-0.25em}
				\item สุ่มเลือกวิดีโอมา 10 ล้านวิดีโอ พร้อมกับหัวข้อของวิดีโอ โดยใช้กฏที่กำหนดไว้ เอาหัวข้อที่มีจำนวนวิดีโอน้อยกว่า 200 วิดีโอออก
				\item ทำให้เหลือจำนวนวิดีโออยู่ 8,264,650 วิดีโอ
				\item แยกออกเป็น 3 ส่วน Train set, Validate set และ Test set ในอัตราส่วน 70:20:10 ตามลำดับ
			\end{enumerate}
		\end{enumerate}
	\end{enumerate}
	\item {โมเดลปัญญาประดิษฐ์}
	\begin{enumerate}
		\setlength\itemsep{-0.25em}
		\item การเตรียมข้อมูล
			\begin{enumerate}  
				\item คุณลักษณะระดับเฟรม : การลดขนาดของข้อมูล เนื่องจากมีข้อมูลที่มีขนาดใหญ่ทำให้ใช้เวลาในการเปิดนาน ซึ่งกระบวนการนี้จะมีการลดความเร็วเฟรมต่อวินาที 
				เวกเตอร์ของคุณลักษณะ และแปลงข้อมูลจาก 32 บิท ให้เป็น 8 บิท
				\item คุณลักษณะระดับวิดีโอ : การแยกเวกเตอร์คุณลักษณะระดับวิดีโอจากคุณลักษณะระดับเฟรมซึ่งการทำแบบนี้ทำให้ได้ประโยชน์ 3 ข้อ 
				คือโมเดลทั่วไปที่ไม่ใช่โครงข่ายประสาทเทียใสามารถนำไปใช้งานได้ ขนาดข้อมูลเล็กลง และเหมาะกับการนำไปสร้างโมเดล domain adaptive มากขึ้น
			\end{enumerate}	
		\item โมเดลปัญญาประดิษฐ์ % --> เติมตรงนี้ <---
		\item เครื่องมือที่ใช้วัดผลสำหรับงานวิจัยนี้ คือ
		\begin{enumerate}
			\setlength\itemsep{-0.25em}
			\item Mean Average Precision (mAP)
			\item Hit@k
			\item Precision at equal recall rate (PERR)
		\end{enumerate}
		\item ความสามารถของ Machine learning model ในปัจจุบัน
		\begin{enumerate}
			\setlength\itemsep{-0.25em}
			\item 1 % --> เติมตรงนี้ <---
			\item 2 % --> เติมตรงนี้ <---
		\end{enumerate}
		\item ปัญหาที่พบ
		\begin{enumerate}
		\item เนื่องจากว่า YouTube-8M นั้นมีจำนวนข้อมูลที่เยอะมาก ทำให้ไม่สามารถตรวจสอบได้ทั้งหมดว่า ground-truth ของแต่ละวิดีโอนั้นมีความถูกต้องมากน้อยขนาดไหน ทำให้อาจเกิดข้อผิดพลาดได้ (ปัจจุบัน ปี 2019 YouTube-8M ได้มีการตรวจสอบข้อมูลอีกครั้ง เพื่อเพิ่มประสิทธิภาพของชุดข้อมูลซึ่งทำให้ปัจจุบันจำนวนข้อมูล และจำนวน category นั้นจะลดน้อยลงจากข้อมูลที่ใช้อ้างอิงในบทความ \footnote{YouTube-8M,https://arxiv.org/pdf/1609.08675.pdf} ข้างต้นที่ได้กล่าวมา)
	\end{enumerate}	

	\end{enumerate}	
	\end{enumerate}	

%%%%%%%%%%%%%%%%%%%%%%%%%%%%%%%%%%%%%%%%%%%%%%%%%%%%%%%%%%%%%%%%%%%%%%%%%%%%%
\subsubsection*{AVA}	
\begin{enumerate}
	\item {ชุดข้อมูล}
	\begin{enumerate}
		\item เป้าหมายของชุดข้อมูล : สนใจการกระทำของมนุษย์เป็นศูนย์กลาง
		\item จำนวนของวิดีโอ : 640 วิดีโอ
		\item ความยาวเฉลี่ยของแต่ละวิดีโอ : 15 นาที และ ถูกสุ่มตัวอย่างด้วยความถี่ 1 hz 
		\item จำนวนของหมวดหมู่ : 80 หมวดหมู่
		\item ขั้นตอนการเก็บข้อมูลสำหรับการทำชุดข้อมูลมีขั้นตอนการทำ 5 ขั้น คือ
	\begin{enumerate}
%
		\item การสร้างคำศัพท์การกระทำ จะมีหลัก 3 ข้อในการรวบรวมคำศัพท์ คือ
		\begin{enumerate}
			\item เก็บรวบรวมคำศัพท์ทั่วไปที่เกิดขึ้นในชีวิตประจำวัน
			\item จะต้องมีเอกลักษณ์ สามาถเห็นได้ชัดเจน เช่น การถือของ
			\item กำหนดรูปแบบของคำศัพท์ขึ้นมาและใช้ความรู้จากชุดข้อมูลอื่น ในการทำให้ได้หมวดหมู่ของการกระทำของมนุษย์ที่ครอบคลุมของชุดข้อมูล AVA
		\end{enumerate}

%
		\item  หนังและส่วนที่เลือกมาใช้วิดิโอที่ใช้ทำชุดข้อมูล AVA ทั้งหมดจะถูกนำมากจาก YouTube โดยเริ่มจากการรวบรวมเอารายชื่อของนักแสดงที่มีชื่อเสียง ซึ่งจะมีความหลากหลายของเชื้อชาติรวมกันอยู่ ซึ่งวิดิโอที่ถูกคัดเลือกจะมีเกณฑ์ดังนี้ คือ
			\begin{enumerate}
				\item วิดิโอต้องอยู่ในหมวด หนัง และ ละครโทรทัศน์
				\item จะต้องมีความยาวมากกว่า 30 นาที
				\item อัพโหลดเป็นเวลาอย่างน้อย 1 ปี
				\item มียอดวิวคนดูมากกว่า 1000 วิว
				\item ละเว้นวิดิโอบางประเภท เช่น ขาว-ดำ , ความละเอียดต่ำ , การ์ตูน , วิดิโอเกม
			\end{enumerate}
%
		\item  การตีกรอบบุคคลที่อยู่ภายในภาพ ประกอบด้วย 2 ขั้นตอน
			\begin{enumerate}
				\item สร้างกรอบสี่เหลี่ยม โดยใช้โมเดล Faster R-CNN สำหรับการตรวจจับมนุษย์
				\item นำมนุษย์มาใช้ในการตรวจสอบและแก้ไขกรอบสี่เหลี่ยมที่พลาดไป หรือ ตรวจจับผิด
			\end{enumerate}	
		\item  การเชื่อมของบุคคลในช่วงระยะเวลาสั้นๆของเฟรม 
\\
ทำการเชื่อมกรอบสี่เหลี่ยมที่อยู่ในช่วงเวลาเดียวกัน ซึ่งใช้วิธีการ track โดยยึดมนุษย์เป็นศูนย์กลาง ซึ่งจะนำมาคำนวณความใกล้เคียงกันโดยการจับคู่กรอบสี่เหลี่ยม และ ใช้ person embedding จากนั้นจะใช้ Hungarian algorithm ในการหาตัวเลือกที่ดีที่สุด

%
		\item การสร้างคำอธิบาย
\\
		การสร้างคำอธิบายของการกระทำจะถูกสร้างจากเหล่าคนที่เป็นผู้สร้างคำอธิบาย ซึ่งจะใช้หน้าต่างโปรแกรมสำหรับช่วยเหลือในการสร้างซึ่งใน 1 กรอบสี่เหลี่ยม สามารถมีคำอธิบายของการกระทำได้สูงสุดถึง 7 labels นอกจากนั้นสามารถตั้งสถานะบล็อกเนื้อหาที่ไม่เหมาะสม หรือ กรอบสี่เหลี่ยมที่ผิดพลาดได้อีกด้วย ในทางปฎิบัติจะสังเกตได้ว่ามันมีโอกาศผิดอย่างหลีกเลี่ยงไม่ได้ เมื่อต้องได้รับคำสั่งให้หาคำอธิบายของการกระทำที่ถูกต้องจาก 80 หมวดหมู่ จึงแบ่งขั้นตอนออกเป็น 2 ขั้นตอน คือ
		\begin{enumerate}
			\item ข้อเสนอของการกระทำสอบถามเหล่าผู้สร้างคำอธิบาย เพื่อสร้างข้อเสนอสำหรับคำอธิบายของการกระทำจากนั้นจับกลุ่มเข้าด้วยกัน ซึ่งจะทำให้มีโอกาสถูกต้องมากกว่าเป็นข้อเสนอแยกเดี่ยว
			\setlength\itemsep{-0.25em}
			\item ผู้ตรวจสอบข้อเสนอจะตรวจสอบข้อเสนอที่ได้จากขั้นตอนแรก ซึ่งในแต่ละวิดิโอคลิปจะใช้มนุษย์ในการตรวจสอบ 3 คน เมื่อคำอธิบายของการกระทำ ถูกตรวจสอบด้วยผู้ตรวจสอบข้อเสนออย่างน้อย 2 คน คำอธิบายของการกระทำนั้นจะถูกยึดเป็นคำอธิบายหลัก
		\end{enumerate}
	\end{enumerate}
	\end{enumerate}
	\item {Machine learning model}
	\begin{enumerate}
		\item Machine learning model ที่งานวิจัยนี้ใช้ two stream variant ซึ่งจะทำการประมวลผลทั้ง RGB flow และ optical flow และ เป็นโครงสร้างของ Faster RCNN ที่นำ Inception network เข้ามาใช้ 
		\item เครื่องมือที่ใช้วัดผลสำหรับงานวิจัยนี้ คือ ค่า IOU และ 3D IOUs 
		\begin{enumerate}
			\item ค่า IOU คือ ค่าที่ใช้วัดความสอดคล้องระหว่างสองเฟรม ซึ่งใช้สำหรับการวัดผลระดับเฟรม โดยจะเป็นการเทียบกันของกรอบสี่เหลี่ยมที่ตรวจเจอและกรอบสี่เหลี่ยมจริงของวัตถุ
			\item ค่า 3D IOUs คือ ค่าที่ใช้วัดความสอดคล้องระหว่างสองวิดีโอ ซึ่งใช้สำหรับการวัดผลระดับวิดิโอโดยเทียบกันของ ground truth tubes และ linked detection tubes  ซึ่งก็คือ การนำเอากรอบสี่เหลี่ยมจริงของวัตถุในเฟรมที่ติดต่อกันมาเรียงต่อกันเป็น tube และ linkded detection tube คือ การนำเอากรอบสี่เหลี่ยม (bounding box) ที่ตรวจเจอมาเรียงต่อกันเป็น tube
		\end{enumerate}	
		\item ความสามารถของ Machine learning model ในปัจจุบัน
		\begin{enumerate}
			\item จากการทดสอบการเทียบ Machine learning model ของงานวิจัยนี้และวิธีการอื่นๆ โดยนำไปทดสอบกับชุดข้อมูลวิดีโอ JHMDB และ UCF101-24 ได้ผลลัพธ์ออกมาดังนี้
	\begin{table}[!ht]
	\centering
	\begin{tabular}{|c|c|c|c|}
			\hline
			{Frame-mAP}&{JHMDB}&{UCF101-24}\\
			\hline
			Actionness 			& 39.9		& 	-						\\
			Peng w/o MR			& 56.9		& 64.8						\\
			Peng w/  MR 			& 58.5		& 65.7						\\
			ACT					& 65.7		& 69.5						\\
			\hline
			Out approach			& 73.3		& 76.3						\\
			\hline
		\end{tabular}
		\caption{ผลการทดลองของวิธีต่างๆบน Frame Level}
		\label{tab: transfer learning}
		\end{table}
	\end{enumerate}
		\item ปัญหาที่พบ

\end{enumerate}
\end{enumerate}
%%%%%%%%%%%%%%%%%%%%%%%%%%%%%%%%%%%%%%%%%%%%%%%%%%%%%%%%%%%%%%%%%%%%%%%%%%%%
\subsubsection*{Moment in Time}
\begin{enumerate}
	\item {ชุดข้อมูล}
	\begin{enumerate}
		\setlength\itemsep{-0.25em}
		\item เป้าหมายของชุดข้อมูล : สนใจการกระทำทุกการกระทำในวิดิโอ เช่น การกระทำของ คน สัตว์ สิ่งของ และ ปรากฎการณ์ธรรมชาติ 
		\item จำนวนของวิดีโอ : >1,000,000 วิดีโอ
		\item ความยาวเฉลี่ยของแต่ละวิดีโอ : 3 วินาที
		\item จำนวนของหมวดหมู่ : 339 หมวดหมู่
		\item วิธีการเก็บรวบรวมข้อมูล : 
	\begin{enumerate}
		\item เริ่มจากการรวบรวมคำ (verb) ที่มีการใช้อยู่ทั่วไปในชีวิตประจำวันมา 4,500 คำจาก VerbNet จากนั้นนำมาแบ่งกลุ่มคำ(verb) ที่มีความหมายใกล้เคียงกันโดยใช้ features จาก Propbank และ FrameNet โดยเก็บข้อมูลเป็นแบบ binary feature vector ซึ่งถ้าคำ (verb) ไหนมีความเกี่ยวข้องกับ feature ก็จะให้ค่าเป็น 1 ถ้าไม่เกี่ยวข้องกันจะให้ค่าเป็น 0 จากนั้นจึงใช้วิธี k-means clustering ในการแบ่งกลุ่ม เมื่อแบ่งกลุ่มแล้วจากนั้นจะเลือกคำ (verb) จากในแต่ละกลุ่มนั้น โดยคำ (verb) ที่เลือกมานั้นจะเป็นที่ใช้บ่อยที่สุดในกลุ่มนั้น และลบคำ (verb) นั้นออกจากกลุ่มทั้งหมด (คำ ๆ หนึ่งสามารถอยู่ได้หลายกลุ่ม) จากนั้นจะทำกระบวนการนี่ไปเรื่อย ๆ แต่คำ (verb) ที่เลือกมาจะต้องไม่มีความหมายคลุมเครือ ไม่สามารถมองเห็นหรือได้ยินได้ และต้องไม่มีความหมายเหมือนกับคำ (verb) ที่เคยเลือกมาก่อน จนสุดท้ายแล้วได้ออกมาที่ 339 class
		\item ต่อมาทำการหาชุดข้อมูลวิดีโอโดยจะตัดออกมาเพียง 3 วินาทีที่เกี่ยวข้องกับคำ (verb) ใน 339 class ที่เลือกมา จากวิดีโอ แหล่งต่างกัน 10 แหล่ง การตัดวิดีโอนั้นจะไม่ใช้พวก Video2Gif (โมเดลที่ระบุตำแหน่งของสิ่งที่น่าสนใจในวิดีโอ) เพราะจะทำให้เกิด bias ขึ้นจะเกิดขึ้นตอนสร้างโมเดลจากนั้นจะทำการส่งข้อมูลของคำ (verb) และวิดีโอที่ตัดไปยัง Amazon Mechanical Turk (AMT หรือตลาดแรงงาน) เพื่อทำการ label โดยพนักงานแต่ละคนของ AMT จะได้ 64 วิดีโอซึ่งเกี่ยวข้องกับคำ (verb) หนึ่ง และอีก 10 วิดีโอที่มีการทำ label อยู่แล้ว โดยวิดีโอที่มีการทำ label ถ้ามีพนักงานของ AMT ตอบเหมือนกันกับที่ทำ label ไว้เกิน 90\% ถึงจะนำเข้าไปรวมกับชุดข้อมูลส่วนอีก 64 วิดีโอถ้าเป็นของ training set จะต้องผ่านพนักงานของ AMT อย่างน้อย 3 ครั้ง และต้อง label เหมือนกัน 75\% ขึ้นไปถึงจะถือว่าเป็น label ที่ถูกต้อง ถ้าเป็นของ validation และ test set จะต้องผ่านพนักงานของ AMT อย่างน้อย 4 ครั้ง และต้อง label เหมือนกัน 85\% ขึ้นไป ที่ไม่ตั่งเกณฑ์ไว้ที่ 100\% เพราะจะทำให้วิดีโอนั้นยากเกินไปที่จะทำให้สามารถจำการกระทำได้	
	\end{enumerate}
\end{enumerate}
	\item การเตรียมข้อมูล
		\begin{enumerate}
			\item training set จะมี 802,264 วิดีโอ และมีวิดีโอในแต่ละ class อยู่ที่ 500 ถึง 5,000 วิดีโอ
			\item validation set จะมี 33,900 วิดีโอ และมีวิดีโอในแต่ละ class อยู่ที่ 100 วิดีโอ
			\item เริ่มการ preprocess จากแยกภาพRGB ออกมาจากวิดีโอ และทำการเปลี่ยนขนาดของภาพให้เป็น 340x256  pixel
			\item ใช้ TVL1 optical flow algorithm จาก opencv เพื่อลดข้อมูลรบกวนที่จะเกิดขึ้น
			\item ทำการแปลงค่าที่อยู่ใน optical flow ให้เป็นเลขจำนวนเต็ม(integer) เพื่อทำให้การคำนวณนั้นเร็วยิ่งขึ้น
			\item ปรับค่า displacement ใน optical flow ให้ค่าสูงสุดเป็น 15 ต่ำสุดเป็น 0 และทำการปรับขนาดให้เป็นช่วง 0-255
			\item เก็บข้อมูลออกมาในรูปแบบของ grayscale image เพื่อลดพื้นที่ ๆ ใช้เก็บข้อมูล
			\item แก้ปัญหาเรื่องการเคลื่อนไหวของกล้อง(camera motion) โดยการนำค่าเฉลี่ยของ เวกเตอร์(vector) ไปลบกับ displacement
			\item สุดท้ายจะเป็นสุ่มตัดภาพออกมาเพื่อเพิ่มจำนวนข้อมูล
		\end{enumerate}
	\item {Machine learning model}
	\begin{enumerate}
		\item ในงานวิจัยนี้มีการทดสอบ Machine learning model หลายอัน ซึ่ง Machine learning model ที่มีประสิทธิภาพการทำงานที่ดีที่สุดตาม 5 ลำดับแรกดังนี้
			\begin{enumerate}
				\item SVM				มีรูปแบบข้อมูลอินพุท คือ Spatial+Temporal+Auditory 	
				\item I3D 				มีรูปแบบข้อมูลอินพุท คือ Spatial+Temporal
				\item TRN-Multiscale		มีรูปแบบข้อมูลอินพุท คือ Spatial+Temporal
				\item TSN-2stream		มีรูปแบบข้อมูลอินพุท คือ Spatial+Temporal
				\item ResNet50-ImageNet	มีรูปแบบข้อมูลอินพุท คือ Spatial
			\end{enumerate}
		\item เครื่องมือที่ใช้วัดผลงานวิจัยนี้
			\begin{enumerate}
				\item Classification accuracy Top-1 , Top-5
			\end{enumerate}
		\item ความสามารถของ Machine learning model ในปัจจุบัน
			\begin{enumerate}
				\item ทำทดสอบ cross dataset transfer โดยการนำโมเดล ResNet50 I3D pretrained ลงทั้งบน Kinetics และ Moments in time และนำมาเทียบกับชุดข้อมูลอื่น โดยชุดข้อมูลแต่ละชุดจะมีการปรับ frame rate ของวิดีโอให้เป็น 5 fps เหมือนกัน
	\begin{table}[!ht]
		\centering
		\begin{tabular}{|c|c|c|c|}
			\hline
			{Pretrained}&\multicolumn{3}{c|}{Fine-Tuned}\\
			\cline{2-4}
			{}			& UCF		& HMDB		& Something			\\
			\hline
			\multirow{2}{*}{Kinetics}		& Top-1 : 92.6		& Top-1 : 62.0		& Top-1 : 48.6		\\
			{}						& Top-5 : 99.2		& Top-5 : 88.2		& Top-5 : 77.9		\\
			\hline
			\multirow{2}{*}{Moments}		& Top-1 : 91.9		& Top-1 : 65.9		& Top-1 : 50.0		\\
			{}						& Top-5 : 98.6		& Top-5 : 89.3		& Top-5 : 78.8		\\
			\hline
		\end{tabular}
		\caption{Data transfer performance ของโมเดล Resnet50 I3D}
		\label{tab: Data transfer performance ของโมเดล Resnet50 I3D}
	\end{table}
				\item จะเห็นได้ว่า Kinetics ให้ผลลัพท์ที่ดีกว่าใน UCF เพราะว่ามีการแชร์ class ด้วยกันอยู่หลายอย่าง ในขณะที่ HMDB นั้นมีการรวบรวม source จากหลายแหล่ง และมีจำนวน class ที่หลากหลายจึงทำให้มีความใกล้เคียงกับตัวข้อมูลของ Moments in time ดังนั้นจึงเทียบผลลัพท์จาก Something ซึ่งจะทำให้เห็นว่า Moments in time มีประสิทธิภาพที่ดีกว่าและวิดีโอที่มีความยาวมากกว่า 3 วินาทีจะไม่ส่งผลกระทบกับประสิทธิภาพของ Moments in time
		\end{enumerate}
	\end{enumerate}
	\item {ปัญหาที่พบ}
	\begin{enumerate}
		\item ผลลัพท์จากการทำนายด้วยโมเดลถ้าผ่านรูปภาพที่มีรายละเอียดเยอะจะทำให้การ ทำนายโอกาสผิดนั้นค่อนข้างสูง ซึ่งปัญหานี่สามารถทำให้เกิดน้อยลงด้วยการนำวิธี Class Activation Mapping(CAM) จะเป็นการเน้นรูปภาพในส่วนที่มีข้อมูลมากที่สุดและ ทำนายผลออกมา แต่ก็ยังมีจุดที่เป็นปัญหาอยู่ เช่น การกระที่เกิดขึ้นเร็วมาก (การลื่นล้ม) จะทำให้การทำนาย นั้นมีโอกาสผิดสูงขึ้น 
	\end{enumerate}


%%%%%%%%%%%%%%%%%%%%%%%%%%%%%%%%%%%%%%%%%%%%%%%%%%%%%%%%%%%%%%%%%%%%%%%%%%%%
\end{enumerate}		

\section{เครื่องมือสำหรับการวิเคราะห์ผลวิดีโอ}
\subsection{โมเดลปัญญาประดิษฐ์}

\subsubsection*{CNN}
Convolution Neural Network (CNN) คือโมเดลปัญญาประดิษฐ์ประเภทหนึ่ง ซึ่งมักจะนำมาใช้กับงานที่เกี่ยวข้องกับรูป โดยการดึงจุดเด่นของภาพออกมา เพื่อใช้สำหรับการจำแนกประเภทของสิ่งต่าง ๆ

\begin{figure}[!ht]
	\centering
	\includegraphics[width=0.3\textwidth]{chapter2/images/CNN.png}
		\caption{ตัวอย่างโครงสร้างของ CNN}
    	\label{fig:CNN architecture}
\end{figure}

ซึ่งการคำนวณ CNN นั้นจะเริ่มจากการแยกคุณลักษณะออกมาจากรูป โดยการใช้เคอร์เนล (kernel) ซึ่งการที่เคอร์เนลมีลักษณะไม่มีเหมือนกันนั้น จะทำให้คุณลักษณะที่ดึงออกมาจากรูปแตกต่างกัน โดยปกติแล้วเคอร์เนลจะมีหลายแบบ เพราะ ใช่สำหรับการหาคุณลักษณะที่มีรูปแบบต่าง ๆ รูปแบบของเคอร์เนลจะเป็นตารางสองมิติที่มีขนาดขึ้นอยู่กับผู้สร้างที่จะออกแบบ รูปด้านล่างจะเป็นรูปตัวอย่างของเคอร์เนล

 \begin{figure}[!ht]
	\centering
	\includegraphics[width=0.3\textwidth]{chapter2/images/kernel_3x3.png}
		\caption{ตัวอย่างเคอร์เนล }
    	\label{fig:CNN architecture}
\end{figure}

เมื่อนำเคอร์เนลไปทาบกับรูปจะทำให้สามารถดึงคุณลักษณะออกมาได้ และเลื่อนตัวเคอร์เนลไปยังพิกเซลต่อไปจนครบทั้งรูป ซึ่งการเลื่อนนั้นขึ้นอยู่กับผู้สร้างว่าต้องการจะให้เลื่อนเท่าไหร แต่ระยะการเลื่อนที่มากขึ้นจะทำให้ความเกี่ยวข้องของคุณลักษณะที่ได้ออกมาน้อยลงด้วย โดยการวางเคอร์เนลเทียบบนรูปนั้นจะวางเคอร์เนลไม่ไห้เกินกรอบรูป แต่ถ้าต้องการทาบเคอร์เนลกับทุกพิกเซลในรูป สามารถทำได้โดยการพื้นที่เกินขอบรูปเท่ากับ 0 ได้ เป็นต้น คุณลักษณะที่ได้ออกมาทั้งหมดจะเรียกว่าผังคุณลักษณะ ตามรูปด่านล่างดังนี้

 \begin{figure}[!ht]
	\centering
	\includegraphics[width=0.3\textwidth]{chapter2/images/feature_map.png}
		\caption{ตัวอย่างการหาผังคุณลักษณะ }
    	\label{fig:example feature map}
\end{figure}

นอกจากนี้การทำเคอร์เนลยังมีการทำอีกแบบนึงซึ่งเรียกว่าการทำ pooling มีความสามารถในย่อรูปภาพแบบนึง ซึ่งนิยมใช้กันอยู่สองประเภทได้แก่ max pooling และ average pooling
โดยที่ max pooling เมื่อนำไปทาบกับรูป จะหาค่าที่มากที่สุดออกมา ตัวอย่างตามรูปด้านล่างดังนี้

 \begin{figure}[!ht]
	\centering
	\includegraphics[width=0.3\textwidth]{chapter2/images/max_pooling.png}
		\caption{ตัวอย่างการทำ max pooling }
    	\label{fig:example max pooling}
\end{figure} 

ในขณะที่ average pooling เมื่อนำไปเทียบกับรูป จะหาค่าเฉลี่ยของบริเวณที่เทียบออกมาตามรูปด่านล่างดังนี้

 \begin{figure}[!ht]
	\centering
	\includegraphics[width=0.3\textwidth]{chapter2/images/average_pooling.png}
		\caption{ตัวอย่างการทำ average pooling }
    	\label{fig:example average pooling}
\end{figure}

\clearpage

\subsubsection*{ResNet}
ในการสร้างโมเดลปัญญาประดิษฐ์นั้นการใช้จำนวนชั้น (layer) เยอะนั้นจะทำให้ได้คุณลักษณะของข้อมูลที่ออกมาเยอะตามไปด้วย แต่การที่คุณลักษณะของข้อมูลเยอะไม่ได้หมายความว่าโมเดลปัญญาประดิษฐ์จะให้ประสิทธิภาพที่ดีเสมอไป ซึ่งสามารถแก้ปัญหานี้ได้โดยใช้ Residual Network (ResNet) ที่เป็น Convolution Neuron Network (CNN) ประเภทหนึ่ง ที่ส่วนใหญ่จะนำมาใช้กับข้อมูลที่เป็นรูปภาพ เช่น การจดจำวัตถุ เป็นต้น โดย ResNet นี้จะสามารถทำการข้ามชั้นของ CNN ที่ไม่จำเป็นได้ โดยในชั้นที่ไม่จำเป็นจะมีการปรับ weight ให้เข้าใกล้ 0 ในขณะที่ train ข้อมูล การข้ามชั้น CNN ที่ไม่จำเป็นจะช่วยลดเวลาที่ใช้ในการ train และทำให้ประสิทธิภาพของโมเดลปัญญาประดิษฐ์ดีขึ้น

\begin{figure}[!ht]
	\centering
	\includegraphics[width=0.2\textwidth]{chapter2/images/example_resnet.png}
		\caption{ResNet}
    	\label{fig:ResNet}
\end{figure}

การทดลองโมเดลปัญญาประดิษฐ์ ResNet ด้วยการทำจำแนกรูปภาพโดยใช้ชุดข้อมูลของ ImageNet ที่ประกอบไป class มากว่า 1,000 class มาเทียบกับโมเดลปัญญาประดิษฐ์ทั่วไป (plain) ที่จำนวนชั้น 18 ชั้น และ 34 ชั้น ผลลัพท์จะได้ออกมาตามตารางด้านล่างดังนี้ (โครงสร้างพื้นฐานของโมเดลปัญญาดิษฐ์ ResNet และโมเดลปัญญาประดิษฐ์ทั่วไปเหมือนกัน)

\begin{table}[!ht]
	\centering
	\begin{tabular}{|c|c|c|}
		\hline
		{จำนวนชั้นของ}&\multicolumn{2}{c|}{training error}\\
		\cline{2-3}
		{}							& plain						& ResNet				\\
		\hline
		18							& 27.94						& 27.88				\\
		34							& 28.54						& 25.03				\\
		\hline
	\end{tabular}
	\caption{Top-1 ของความผิดพลาดของชุดข้อมูลทดสอบ ImageNet}
	\label{tab: Top-1 error of ImageNet}
\end{table}

จากตาราง \ref{tab: Top-1 error of ImageNet} จะเห็นได้ว่าโมเดลปัญญาประดิษฐ์ทั่วไป 34 ชั้นมีค่าเปอเซนต์ความผิดพลาดสูงกว่าโมเดลปัญญาประดิษฐ์ ResNet แบบชัดเจน ในขณะที่โมเดลปัญญาประดิษฐ์ทั่วไปจะมีเปอเซนต์ความผิดพลาดสูงขึ้นเมื่อเทียบกันระหว่าง 18 ชั้นและ 34 ชั้น
\par
ต่อมาจะนำโมเดลปัญญาประดิษฐ์ ResNet มาทดสอบกับชุดข้อมูล CIFAR-10 ซึ่งเป็นชุดข้อมูลที่มีภาพสำหรับ train 50,000 ภาพ ภาพสำหรับทดสอบ 10,000 ภาพ และมีจำนวน class ทั้งหมด 10 class โดยจะมีการออกแบบของจำนวนชั้นของโมเดลปัญญาประดิษฐ์ ResNet ตามจำนวนของชั้น Convolution ที่มีผังคุณลักษณะเท่ากัน 6 ชั้นติดกันและการข้ามชั้นที่ละ 2 จึงทำให้ได้รูปแบบการคิดชั้นดังนี้ 6n + 2 สำหรับการทดสอบจะให้ค่า n = [3,5,7,9,200] ดังตารางต่อไปนี้

\begin{table}[!ht]
	\centering
	\begin{tabular}{|c|c|c|}
		\hline
		{โมเดลปัญญาประดิษฐ์}				&{จำนวนชั้น}					&{training error}		\\
		\hline
		ResNet						& 20							& 8.75				\\
		ResNet						& 32							& 7.51				\\
		ResNet						& 44							& 7.17				\\
		ResNet						& 56							& 6.97				\\
		ResNet						& 110						& 6.43				\\
		ResNet						& 1202						& 7.93				\\
		\hline
	\end{tabular}
	\caption{ค่าความผิดพลาดที่ได้จากการทดลองจำนวนชั้นของโมเดลปัญญาประดิษฐ์ ResNet บนชุดของข้อมูล CIFAR-10}
	\label{tab: Classification error}
\end{table}
จากตาราง \ref{tab: Classification error} จะเห็นได้ว่าที่ีโมเดลปัญญาประดิษฐ์ ResNet ที่มีจำนวนชั้น 1202 นั้นมีค่าความผิดพลาดเกิดขึ้นมากกว่าจำนวนชั้น 110 ซึ่งอาจจะเป็นไปได้ว่าขนาดของโมเดลปัญญาประดิษฐ์ ResNet ที่มีจำนวนชั้น 1202 นั้นมากเกินไปสำหรับชุดข้อมูลขนาดเล็กนี้

\clearpage
\subsubsection*{Inflated 3D convolutional network}
ในการพัฒนาโมเดลปัญญาประดิษฐ์สำหรับจำแนกการกระทำของมนุษย์นั้นมีพื้นฐานมาจากการจำแนกวัตถุ (object classification)
หมายถึงการใช้รูปภาพหนึ่งรูปในการประมวลผลและทำนายออกมาว่าภายในรูปนั้นมีบริบทการกระทำอย่างไร โดยไม่ได้คำนึงถึงข้อมูลเชิงต่อเนื่อง (spatio-temporal information)
จากบทความ "Quo Vadis, Action Recognition? A New Model and the Kinetics Dataset"\textsuperscript{\cite{I3D}} นั้นได้พัฒนาโครงสร้างของโมเดลปัญญาประดิษฐ์ (architecture) 
ที่มีประสิทธิภาพในการประมวลผลภาพเคลื่อนไหวได้ชื่อว่า I3D หรือ inflated 3D-convolution network
โดยโครงสร้างพื้นฐานของ I3D นั้นมาจาก Inception-v1\textsuperscript{\cite{Inception}} ที่ถูกพัฒนาโดย Goggle ซึ่งเป็นโครงสร้างที่มีประสิทธิภาพสูงในการจำแนกวัตถุในรูปภาพ
แล้ว I3D นั้นได้ทำการขยายมิติของโครงสร้างจาก 2 มิติ เป็น 3 มิติ เพื่อให้โมเดลปัญญาประดิษฐ์สามารถเรียนรู้ข้อมูลเชิงต่อเนื่องได้
\begin{figure}[!ht]
    \centering
    \includegraphics[width=1.0\textwidth]{chapter2/images/I3D.png}
    \caption{โครงสร้างของโมเดลปัญญาประดิษฐ์ I3D\textsuperscript{\cite{I3D}}}
    \label{fig:I3DArch}
\end{figure}

\begin{figure}[!ht]
    \centering
    \includegraphics[width=0.7\textwidth]{chapter2/images/inceptionModule.png}
    \caption{โครงสร้างของโมเดลปัญญาประดิษฐ์ I3D\textsuperscript{\cite{I3D}}}
    \label{fig:I3DArch}
\end{figure}

\clearpage
ประสิทธิภาพของโมเดล I3D แบบ two-stream เมื่อเทียบกับ long-short term memory (LSTM),
3D-convolution network, two-stream และ 3D-fused โดยใช้เครื่องมือในการวัดผลคือความแม่นยำจากการทำนายอันดับแรกสุด (Top@1 accuracy) 
และความแม่นยำจากการทำนาย 5 อันดับแรก (Top@5 accuracy) ตามตารางที่ \ref{tab:I3DPerformance}
\begin{table}[ht]
    \begin{tabular}{|*{10}{c|}}
        \hline
        \multirow{2}{*}{Architecture} & \multicolumn{3}{c|}{UCF-101} & \multicolumn{3}{c|}{HMDB-51} & \multicolumn{3}{c|}{Kinetics}\\
        \cline{2-10}
            & RGB & Flow & RGB + Flow & RGB & Flow & RGB + Flow & RGB & Flow & RGB + Flow\\
        \hline\hline
        LSTM            & 81.0 & – & – & 36.0 & – & – & 63.3 & – & –\\
        3D-ConvNet      & 51.6 & – & – & 24.3 & – & – & 56.1 & – & –\\
        Two-Stream      & 83.6 & 85.6 & 91.2 & 43.2 & 56.3 & 58.3 & 62.2 & 52.4 & 65.6\\
        3D-Fused        & 83.2 & 85.8 & 89.3 & 49.2 & 55.5 & 56.8 & – & – & 67.2\\
        Two-Stream I3D  & 84.5 & 90.6 & 93.4 & 49.8 & 61.9 & 66.4 & 71.1 & 63.4 & 74.2\\
        \hline
    \end{tabular}
    \caption{ประสิทธิภาพของโมเดล I3D แบบ two-stream เมื่อใช้ข้อมูลจาก UCF-101, HMDB-51 และ Kinetics ในการสร้างและทดสอบด้วยเครื่องมือวัดผลแบบความแม่นยำจากการทำนายอันดับแรกสุด}
    \label{tab:I3DPerformance}
\end{table}
\clearpage

\subsubsection*{YOLO}
\begin{figure}[!ht]
    \centering
    \includegraphics[width=0.7\textwidth]{chapter2/images/yolo.jpg}
    \caption[กระบวนการทำงานของโครงสร้างโมเดลปัญญาประดิษฐ์ของ YOLO]{กระบวนการทำงานของโครงสร้างโมเดลปัญญาประดิษฐ์ของ YOLO\textsuperscript{\cite{yolo}}}
    \label{fig:yolo}
\end{figure}

โครงสร้างโมเดลปัญญาประดิษฐ์ของ YOLO\textsuperscript{\cite{yolo}} เป็นโครงสร้างที่มีความเร็วในการประมวลผลถึง 45 เฟรมต่อวินาที ทำให้สามารถประมวลผลแบบเรียลไทม์ได้ นอกจากนั้นยังมีความแม่นยำ mAP 
มากกว่าโมเดลสำหรับตรวจจับวัตถุอื่นๆถึง 2 เท่า ซึ่งเหตุผลที่โครงสร้างโมเดลปัญญาประดิษฐ์ของ YOLO เร็วกว่าโมเดลปัญญาประดิษฐ์ตัวอื่นๆ เนื่องจาก
การตรวจจับวัตถุในวิธีการก่อนหน้าจะใช้วิธีทำนายกรอบสี่เหลี่ยมก่อน แล้วจึงค่อยนำกรอบสี่เหลี่ยมไปทำนายว่าเป็นหมวดหมู่อะไร ซึ่ง YOLO มีวิธีการที่ต่างออกไป คือ 
ทำนายตำแหน่งของกรอบสี่เหลี่ยมและทำนายว่าเป็นหมวดหมู่อะไรพร้อมกัน โดยใช้โครงข่ายประสาทแบบคอนโวลูชัน ด้วยแนวคิดนี้จึงเป็นที่มาของชื่อ YOLO 
หรือ you only look once
\subsubsection*{โครงสร้างของโมเดลปัญญาประดิษฐ์ของ YOLO} 
\begin{figure}[!ht]
    \centering
    \includegraphics[width=0.7\textwidth]{chapter2/images/yolo_architecture.jpg}
    \caption[โครงสร้างทั่วไปของโมเดลปัญญาประดิษฐ์ของ YOLO]{โครงสร้างทั่วไปของโมเดลปัญญาประดิษฐ์ของ YOLO\textsuperscript{\cite{ssd_yolo_pic}}}
    \label{fig:yolo_architecture}
\end{figure}

จากรูปภาพที่ \ref{fig:yolo_architecture} จะเห็นได้ว่า YOLO ใช้โครงข่ายประสาทเทียมเพียงตัวเดียวซึ่งภายในโครงข่ายจะมีกระบวนการหลักๆอยู่สามอย่าง 
กระบวนการแรกคือการสกัดคุณลักษณะ กระบวนการนี้จะมีจำนวนชั้นของเลเยอร์ที่แตกต่างกันไปตามความลึกของการสกัดแล้วแต่โมเดล ซึ่งตัวอย่างจะเป็นดังรูปที่ \ref{fig:yolo-architecture} 
และขั้นตอนถัดมาคือการทำนายผล หลังจากที่ได้คุณลักษณะมาแล้วจะนำไปทำนายผลผ่านชั้น fully connected ซึ่งจะได้ผลลัพธ์เป็นหมวดหมู่และตำแหน่งของกรอบสี่เหลี่ยม 
และขั้นตอนสุดท้ายคือการทำ NMS เพื่อให้ได้ผลลัพธ์ที่ดีที่สุดออกมา
\clearpage
\par ซึ่งโครงสร้างโมเดลปัญญาประดิษฐ์ของ YOLO ที่ถูกใช้ในงานวิจัยนี้ประกอบไปด้วย YOLO-v3 tiny, YOLO-v3 และ YOLO-v3 spp ซึ่งทั้งสามโครงสร้างจะแตกต่างดังนี้
\begin{enumerate}
	\setlength\itemsep{-0.25em}
	\item YOLO-v3 tiny ใช้ชั้น max pooling ในขั้นตอนของการลดจำนวนข้อมูลตัวอย่าง
	\item YOLO-v3 ใช้ชั้นคอนโวลูชั่น ในขั้นตอนของการลดจำนวนข้อมูลตัวอย่าง
	\item YOLO-v3 spp ใช้ชั้นคอนโวลูชั่น และคุณลักษณะที่ดีที่สุดของ max pooling ในขั้นตอนของการลดจำนวนข้อมูลตัวอย่าง
\end{enumerate}

\begin{figure}[!ht]
    \centering
    \begin{subfigure}[b]{0.2\textwidth}
        \centering
        \includegraphics[width=\textwidth]{chapter2/images/yolo_tiny.jpg}
	 \caption[โครงสร้างโมเดลปัญญาประดิษฐ์ของ YOLO-v3 tiny]{โครงสร้างโมเดลปัญญาประดิษฐ์ของ YOLO-v3 tiny\textsuperscript{\cite{tiny_pic}}}
        \label{fig:tiny}
    \end{subfigure}
    \begin{subfigure}[b]{0.3\textwidth}
        \centering
        \includegraphics[width=\textwidth]{chapter2/images/yolo_darknet.png}
	 \caption[โครงสร้างโมเดลปัญญาประดิษฐ์ของ YOLO-v3]{โครงสร้างโมเดลปัญญาประดิษฐ์ของ YOLO-v3\textsuperscript{\cite{darknet_pic}}}
       \label{fig:darknet}
    \end{subfigure}
    \begin{subfigure}[b]{0.5\textwidth}
        \centering
        \includegraphics[width=\textwidth]{chapter2/images/yolo_spp.png}
	 \caption[โครงสร้างโมเดลปัญญาประดิษฐ์ของ YOLO-v3 spp]{โครงสร้างโมเดลปัญญาประดิษฐ์ของ YOLO-v3 spp\textsuperscript{\cite{spp_pic}}}
       \label{fig:spp}
    \end{subfigure}
    \caption{โครงสร้างโมเดลปัญญาประดิษฐ์ของ YOLO}
    \label{fig:yolo-architecture}
\end{figure}


\clearpage

\subsubsection*{Faster-RCNN}
\begin{figure}[!ht]
    \centering
    \includegraphics[width=0.9\textwidth]{chapter2/images/faster_rcnn.png}
    \caption{โครงสร้างทั่วไปของโมเดลปัญญาประดิษฐ์ของ Faster RCNN}
    \label{fig:faster_rcnn_architecture}
\end{figure}
\par faster-rcnn มีการพัฒนาในการหาพื้นที่ที่สนใจ (ROI) โดยการเปลี่ยนจากใช้โครงข่ายหาพื้นที่ที่สนใจแยกเฉพาะ (selective search) นำมารวมในโครงข่ายเดียวกัน ดังนั้น faster-rcnn จึงมีโครงข่ายประสาทเทียมเดียวในการทำงาน ซึ่งภายในโครงข่ายจะประกอบไปด้วยการทำงานหลัก 3 อย่าง คือ
\begin{enumerate}
	\setlength\itemsep{-0.25em}
	\item การสกัดคุณลักษณะ
	\\	นำรูปภาพทั้งรูปภาพเข้าโครงข่ายคอนโวลูชั่นเพื่อการสกัดคุณลักษณะของรูปภาพ
	\item การเสนอพื้นที่ที่คาดว่าจะมีวัตถุอยู่
	\\	หลังจากที่รูปภาพผ่านการสกัดคุณลักษณะแล้ว จะถูกนำเข้าไปใน region proposal network เพื่อสร้างข้อเสนอพื้นที่ที่คาดว่าจะมีวัตถุอยู่
	\item การทำนายผล
	\\	ทำการ pooling คุณลักษณะของรูปภาพและพื้นที่ที่คาดว่าจะมีวัตถุอยู่ และ นำเข้าไปในชั้นการทำนายผล (full connected layer) สุดท้ายจะได้ผลลัพธ์เป็นหมวดหมู่ของกรอบสี่เหลี่ยม และ ตำแหน่งของกรอบสี่เหลี่ยม  
\end{enumerate}
\par region proposal network (RPN) คือ โครงข่ายที่เสนอพื้นที่ที่คาดว่าจะมีวัตถุอยู่ จะถูกใช้หลังรูปภาพผ่านการสกัดคุณลักษณะ RPN มีโครงสร้างที่เป็นเอกลักษณ์เฉพาะตัว
คือมีการบอกว่าบริเวณนั้นมีวัตถุอยู่หรือไม่ (classification layer) และ สำหรับการระบุพิกัดของกรอบสี่เหลี่ยมที่คาดว่าจะมีวัตถุอยู่ (regression layer) ซึ่งผลลัพธ์จะได้ ROI (พื้นที่บริเวณที่เราสนใจ)
\clearpage


\subsection{เครื่องมือกำกับคุณลักษณะ}
จากการค้นคว้าหาเครื่องมือในการสร้างคำกำกับข้อมูลเพื่อใช้เป็นแนวทางในการออกเครื่องมือกำกับข้อมูลด้วยปัญญาประดิษฐ์ พบเครื่องมือที่เปิดให้ใช้งานสาธารณะ 2 โปรแกรม 
คือ DarkLabel และ OpenLabeling โดยสรุปข้อมูลสำคัญได้ดังนี้ 
\subsubsection*{โปรแกรม DarkLabel\textsuperscript{\cite{dark}}}
เป็นโปรแกรมที่ช่วยในการทำนายคำกำกับและบันทึกในรูปแบบต่างๆ รองรับข้อมูลป้อนเข้าในรูปแบบไฟล์วิดีโอ avi, mpg หรือกลุ่มรูปภาพ มีขั้นตอนการสร้างคำกำกับดังนี้ 
\begin{enumerate}
	\setlength\itemsep{-0.25em}
	\item สร้างกรอบสี่เหลี่ยมครอบบริเวณวัตถุที่สนใจโดยใช้มนุษย์เป็นคนสร้าง
	\item กดปุ่ม Next และ Predict อย่างต่อเนื่อง เพื่อทำนายตำแหน่งต่อไปของกรอบสี่เหลี่ยมในเฟรมถัดๆไป จนกระทั่งการเกิดข้อผิดพลาด
	\item ลบกรอบสี่เหลี่ยมที่พลาด และเริ่มทำขั้นตอนที่ 1 ใหม่ อีกครั้งจนครบทุกเฟรมในวิดีโอ
\end{enumerate}

หลังจากที่ผู้วิจัยได้ทดลองใช้โปรแกรม DarkLabel พบว่าโปรแกรมมีการทำงานส่วนใหญ่ใช้มนุษย์ในการทำด้วยตัวเอง ซึ่งทำให้ใช้เวลาในการทำนาน

\begin{figure}[!ht]
	\centering
	\includegraphics[width=0.7\textwidth]{chapter2/images/darklabel.png}
		\caption{หน้าต่างการทำงานของโปรแกรม DarkLabel}
    	\label{fig:darklabel}
\end{figure}
\clearpage

\subsubsection*{โปรแกรม OpenLabeling\textsuperscript{\cite{open}}}
เป็นโปรแกรมที่ช่วยในการสร้างคำกำกับ โดยโปรแกรมจะมีการทำงานอยู่ 2 รูปแบบการทำงาน คือแบบทำด้วยตัวเอง (Mode Manual) และแบบอัตโนมัติ (Mode Auto) 
ซึ่งมีการทำงานแยกจากกันอย่างชัดเจน 

\begin{enumerate}
	\setlength\itemsep{-0.25em}
	\item การทำงานแบบอัตโนมัติ 
	\\ หลังจากป้อนวิดีโอเข้าไปในโปรแกรมแล้วมีขั้นตอนการสร้างคำกำกับดังนี้ 
   	\begin{enumerate}
	\setlength\itemsep{-0.25em}
		\item โปรแกรมจะทำงานอัตโนมัติโดยใช้โมเดลปัญญาประดิษฐ์ในการทำนายคีย์เฟรม (keyframe) 
		และทำนายตำแหน่งต่อไปของกรอบสี่เหลี่ยมในเฟรมถัดไปด้วยอัลกอริทึมที่ใช้การคำนวณคณิตศาสตร์และการประมวลผลภาพในภาพที่เหลือ ผลลัพธ์ที่ได้คือรูปภาพและไฟล์คำกำกับภาพ
 	\end{enumerate}
	\item การทำงานแบบทำด้วยตัวเอง 
	\\ หลังจากป้อนวิดีโอเข้าไปในโปรแกรมแล้วมีขั้นตอนการสร้างคำกำกับดังนี้ 
	\begin{enumerate}
	\setlength\itemsep{-0.25em}
		\item สร้างกรอบสี่เหลี่ยมขึ้นมาโดยใช้มนุษย์เป็นคนสร้าง
		\item กดปุ่มเพื่อทำนายตำแหน่งต่อไปของกรอบสี่เหลี่ยมในเฟรมถัดไป จนกระทั่งเกิดข้อผิดพลาด
		\item ลบกรอบสี่เหลี่ยมที่พลาด และเริ่มทำขั้นตอนที่ 1 อีกครั้งจนครบทุกเฟรมในวิดีโอ
 	\end{enumerate}
 \end{enumerate}

หลังจากที่ได้ทดลองใช้โปรแกรม OpenLabeling ทั้ง 2 รูปแบบการทำงานแล้วพบว่า การทำงานแบบอัตโนมัติไม่สามารถปรับแก้ไขสิ่งใดในระหว่างกระบวนการนั้น 
ทำให้หากเกิดกรณีที่โมเดลทำนายกรอบสี่เหลี่ยมพลาดจะไม่สามารถแก้ไขได้ และการทำงานแบบทำด้วยตัวเองผู้ใช้งานจะต้องสร้างกรอบสี่เหลี่ยมขึ้นมาเอง

\begin{figure}[!ht]
	\centering
	\includegraphics[width=0.7\textwidth]{chapter2/images/openlabel.png}
		\caption{หน้าต่างการทำงานของโปรแกรม OpenLabeling}
    	\label{fig:openlabel}
\end{figure}




\clearpage
%\section{ชุดข้อมูล}
%\subsubsection*{Youtube-8M}
YouTube-8M คือชุดข้อมูลวิดีโอที่เป็น multi-label ที่มีจำนวนวิดีโอเยอะที่สุด ซึ่งมีจำนวนมากถึง 8 ล้านวิดีโอ(ในปี 2016) โดยมีจุดมุ่งหมายหลักในการอธิบายธีมหลักของวิดีโอด้วยคำสั้นๆ เช่น ถ้าวิดีโอนั้นเป็นวิดีโอที่มี มนุษย์กำลังปั่นจักรยานบนถนนดินกับหน้าผา ชุดข้อมูลนี้จะอธิบายวิดีโอนี้ว่า mountain biking ซึ่งทำให้ YouTube-8M แตกต่างจากชุดข้อมูลวิดีโออื่นๆส่วนใหญ่ที่จะเน้น action หรือ activity ของมนุษย์ ซึ่งข้อมูลเชิงสถิติจะเป็นดังตารางที่ 1

\begin{figure}[!ht]
	\centering
	\includegraphics[width=1\textwidth]{chapter2/images/youtube-8m.png}
		\caption{ตัวอย่าง catagories ต่างๆของ YouTube-8M}
    	\label{fig:youtube-8m}
\end{figure}

\begin{table}[!ht]
\begin{tabular}{|c|c|c|c|}
		\hline
		{Number of video}&{Class of video}&{Avg. length of each video(s.)}&{Avg. class of video}				\\
		\hline
		8,264,650		& 4800		& 229.6		& 1.8											\\
		\hline
	\end{tabular}
	\caption{ข้อมูลเชิงสถิติของ YouTube-8M}
	\label{tab: ข้อมูลเชิงสถิติของ YouTube-8M}
\end{table}

\subsubsection*{1. วิธีการรวบรวมข้อมูล}
การเก็บข้อมูลของ YouTube-8M นั้นใช้เครื่องมือที่ชื่อว่า YouTube annotation system ในการเก็บรวบรวมข้อมูลโดยอาศัยผังความรู้(knowledge graph)ของ Google ในการค้นหาและรวบรวมข้อมูลในฐานข้อมูลของ YouTube
\begin{enumerate}
	\setlength\itemsep{-0.25em}
	\item กฏในการรวบรวมข้อมูลดังนี้
	\begin{enumerate}
		\setlength\itemsep{-0.25em}
		\item ทุกๆ หัวข้อต้องเป็นรูปธรรม
		\item ในแต่ละหัวข้อต้องมีจำนวนวิดีโอไม่น้อยกว่า 200 วิดีโอ
		\item ความยาวของวิดีโอต้องอยู่ระหว่าง 120 - 500 วินาที
	\end{enumerate}
หลังจากได้กฏในการรวบรวมข้อมูลแล้ว ขั้นตอนต่อไปคือการสร้างคำศัพท์(vocabulary)ที่ใช้ในการค้นหาข้อมูลวิดีโอจากใน YouTube 
	\item ขั้นตอนในการสร้างคำศัพท์มีดังนี้
	\begin{enumerate}
		\setlength\itemsep{-0.25em}
		\item กำหนด whitelist หัวข้อที่เป็นรูปธรรมมา 25 ชนิด เช่น game เป็นต้น
		\item กำหนด blacklist หัวข้อที่คิดว่าไม่เป็นรูปธรรมไว้ เช่น software เป็นต้น
		\item รวบรวมหัวข้อที่มีอยู่ใน whitelist อย่างน้อย 1 หัวข้อ และต้องไม่มีอยู่ใน blacklist ซึ่งจะทำให้ได้หัวข้อที่ต้องการมาประมาณ 50,000 หัวข้อ
		\item จากนั้นใช้ผู้ประเมินจำนวน 3 คน ในการคัดหัวข้อที่คิดว่าเป็นรูปธรรม และสามารถจดจำหรือเข้าใจได้ง่ายโดยไม่ต้องเชี่ยวชาญในด้านนั้นๆ ซึ่งผู้ประเมิน ก็จะมีคำถามว่า “ มันยากขนาดไหนถึงจะระบุได้ว่ามีหัวข้อดังกล่าวอยู่ในรูปหรือวิดีโอ โดยใช้เพียงแค่การมองรูปภาพเท่านั้น? ” โดยแบ่งเป็นระดับดังนี้
		\begin{enumerate}
			\setlength\itemsep{-0.25em}
			\item บุคคลทั่วไปสามารถเข้าใจได้
			\item บุคคลทั่วไปที่ผ่านการอ่านบทความที่เกี่ยวข้องมาแล้วสามารถเข้าใจได้
			\item ต้องเชี่ยญในด้านใดซักด้านจึงจะเข้าใจได้
			\item เป็นไปไม่ได้ ถ้าไม่มีความรู้ที่ไม่ได้เป็นรูปธรรม
			\item ไม่เป็นรูปธรรม
		\end{enumerate}
		\item หลังจากคำถามข้างบนและการให้คะแนน จะทำการเก็บไว้เฉพาะหัวข้อที่มีคะแนนเฉลี่ยมากที่สุดอยู่ที่ประมาณ 2.5 คะแนนเท่านั้น
		\item ทำให้สุดท้ายเหลือเพียงประมาณ 10,000 หัวข้อที่สามารถใช้ได้
		\item หลังจากได้หัวข้อที่คิดว่าเป็นรูปธรรมแล้วก็นำไปค้นหาและรวบรวมด้วย YouTube annotation system โดยมีขั้นตอนดังนี้
		\begin{enumerate}
			\setlength\itemsep{-0.25em}
			\item สุ่มเลือกวิดีโอมา 10 ล้านวิดีโอ พร้อมกับหัวข้อของวิดีโอ โดยใช้กฏที่กำหนดไว้ เอาหัวข้อที่มีจำนวนวิดีโอน้อยกว่า 200 วิดีโอออก
			\item ทำให้เหลือจำนวนวิดีโออยู่ 8,264,650 วิดีโอ
			\item แยกออกเป็น 3 ส่วน Train set, Validate set และ Test set ในอัตราส่วน 70:20:10 ตามลำดับ
		\end{enumerate}
	\end{enumerate}
\end{enumerate}

เนื่องจากชุดข้อมูลนี้มีขนาดมากกว่า 100 Terabytes และมีความยาวรวมประมาณ 500,000 ชั่วโมง ทำให้การจะใช้คอมพิวเตอร์ทั่วไปเปิดอาจจะใช้เวลานานถึง 50 ปี ทำให้ Google ทำการลดขนาดของข้อมูลลงโดยมีขั้นตอนดังนี้
\begin{figure}[!ht]
	\centering
	\includegraphics[width=1\textwidth]{chapter2/images/decrease_data.png}
		\caption{ขั้นตอนกระบวนการการลดขนาดของชุดข้อมูลให้สามารถใช้งานได้ง่ายยิ่งขึ้น}
    	\label{fig:decrease_data}
\end{figure}




\subsubsection*{2. การทดลองและวิเคราะห์ผล}
ในบทความ \footnote{YouTube-8M,https://arxiv.org/pdf/1609.08675.pdf} นั้นได้นำเสนอวิธีการในการจัดการข้อมูลซึ่งแบ่งเป็น 2 รูปแบบตามลักษณะของข้อมูลที่ใช้ และอัลกอริทึมหรือเทคนิคที่ใช้ในการสร้างโมเดล ดังนี้
\begin{enumerate}
	\setlength\itemsep{-0.25em}
	\item คุณลักษณะระดับเฟรม (Frame-level feature)
	\begin{enumerate}
		\setlength\itemsep{-0.25em}
		\item Frame-Level Models and Average Pooling
		\\ อันดับแรกเนื่องจากว่าชุดข้อมูลนี้ไม่มีการระบุหัวข้อในระดับเฟรม จึงใช้วิธีการนำหัวข้อในระดับวิดีโอ มากำหนดให้กับทุกๆเฟรมในวิดีโอแทน จากนั้นสุ่มเฟรมมา 20 เฟรมในแต่ละวิดีโอ ทำให้มีเฟรมถึง 120 ล้านเฟรม ซึ่งในแต่ละหัวข้อ $e$ ทำให้มี $(x_{i}, y_{i}^{e})$ 120 ล้านคู่ โดยที่ $x_{i} \epsilon  R^{1024}$ คือ คุณลักษณะที่ได้มาจาก hidden layer สุดท้ายก่อนจะเป็น fully connected และ $y_{i}^{e} \epsilon  0,1$ คือหัวข้อสำหรับหัวข้อ $e$ ของตัวอย่างที่ $i^{th}$ แล้วสร้างโมเดลทั้งหมด 4,800 โมเดลที่เป็นโมเดลแบบ one vs all classifier และเป็นอิสระต่อกันสำหรับแต่ละหัวข้อ และเนื่องจากการประเมินผลนั้นมีพื้นฐานมาจากหัวข้อในระดับวิดีโอ ทำให้ต้องทำการรวมความน่าจะเป็นของแต่ละหัวข้อในระดับเฟรมไปเป็นความน่าจะเป็นในระดับวิดีโอ โดยใช้การเฉลี่ยค่าความน่าจะเป็นทั้งหมดในหัวข้อนั้นๆ และใช้ average pooling เพื่อลดผลจากการตรวจจับความผิดปกติและความโดดเด่นของข้อมูลของแต่ละหัวข้อภายในวิดีโอ
		
		\item Deep Bag of Frames (DBoF) Pooling
		\begin{figure}[!ht]
			\centering
			\includegraphics[width=0.5\textwidth]{chapter2/images/DBoF.png}
				\caption{โครงสร้างของโมเดล DBoF}
    			\label{fig:DBoF}
		\end{figure}
		\\ หลักการคล้ายๆกับ Deep Bag of Words โดยที่จะสุ่มเฟรม มา k เฟรม โดยที่แต่ละเฟรมเป็น N dimension input มาผ่าน fully connected ที่มี M units (M > N) และใช้ RELU activations แล้วทำ batch normalization ก่อนจะนำมารวมด้วย max pooling โดยที่ทั้งโครงข่ายใช้ Stochastic  Gradient Descent(SGD) 
		\clearpage
		\item Long short-term memory(LSTM)
		\\ ในบทความ \footnote{YouTube-8M,https://arxiv.org/pdf/1609.08675.pdf} นี้ได้ใช้ LSTM แบบเดียวกับของ Beyond Short Snippets: Deep Networks for Video Classification \footnote{AVA,https://arxiv.org/pdf/1705.08421.pdf} แต่เนื่องจาก YouTube-8M นั้นผ่านการทำ preprocess มาแล้วทำให้ไม่สามารถใช้ raw video frame ได้ จึงทำได้เฉพาะ LSTM และ softmax layer เท่านั้น ตามรูปที่ \ref{fig:BSS}
		\begin{figure}[!ht]
			\centering
			\includegraphics[width=1\textwidth]{chapter2/images/BSS.png}
				\caption{(ซ้าย) โครงสร้างจาก Beyond Short Snippets: Deep Networks for Video Classification, (ขวา) ส่วนที่สามารถใช้งานกับ YouTube-8M ได้}
    			\label{fig:BSS}
		\end{figure}
	\end{enumerate}
	\item คุณลักษณะระดับวิดีโอ (Video-level feature)
	\begin{enumerate}
		\setlength\itemsep{-0.25em}
		\item Video-level representation 
		\\ ในบทความ \footnote{YouTube-8M,https://arxiv.org/pdf/1609.08675.pdf}นี้ได้สำรวจวิธีการแยกเวกเตอร์คุณลักษณะระดับวิดีโอความยาวคงที่จากคุณลักษณะระดับเฟรมซึ่งการทำแบบนี้ทำให้ได้ประโยชน์ 3 ข้อ คือ 1) โมเดลทั่วไปที่ไม่ใช่ neural network สามารถนำไปใช้งานได้  2) ขนาดข้อมูลเล็กลง  3) เหมาะกับการนำไปสร้างโมเดล domain adaptive มากขึ้น
		\begin{enumerate}
			\setlength\itemsep{-0.25em}
			\item First, Second order and ordinal statistic
			\\ จากคุณลักษณะในระดับเฟรม $x_{1:F_{v}}^{v}$ โดยที่ $x_{j}^{v}$ คือคุณลักษณะระดับเฟรมในเฟรมที่ $j$ ของวิดีโอ $v$ และ $F_{v}$ คือจำนวนเฟรมทั้งหมดของวิดีโอ $v$ ทำการหาค่าเฉลี่ย $\mu_{v}$ และส่วนเบี่ยงเบนมาตรฐาน $\sigma_{v}$ พร้อมทั้งดึง ordinal statistics 5 อันดับแรกของแต่ละ dimension K ออกมา $Top_{k}(x^{v}(j)_{1:F_{v}})$ จะทำให้ได้เวคเตอร์คุณลักษณะ(feature-vector) $\varphi_{1:F_{v}}^{v}$ ของวิดีโอเป็นดังนี้ \\
			\centerline{$\varphi_{1:F_{v}}^{v} = \begin{bmatrix}
								\mu_{1:F_{v}}^{v}\\ 
								\sigma_{1:F_{v}}^{v}\\ 
								Top_{k}(x^{v}(j)_{1:F_{v}})
								\end{bmatrix}$}
			\item Feature normalization \\
			ก่อนที่จะทำการสร้าง one vs all classifiers แต่ละตัวนั้นได้ทำ normalization เวกเตอร์คุณลักษณะ $\varphi_{1:F_{v}}^{v}$ จากนั้นนำค่าเฉลี่ย $\mu_{v}$ ออกแล้วใช้ PCA ในการลด มิติของข้อมูล ซึ่งการทำแบบนี้นั้นทำให้การสร้างโมเดลเป็นไปได้เร็วขึ้น
		\end{enumerate}
		โดยการสร้างโมเดลด้วย video-level presentation นั้น บทความ \footnote{YouTube-8M,https://arxiv.org/pdf/1609.08675.pdf} นี้ได้หยิบมาทดสอบ 3 อัลกอริทึม
		\item Model training algorithm approaches 
		\begin{enumerate}
			\setlength\itemsep{-0.25em}
			\item Logistic Regression
			\item Hinge Loss
			\item Mixture of Experts (MoE)
		\end{enumerate}
		\item Evaluation metrics
		\begin{enumerate}
			\setlength\itemsep{-0.25em}
			\item Mean Average Precision (mAP)
			\item Hit@k
			\item Precision at equal recall rate (PERR)
		\end{enumerate}
	\end{enumerate}
	\item Results
	\begin{enumerate}
		\setlength\itemsep{-0.25em}
		\item Baseline on YouTube-8M dataset
\begin{table}[!ht]
\centering
\begin{tabular}{|c|c|c|c|c|}
		\hline
		{Inpute Features}&{Modeling Approach}&{mAP}&{Hit@1}&{(PERR)}\\
		\hline
		Frame-level, $(x_{1:F_{v}}^{v})$	& Logistic + Average		& 11.0		& 50.8		& 42.2					\\
		Frame-level, $(x_{1:F_{v}}^{v})$	& Deep Bag of Frames	& 26.9		& 62.7		& 55.1					\\
		Frame-level, $(x_{1:F_{v}}^{v})$	& LSTM				& 26.6		& 64.5		& 57.3					\\
		\hline
		Video-level, $\mu$					& Hinge loss					& 26.6		& 64.5		& 57.3				\\
		Video-level, $\mu$					& Logistic Regression				& 26.6		& 64.5		& 57.3				\\
		Video-level, $\mu$					& Mixture-of-2-Expert				& 26.6		& 64.5		& 57.3				\\
		Video-level, $\mu ; \sigma ; Top_{5} $	& Mixture-of-2-Expert				& 26.6		& 64.5		& 57.3				\\
		\hline
	\end{tabular}
	\caption{ประสิทธิภาพของโมเดลที่สร้างจาก YouTube-8M ด้วยวิธีต่างๆตามหัวข้อที่ 1 และ 2 โดยแถวที่ 1 คือ frame-level โมเดลและแถวที่ 2 คือ video-level โมเดล}
	\label{tab: ประสิทธิภาพของโมเดลที่สร้างจาก YouTube-8M}
\end{table}
\\
		จากตารางที่ \ref{tab: ประสิทธิภาพของโมเดลที่สร้างจาก YouTube-8M} จะเห็นว่าการทำ video-level features จากการหาค่าเฉลี่ยของ frame-level features แล้วสร้างโมเดลด้วย Hinge loss หรือ โมเดล Logistic Regression นั้นสามารถเพิ่มประสิทธิภาพได้ไม่น้อย และจากการทดลองทำให้เห็นว่า LSTM ที่มีความลึก 2 layers นั้นสามารถทำให้ผลลัพธ์เป็น state-of-the-art ในขณะนั้นได้ เนื่องจากในขณะที่ DBoF นั้นไม่ได้สนใจลำดับของเฟรม แต่ LSTM ใช้ state information เพื่อคงลำดับของเฟรมเอาไว้
\\
\\
		 LSTM นั้นดีที่สุดยกเว้น mAP, เนื่องจาก one-vs-all binary MoE classifier นั้นมีประสิทธิภาพดีกว่า, LSTM สามารถเพิ่มประสิทธิภาพบน Hit@1 และ PERR ได้เนื่องจากความสามารถในการเรียนรู้ความสัมพันธ์ระยะยาวในโดเมนของเวลา
		\clearpage
		\item Transfer learning video-level presentation from YouTube-8M to Sports-1M dataset
\begin{table}[!ht]
\centering
\begin{tabular}{|c|c|c|c|}
		\hline
		{Approach}&{mAP}&{Hit@1}&{(Hit@1)}\\
		\hline
		Logistic Regression ($\mu$)					& 58.0		& 60.1		& 79.6					\\
		Mixture-of-2-Expert ($\mu$)					& 59.1		& 61.5		& 80.4					\\
		Mixture-of-2-Expert ([$\mu ; \sigma ; Top_{5}$		& 61.3		& 63.2		& 82.6					\\
		LSTM									& 66.7		& 64.9		& 85.6					\\
		+Pretrained on YT-8M							& 67.6		& 65.7		& 86.2					\\
		\hline
		Hierarchical 3D Convolution						& -			& 61.0		& 80.0					\\
		Stacked 3D Convolutions						& -			& 61.0		& 85.0					\\
		LSTM with Optical Flow and Pixels				& -			& 73.0		& 91.0					\\
		\hline
	\end{tabular}
	\caption{ประสิทธิภาพของโมเดลเมื่อถูก transfer learning ด้วยชุดข้อมูล Sports-1M โดยใช้ video-level presentation}
	\label{tab: transfer learning}
\end{table}
\\
		จากตารางที่  \ref{tab: transfer learning} จะเห็นว่าโมเดล LSTM ที่ถูก pretrained จาก YouTube-8M นั้นมีประสิทธิภาพที่ดีกว่า ยกเว้น LSTM with Optical Flow and Pixels ที่มีการใช้ข้อมูลการเคลื่อนไหว(optical flow) ในการสร้างโมเดลด้วย
\\
		
		\item Transfer learning video-level presentation from YouTube-8M to ActivityNet dataset
\begin{table}[!ht]
\centering
\begin{tabular}{|c|c|c|c|}
		\hline
		{Approach}&{mAP}&{Hit@1}&{(Hit@1)}\\
		\hline
		Mixture-of-2-Expert ($\mu$)					& 69.1		& 68.7		& 85.4					\\
		+Pretrained PCA on YT-8M						& 74.1		& 72.5		& 89.3					\\
		Mixture-of-2-Expert ([$\mu ; \sigma ; Top_{5}$		& NO			& 74.2		& 72.3					\\
		+Pretrained PCA on YT-8M						& 77.6		& 74.9		& 91.6					\\
		LSTM									& 57.9		& 63.4		& 81.0					\\
		+Pretrained on YT-8M							& 75.6		& 74.2		& 92.4					\\
		\hline
		Ma, Bargal et al.								& 53.8		& -			& -						\\
		Heilbron et al.								& 43.0		& -			& -						\\
		\hline
	\end{tabular}
	\caption{ประสิทธิภาพของโมเดลเมื่อถูก transfer learning ด้วยชุดข้อมูล ActivityNet โดยใช้ video-level presentation}
	\label{tab: transfer learning ActivityNet}
\end{table}
\\
		จากตารางที่ \ref{tab: transfer learning ActivityNet} จะเห็นว่าโมเดลที่ถูก pretrained จาก YouTube-8M นั้นมีประสิทธิภาพที่ดีขึ้นมากเมื่อเทียบกับ benchmark ก่อนหน้า
	\end{enumerate}
\end{enumerate}

\subsubsection*{3. ปัญหาที่พบ}
เนื่องจากว่า YouTube-8M นั้นมีจำนวนข้อมูลที่เยอะมาก ทำให้ไม่สามารถตรวจสอบได้ทั้งหมดว่า ground-truth ของแต่ละวิดีโอนั้นมีความถูกต้องมากน้อยขนาดไหน ทำให้อาจเกิดข้อผิดพลาดได้ (ปัจจุบัน ปี 2019 YouTube-8M ได้มีการตรวจสอบข้อมูลอีกครั้ง เพื่อเพิ่มประสิทธิภาพของชุดข้อมูลซึ่งทำให้ปัจจุบันจำนวนข้อมูล และจำนวน category นั้นจะลดน้อยลงจากข้อมูลที่ใช้อ้างอิงในบทความ \footnote{YouTube-8M,https://arxiv.org/pdf/1609.08675.pdf} ข้างต้นที่ได้กล่าวมา)







%\clearpage
%AVA\textsuperscript{\cite{AVA}} คือ ชุดข้อมูลที่รวบรวมวิดีโอที่มีความยาว 15 นาที ถูกแบ่งด้วยความถี่ 1 hz (900 keyframes) จากในภาพยนต์โดยยึดการกระทำของมนุษย์เป็นศูนย์กลาง
เพื่อใช้สำหรับสร้างโมเดลที่เข้าใจกิจกรรมของมนุษย์ในวิดีโอว่ามนุษย์กำลังทำอะไรอยู่ ซึ่งข้อดีของ AVA คือ ชุดข้อมูลจะมีคำกำกับเป็นแบบทวิคำกำกับ (multiple label)
และคำกำกับของ AVA มีจำนวน 80 ประเภท สามารถแบ่งได้เป็นสามหมวดหมู่คือ ท่าทาง (Pose), ปฏิสัมพันธ์กับวัตถุ (Interaction with object) 
และปฏิสัมพันธ์กับบุคคล (Interaction with people) ซึ่งสามารถมีคำกำกับได้มากสูงสุดถึง 7 คำกำกับ
\begin{enumerate}
	\item {รายละเอียดชุดข้อมูล}
	\begin{enumerate}
		\item ขั้นตอนการเก็บข้อมูลสำหรับการทำชุดข้อมูลมีขั้นตอนการทำ 5 ขั้นดังนี้
		\begin{enumerate}
			\item การสร้างคำศัพท์การกระทำจะมีหลักการ 3 ข้อในการรวบรวมคำศัพท์ดังนี้
			\begin{enumerate}
				\item เก็บรวบรวมคำศัพท์ทั่วไปที่เกิดขึ้นในชีวิตประจำวัน
				\item จะต้องมีเอกลักษณ์สามารถเห็นได้ชัดเจน เช่น การถือของ
				\item กำหนดรูปแบบของคำศัพท์ขึ้นมา และใช้ความรู้จากชุดข้อมูลอื่นในการทำให้ได้หมวดหมู่การกระทำของมนุษย์ที่ครอบคลุม
			\end{enumerate}
			\item ภาพยนต์และส่วนที่เลือกมาใช้ทำชุดข้อมูล AVA ทั้งหมดจะถูกนำมาจาก YouTube โดยเริ่มจากการรวบรวมเอารายชื่อของนักแสดงที่มีชื่อเสียง
			ซึ่งจะมีความหลากหลายของเชื้อชาติรวมกันอยู่ วิดีโอที่ถูกคัดเลือกจะมีเกณฑ์ดังนี้
			\begin{enumerate}
				\item วิดีโอต้องอยู่ในหมวด ภาพยนต์ และละครโทรทัศน์
				\item วิดีโอจะต้องมีความยาวมากกว่า 30 นาที
				\item เผยแพร่มาแล้วเป็นระยะเวลาอย่างน้อย 1 ปี
				\item มีจำนวนยอดคนดูมากกว่า 1,000 ครั้ง
				\item ละเว้นวิดีโอบางประเภท เป็นภาพขาว-ดำ มีความละเอียดต่ำ การ์ตูน หรือวิดีโอเกม
			\end{enumerate}
			\item การสร้างกรอบสี่เหลี่ยมครอบมนุษย์ที่อยู่ภายในภาพประกอบด้วย 2 ขั้นตอน
			\begin{enumerate}
				\item สร้างกรอบสี่เหลี่ยมโดยใช้โมเดลปัญญาประดิษฐ์ faster RCNN สำหรับการตรวจจับมนุษย์
				\item ใช้มนุษย์ในการตรวจสอบและแก้ไขกรอบสี่เหลี่ยมที่ผิดพลาด
			\end{enumerate}	
			\item การติดตามตำแหน่งของบุคคล\\
			ทำการติดตามตำแหน่งของบุคคลที่อยู่ในช่วงเวลาเดียวกันด้วยใช้วิธีการแทร็กโดยยึดมนุษย์เป็นศูนย์กลาง โดยการคำนวณค่าความใกล้เคียงกันระหว่างบุคคล 
			โดยใช้ person embedding (ใช้โครงข่ายประสาทเทียมในการหาฟีเจอร์ขั้นสูงและใช้เมทริกซ์ในการหาความสัมพันธ์ของแต่ละคน) จากนั้นจะใช้อัลกอริทึม Hungarian distance (อัลกอริทึ่มสำหรับการหาข้อเสนอที่ดีที่สุด) ในการหาตัวเลือกคู่ของกรอบสี่เหลี่ยมที่ดีที่สุด
			\item การสร้างคำกำกับคุณลักษณะ\\
			การสร้างคำกำกับของการกระทำจะถูกสร้างขึ้นโดยมนุษย์ ซึ่งผู้วิจัยจะใช้โปรแกรมสำหรับช่วยเหลือในการสร้างคำกำกับคุณลักษณะ โดยสามารถกำหนดคำกำกับของการกระทำได้สูงสุดถึง 7 คำต่อ 1 กรอบสี่เหลี่ยม นอกจากนั้นสามารถตั้งสถานะเนื้อหาที่ไม่เหมาะสม หรือ กรอบสี่เหลี่ยมที่ผิดพลาดได้อีกด้วย ซึ่งในทางปฎิบัติเพื่อลดโอกาสที่จะเกิดข้อผิดพลาด จึงแบ่งขั้นตอนในการสร้างคำกำกับออกเป็น 2 ขั้นตอนดังนี้
			\begin{enumerate}
				\setlength\itemsep{-0.25em}
				\item สร้างข้อเสนอสำหรับคำกำกับของการกระทำ
				\item ข้อเสนอจะถูกตรวจสอบข้อเสนอที่ได้จากขั้นตอนแรก ซึ่งจะใช้มนุษย์ในการตรวจสอบ 3 คน โดยคำกำกับจะต้องถูกตรวจสอบด้วยผู้ตรวจสอบอย่างน้อย 2 คน จึงจะถูกยึดเป็นคำกำกับหลัก
			\end{enumerate}
		\end{enumerate}
	\end{enumerate}
	\item {โมเดลปัญญาประดิษฐ์}
	\begin{enumerate}
		\item โมเดลปัญญาประดิษฐ์ที่งานวิจัยนี้ใช้ คือ two stream variant ซึ่งจะทำการประมวลผลทั้ง RGB flow และ optical flow 
		โดยเป็นโครงสร้างของ faster RCNN ที่นำ Inception network เข้ามาใช้
		\item เครื่องมือที่ใช้วัดผลสำหรับงานวิจัยนี้ คือค่า IoU และ 3D IoUs 
		\begin{enumerate}
			\item ค่า IoU คือค่าที่ใช้วัดความสอดคล้องระหว่างสองกรอบสี่เหลี่ยม(กรอบสี่เหลี่ยมจริงของเฟรม และ กรอบสี่เหลี่ยมที่ทำนายขึ้นมา) ซึ่งใช้สำหรับการวัดผลระดับเฟรม 
			\item ค่า 3D IoU คือค่าที่ใช้วัดความสอดคล้องระหว่างกรอบสี่เหลี่ยมภายในสองวิดีโอใช้สำหรับการวัดผลระดับวิดีโอ โดยเทียบกันระหว่างกรอบสี่เหลี่ยมจริงในช่วงเฟรมที่ต่อกัน (ground-truth tubes) 
			และกรอบสี่เหลี่ยมที่ทำนายขึ้นมาในช่วงของเฟรมที่ต่อกัน (linked detection tubes) 
		\end{enumerate}
		\item ประสิทธิภาพของโมเดลปัญญาประดิษฐ์ในปัจจุบัน
		\\ข้อมูลโมเดลปัญญาประดิษฐ์ที่นำมาทดสอบ
		\begin{enumerate}				
			\item Actionness\textsuperscript{\cite{actioness}} เป็นการหาความน่าจะเป็นของการกระทำ โดยใช้โครงสร้างของ hybrid fully convolutional network (HFCN) hybrid fully เป็นโครงสร้างที่ประกอบด้วยโครงข่ายประสาทเทียม 2 ชนิด คือ
			\begin{enumerate}
				\item Appearance-FCN (A-FCN) คือ โครงข่ายประสาทเทียมที่นำมาใช้แสดงลักษณะของวัตถุ (ตำแหน่งวัตถุ, ความตื้นลึกวัตถุ) ที่ปรากฎบนภาพ RGB1
				\item MotionFCN (M-FCN) คือ โครงข่ายประสาทเทียมที่แยกการเคลื่อนไหวจากข้อมูลของ optical flow
			 \end{enumerate}
			\item Peng without MR, Peng with MR (Multi-region two-stream R-CNN)\textsuperscript{\cite{peng}} เป็นโมเดลปัญญาประดิษฐ์ที่ใช้สำหรับตรวจจับวิดีโอในชีวิตจริง ซึ่งพื้นฐานของโมเดลนี้เป็น Faster R-CNN โดยโมเดลนี้มีกระบวนการ 3 กระบวนการคือ
			\begin{enumerate}
					\item สร้างข้อเสนอพื้นที่ที่มีการเคลื่อนไหว
					\item สะสม Optical flow จากเฟรมหลายๆเฟรม เพื่อนำไปปรับปรุงการตรวจจับการกระทำ
					\item นำพื้นที่หลายๆส่วนมาวิเคราะห์ผ่านโมเดล Faster R-CNN
			\end{enumerate}
			\item ACT Action Tubelet Detector\textsuperscript{\cite{act}} เป็นการระบุตำแหน่งของการกระทำที่มีระยะเวลาๆสั้นๆ ซึ่งใช้วิธีการตรวจจับระดับเฟรม และ ใช้การติดตามตำแหน่งในการเชื่อมระหว่างเฟรมปัจจุบันไปยังเฟรมถัดไป. ACT ถูกสร้างต่อจาก SSD framework และ ใช้คอนโวลูชันในการสกัดคุณลักษณะในแต่ละเฟรมซึ่งการคิดคะแนนและความน่าจะเป็นของหมวดหมู่จะคิดจากการนำคุณลักษณะเรียงต่อกัน และ หาข้อมูลจากลำดับข้อมูลนั้น
		\end{enumerate}
		จากการทดสอบการเทียบโมเดลปัญญาประดิษฐ์ของงานวิจัยนี้และวิธีการอื่นๆ โดยนำไปทดสอบกับชุดข้อมูลวิดีโอ JHMDB และ UCF101-24 ได้ผลลัพธ์ออกมาดังนี้
			\begin{table}[!ht]
				\centering
				\begin{tabular}{|c|c|c|c|}
					\hline
					{Frame-mAP}&{JHMDB (mAP)}&{UCF101-24 (mAP)}								\\
					\hline
					Actionness 			& 39.9				& 	-						\\
					Peng w/o MR			& 56.9				& 64.8						\\
					Peng w/  MR 			& 58.5				& 65.7						\\
					ACT					& 65.7				& 69.5						\\
					\hline
					2 stream(Our approach)		& \textbf{73.3}		& \textbf{76.3}				\\
					\hline
				\end{tabular}
				\caption{ผลการทดลองของวิธีต่างๆบนคุณลักษณะระดับเฟรม}
				\label{tab: transfer learning}
			\end{table}
		\item ปัญหาที่พบ\\
		ในปัจจุบันยังไม่มีโมเดลปัญญาประดิษฐ์ที่ทดสอบด้วยชุดข้อมูล AVA และได้ผลการทำงานที่ดี เนื่องจากชุดข้อมูลนี้สนใจการกระทำของมนุษย์ที่มีรายละเอียดเล็กๆน้อยๆ 
		ทำให้ยากต่อการทำนายสำหรับโมเดลปัญญาประดิษฐ์
	\end{enumerate}
\end{enumerate}
%\clearpage
%\subsubsection*{Moments in time}
Moments in time \footnote{Moment,http://moments.csail.mit.edu/TPAMI.2019.2901464.pdf} คือชุดข้อมูลที่ใช้มนุษย์ในการ label ทั้งหมดให้กับวิดีโอสั้นถึง 1 ล้านวิดีโอ และมีจำนวน activity หรือกระทำต่างกัน 339 class โดยแต่ละวิดีโอจะมีความยาวอยู่ที่ 3 วินาที เนื่องจากเป็นเวลาเฉลี่ยที่มนุษย์ใช้ในการเข้าใจกับเหตุการณ์ที่เกิดขึ้น (human working memory) รูปแบบของชุดข้อมูลจะมีอยู่ทั้งหมดอยู่ 3 รูปแบบ ได้แก่ ภาพนิ่ง (spatial) เสียง (auditory) และการเคลื่อนไหว (temporal) นอกจากนี้ชุดข้อมูลนี้นั้นไม่รวบรวมเพียงแค่การกระทำของมนุษย์เท่านั้น ยังรวมไปถึง สัตว์ สิ่งของ และ ปรากฏการณ์ธรรมชาติ ทำให้ ชุดข้อมูลนี่เป็นการท้าทายรูปแบบใหม่เพราะด้วยข้อมูลที่มีความซับซ้อนมากขึ้น เช่น การสร้างโมเดลที่สามารถบอกถึงการกระทำ (action) ได้ถึงแม้ว่าสิ่งที่เราสนใจ (มนุษย์ สัตว์ สิ่งของ หรือปรากฏการณ์ธรรมชาติ) จะแตกต่างกัน เป็นต้น

\begin{figure}[!ht]
	\centering
	\includegraphics[width=1\textwidth]{chapter2/images/Example_of_class.png}
		\caption{ตัวอย่างของวิดีโอ class เดียวกันไม่จำเป็นต้องเป็น agents เดียวกัน}
    	\label{fig:moment_class}
\end{figure}

เป้าหมายของชุดข้อมูล Moments in time คือการออกแบบชุดข้อมูลให้มีความหลากหลาย ครอบคลุม ความสมดุล และจำนวนข้อมูลที่สูง โดยที่แต่ละ activity หรือการกระทำนั้นจะประกอบไปด้วยวิดีโอมากกว่า 1,000 วิดีโอ และมีการออกแบบมาเพื่อให้สามารถพัฒนาต่อได้ เช่น จำนวน class และข้อมูลภายใน class นั้น ๆ

\clearpage
\subsubsection*{1. วิธีการรวบรวมข้อมูล}
เริ่มจากการรวบรวมคำ (verb) ที่มีการใช้อยู่ทั่วไปในชีวิตประจำวันมา 4,500 คำจาก VerbNet จากนั้นนำมาแบ่งกลุ่มคำ(verb) ที่มีความหมายใกล้เคียงกันโดยใช้ features จาก Propbank และ FrameNet โดยเก็บข้อมูลเป็นแบบ binary feature vector ซึ่งถ้าคำ (verb) ไหนมีความเกี่ยวข้องกับ feature ก็จะให้ค่าเป็น 1 ถ้าไม่เกี่ยวข้องกันจะให้ค่าเป็น 0 จากนั้นจึงใช้วิธี k-means clustering ในการแบ่งกลุ่ม เมื่อแบ่งกลุ่มแล้วจากนั้นจะเลือกคำ (verb) จากในแต่ละกลุ่มนั้น โดยคำ (verb) ที่เลือกมานั้นจะเป็นที่ใช้บ่อยที่สุดในกลุ่มนั้น และลบคำ (verb) นั้นออกจากกลุ่มทั้งหมด (คำ ๆ หนึ่งสามารถอยู่ได้หลายกลุ่ม) จากนั้นจะทำกระบวนการนี่ไปเรื่อย ๆ แต่คำ (verb) ที่เลือกมาจะต้องไม่มีความหมายคลุมเครือ ไม่สามารถมองเห็นหรือได้ยินได้ และต้องไม่มีความหมายเหมือนกับคำ (verb) ที่เคยเลือกมาก่อน จนสุดท้ายแล้วได้ออกมาที่ 339 class
\par
ต่อมาทำการหาชุดข้อมูลวิดีโอโดยจะตัดออกมาเพียง 3 วินาทีที่เกี่ยวข้องกับคำ (verb) ใน 339 class ที่เลือกมา จากวิดีโอ แหล่งต่างกัน 10 แหล่ง การตัดวิดีโอนั้นจะไม่ใช้พวก Video2Gif (โมเดลที่ระบุตำแหน่งของสิ่งที่น่าสนใจในวิดีโอ) เพราะจะทำให้เกิด bias ขึ้นจะเกิดขึ้นตอนสร้างโมเดลจากนั้นจะทำการส่งข้อมูลของคำ (verb) และวิดีโอที่ตัดไปยัง Amazon Mechanical Turk (AMT หรือตลาดแรงงาน) เพื่อทำการ label โดยพนักงานแต่ละคนของ AMT จะได้ 64 วิดีโอซึ่งเกี่ยวข้องกับคำ (verb) หนึ่ง และอีก 10 วิดีโอที่มีการทำ label อยู่แล้ว โดยวิดีโอที่มีการทำ label ถ้ามีพนักงานของ AMT ตอบเหมือนกันกับที่ทำ label ไว้เกิน 90\% ถึงจะนำเข้าไปรวมกับชุดข้อมูลส่วนอีก 64 วิดีโอถ้าเป็นของ training set จะต้องผ่านพนักงานของ AMT อย่างน้อย 3 ครั้ง และต้อง label เหมือนกัน 75\% ขึ้นไปถึงจะถือว่าเป็น label ที่ถูกต้อง ถ้าเป็นของ validation และ test set จะต้องผ่านพนักงานของ AMT อย่างน้อย 4 ครั้ง และต้อง label เหมือนกัน 85\% ขึ้นไป ที่ไม่ตั่งเกณฑ์ไว้ที่ 100\% เพราะจะทำให้วิดีโอนั้นยากเกินไปที่จะทำให้สามารถจำการกระทำได้

\begin{figure}[!ht]
	\centering
	\includegraphics[width=0.5\textwidth]{chapter2/images/UI.png}
		\caption{User interface ของโปรแกรมทำ label}
    	\label{fig:User interface}
\end{figure}
\clearpage
\subsubsection*{2. ข้อมูลของ Moments in time}
มีวิดีโอมากกว่า 1 ล้านวิดีโอ และมี class ถึง 339 class ที่แตกต่างกัน มีค่าเฉลี่ยวิดีโอของแต่ละ class อยู่ที่ 1,757 และค่า median อยู่ที่ 2,775

\begin{figure}[!ht]
	\centering
	\includegraphics[width=1\textwidth]{chapter2/images/statistic_moment.png}
		\caption{สถิติของชุดข้อมูลของ Moments in timel}
    	\label{fig:statistic_moment}
\end{figure}
\subsubsection*{3. วิธีการทดสอบชุดข้อมูลและผลลัพธ์ที่ได้}
โดยการทดสอบแรกจะเป็นการทดสอบเทียบกับชุดข้อมูลอื่นดังภาพด้านล่าง

\begin{figure}[!ht]
	\centering
	\includegraphics[width=1\textwidth]{chapter2/images/compare_dataset.png}
		\caption{เปรียบเทียบขอมูลระหว่าง Dataset}
    	\label{fig:compare_dataset}
\end{figure}

จากภาพจะเห็นได้ว่า Moments in time นั้นมีฉากหรือสถานที่ที่เหมือน Places = 100\% และมีวัตถุเหมือนกับ ImageNet ถึง 99.9 \%. ส่วนชุดข้อมูลที่มีความได้เคียงกับ Moments in time มากที่สุดคือชุดข้อมูล Kinetics ที่มีฉากหรือสถานที่ที่เหมือน Places = 99.5\% และมีวัตถุเหมือน ImageNet ถึง 96.6\%
\par
การทดสอบต่อมาจะเป็นการนำ Moments in time มาทดสอบสร้างโมเดลด้วยวิธีต่าง ๆ โดยจะเริ่มจากการเตรียมข้อมูลข้อมูลดังนี้
\begin{enumerate}
	\setlength\itemsep{-0.25em}
	\item training set จะมี 802,264 วิดีโอ และมีวิดีโอในแต่ละ class อยู่ที่ 500 ถึง 5,000 วิดีโอ
	\item validation set จะมี 33,900 วิดีโอ และมีวิดีโอในแต่ละ class อยู่ที่ 100 วิดีโอ
	\item เริ่มการ preprocess จากแยกภาพRGB ออกมาจากวิดีโอ และทำการเปลี่ยนขนาดของภาพให้เป็น 340x256  pixel
	\item ใช้ TVL1 optical flow algorithm จาก opencv เพื่อลดข้อมูลรบกวนที่จะเกิดขึ้น
	\item ทำการแปลงค่าที่อยู่ใน optical flow ให้เป็นเลขจำนวนเต็ม(integer) เพื่อทำให้การคำนวณนั้นเร็วยิ่งขึ้น
	\item ปรับค่า displacement ใน optical flow ให้ค่าสูงสุดเป็น 15 ต่ำสุดเป็น 0 และทำการปรับขนาดให้เป็นช่วง 0-255
	\item เก็บข้อมูลออกมาในรูปแบบของ grayscale image เพื่อลดพื้นที่ ๆ ใช้เก็บข้อมูล
	\item แก้ปัญหาเรื่องการเคลื่อนไหวของกล้อง(camera motion) โดยการนำค่าเฉลี่ยของ เวกเตอร์(vector) ไปลบกับ displacement
	\item สุดท้ายจะเป็นสุ่มตัดภาพออกมาเพื่อเพิ่มจำนวนข้อมูล
\end{enumerate}
หลังจากการเตรียมข้อมูลเรียบร้อยแล้วจะนำข้อมูลเหล่านั้นมาสร้างโมเดลด้วยวิธีการต่าง ๆ ดังตารางด้านล่าง

\begin{table}[!ht]
\centering
\begin{tabular}{|c|c|c|c|}
		\hline
		{Model}&{Modelity}&{Top-1(\%)}&{Top-5(\%)}\\
		\hline
		Chance			& -				& 0.29		& 1.47						\\
		\hline
		ResNet50-scratch	& Spatial			& 23.65		& 46.76						\\
		ResNet50-Places		& Spatial			& 26.44		& 50.56						\\	
		ResNet50-ImageNet	& Spatial			& 27.16		& 51.68						\\
		TSN-Spatial		& Spatial			& 24.11		& 49.10						\\
		\hline
		BNIncepion-Flow		& Temporal		& 11.60		& 27.40						\\
		TSN-Flow			& Temporal		& 15.71		& 34.65						\\
		\hline
		SoundNet			& Auditory			& 7.60		& 18.00						\\
		\hline
		TSN-2stream		& Spatial+Temporal	& 25.32		& 50.10						\\
		TRN-Multiscale		& Spatial+Temporal	& 28.27		& 53.87						\\
		I3D 				& Spatial+Temporal	& 29.51		& 56.06						\\
		\hline
		Ensemble(SVM)		& S+T+A 			& 31.16		& 57.67						\\
		\hline
	\end{tabular}
	\caption{Classification accuracy ของ TOP-1 และ TOP-5}
	\label{tab: Classification accuracy ของ TOP-1 และ TOP-5}
\end{table}

จากภาพจะเห็นได้ว่าผลลัพท์ที่ดีสุดคือการทำ ensemble(SVM) ซึ่งเป็นรวมของโมเดล ReNet50-ImageNet, I3D และ SoundNet จากผลลัพท์จะเห็นค่าที่ได้ออกมาจาก ensemble(SVM)  มีค่าใกล้เคียงกับรูปแบบ spatial เพราะประสิทธิของภาพเคลื่อนไหว(temporal) และ เสียง(auditory) นั้นมีประสิทธิภาพต่ำ ซึ่งจุดนี่จะทำให้เห็นว่าตัว Moments in time ยังทำให้สามารถพัฒนาต่อไปได้อีก
\par
ต่อมาจะทำทดสอบ cross dataset transfer โดยการนำโมเดล ResNet50 I3D pretrained ลงทั้งบน Kinetics และ Moments in time และนำมาเทียบกับชุดข้อมูลอื่น โดยชุดข้อมูลแต่ละชุดจะมีการปรับ frame rate ของวิดีโอให้เป็น 5 fps เหมือนกัน

\begin{table}[!ht]
	\centering
	\begin{tabular}{|c|c|c|c|}
		\hline
		{Pretrained}&\multicolumn{3}{c|}{Fine-Tuned}\\
		\cline{2-4}
		{}			& UCF		& HMDB		& Something			\\
		\hline
		\multirow{2}{*}{Kinetics}		& Top-1 : 92.6		& Top-1 : 62.0		& Top-1 : 48.6		\\
		{}						& Top-5 : 99.2		& Top-5 : 88.2		& Top-5 : 77.9		\\
		\hline
		\multirow{2}{*}{Moments}		& Top-1 : 91.9		& Top-1 : 65.9		& Top-1 : 50.0		\\
		{}						& Top-5 : 98.6		& Top-5 : 89.3		& Top-5 : 78.8		\\
		\hline
	\end{tabular}
	\caption{Data transfer performance ของโมเดล Resnet50 I3D}
	\label{tab: Data transfer performance ของโมเดล Resnet50 I3D}
\end{table}

จะเห็นได้ว่า Kinetics ให้ผลลัพท์ที่ดีกว่าใน UCF เพราะว่ามีการแชร์ class ด้วยกันอยู่หลายอย่าง ในขณะที่ HMDB นั้นมีการรวบรวม source จากหลายแหล่ง และมีจำนวน class ที่หลากหลายจึงทำให้มีความใกล้เคียงกับตัวข้อมูลของ Moments in time ดังนั้นจึงเทียบผลลัพท์จาก Something ซึ่งจะทำให้เห็นว่า Moments in time มีประสิทธิภาพที่ดีกว่าและวิดีโอที่มีความยาวมากกว่า 3 วินาทีจะไม่ส่งผลกระทบกับประสิทธิภาพของ Moments in time

\subsubsection*{4. ปัญหาที่พบ}
ผลลัพท์จากการทำนายด้วยโมเดลถ้าผ่านรูปภาพที่มีรายละเอียดเยอะจะทำให้การ ทำนายโอกาสผิดนั้นค่อนข้างสูง ซึ่งปัญหานี่สามารถทำให้เกิดน้อยลงด้วยการนำวิธี Class Activation Mapping(CAM) จะเป็นการเน้นรูปภาพในส่วนที่มีข้อมูลมากที่สุดและ ทำนายผลออกมา แต่ก็ยังมีจุดที่เป็นปัญหาอยู่ เช่น การกระที่เกิดขึ้นเร็วมาก (การลื่นล้ม) จะทำให้การทำนาย นั้นมีโอกาสผิดสูงขึ้น 

\begin{figure}[!ht]
	\centering
	\includegraphics[width=1\textwidth]{chapter2/images/CAM.png}
		\caption{ภาพที่ได้จากการทำ CAM และผลลัพท์ที่ได้จากการทำนายด้วยโมเดล resnet50-ImageNet}
    	\label{fig:CAM}
\end{figure}




\section{ทฤษฎีที่เกี่ยวข้อง}
\subsection{Optical flow}
Optical flow \footnote{Optical flow,shorturl.at/mrtEZ}  คือรูปแบบของการเคลื่อนที่ของวัตถุในรูปภาพระหว่างภาพซึ่งอาจจะการจากเคลื่อน ที่ของวัตถุหรือตัวกล้อง ออกมาในรูปแบบของ เวกเตอร์(vector) 2 มิติ โดยที่เวกเตอร์แต่ละตัวจะแสดงถึงทิศทางการเคลื่อนที่ระหว่างภาพดังรูปด้านล่าง

\begin{figure}[!ht]
	\centering
	\includegraphics[width=1\textwidth]{chapter2/images/vector_optical.png}
		\caption{ตัวอย่างการเคลื่อนที่ของลูกบอล}
    	\label{fig:vector_optical}
\end{figure}

จากรูปภาพจะแสดงให้เห็นถึงการเคลื่อนที่ของลูกบอลของภาพที่ต่อเนื่องกัน 5 ภาพโดยที่ลูกศรแสดงถึงทิศทางการเคลื่อนที่ของเวกเตอร์
\\
\par
การทำงานของ optical flow อยู่บนสมมติฐานหลายประการได้แก่
\begin{enumerate}
	\setlength\itemsep{-0.25em}
	\item ความเข้มของพิกเซล(pixel) ของวัตถุจะไม่เปลี่ยนแปลงระหว่างภาพที่ต่อเนื่องกัน
	\item พิกเซลที่อยู่ใกล้กันจะมีการเคลื่อนไหวที่คล้ายกัน
\end{enumerate}

เมื่อพิจารณาพิกเซล I(x,y,t) จากภาพแรกจะเคลื่อนไหวเป็นระยะทาง (dx,dy) ไปยังภาพต่อไปหลังจากผ่านไปแล้ว dt เวลา ดังนั้นเนื่องจาก พิกเซล เหล่านี้เหมือนกันและความเข้มไม่มีการเปลี่ยนแปลง จึงทำให้พูดได้ว่า
\\
\begin{equation}
I(x,y,t) = I(x + dx, y + dy, t + dt)
\end{equation}
โดยที่
\begin{conditions}
I 		&	พิกเซลจากภายในภาพ				\\
x 		&	ตำแหน่งของพิกเซล ในแกน x 		\\
dx		&	ระยะทางที่เคลื่อนที่ในแกน x 			\\
y		&	ตำแหน่งของพิกเซล ในแกน y 		\\
dy		&	ระยะทางที่เคลื่อนที่ในแกน y 			\\
t 		&	เวลา							\\
dt		&	ระยะเวลาที่เปลี่ยนไประหว่างภาพ
\end{conditions}

จากนั้นใช้การประมาณค่าของ taylor series ทางฝั่งขวามือและ ลบค่า common term และหารด้วย dt เพื่อให้ได้สมการดังต่อไปนี้
\begin{equation}
f_{x}u + f_{y}v + f_{t}
\end{equation}
\begin{equation}
f_{x} = \frac{\delta f}{\delta x} ; f_{y} = \frac{\delta f}{\delta y}
\end{equation}
\begin{equation}
u = \frac{\delta x}{\delta t} ; v = \frac{\delta y}{\delta t}
\end{equation}
โดยที่
\begin{conditions}
f_{x}		&	เกรเดียน(gradient) ในแกน x 		\\
f_{y}		&	เกรเดียนในแกน y				\\
f_{t}		&	เกรเดียนของเวลา				\\
u 		&	เวกเตอร์การเคลื่อนที่ของแกน x 	\\
v		&	เวกเตอร์การเคลื่อนที่ของแกน y	\\
\end{conditions}
สมการข้างบนนี้จะเรียกว่าสมการ optical flow จากสมการทำให้สามารถหา $f_{x}$ และ $f_{y}$ โดยเป็น เกรเดียนของภาพ และ  $f_{t}$ เป็นเกรเดียน(gradient)ของเวลา แต่ $u$ กับ $v$ เป็นตัวแปรที่ไม่ทราบ ทำให้สมการนี้ไม่สามารถแก้ไขโดยมีตัวแปรที่ไม่ทราบถึง 2 ตัว จึงมีการนำวิธีการต่าง ๆ เข้ามาใช้ในการแก้ปัญหานี้ โดยวิธีการที่นำเข้ามาใช้ในการแก้ปัญหาคือ dense optical flow ซึ่งใช้อัลกอริทึมของ Gunner Farneback ซึ่งจะใช้วิธีการขยายพหุนาม\footnote{polynomial expansionfile:http://www.diva-portal.org/smash/get/diva2:273847/FULLTEXT01.pdf} (polynomial expansion)


\clearpage
\subsection{Intersection Over Union (IoU)}

\begin{figure}[!ht]
	\centering
	\includegraphics[scale=0.5]{chapter2/images/iou_equation.png}
		\caption{ตัวอย่างการเคลื่อนที่ของลูกบอล}
    	\label{fig:iou_equation}
\end{figure}

เป็นหนึ่งวิธีในการประเมินผลการทดลองสำหรับการตรวจจับวัตถุ โดยหลักการของการคำนวณ IoU สำหรับการประเมินผลการตรวจจับวัตถุ คือ การนำกรอบสี่เหลี่ยมจริงของเฟรม และ กรอบสี่เหลี่ยมที่ทำนายขึ้นมา มาหาอัตราส่วนระหว่าง พื้นที่ที่กรอบสี่เหลี่ยมทั้งสองทับซ้อนกัน และ พื้นที่ทั้งหมดของกรอบสี่เหลี่ยมทั้งสองรวมกัน ผลลัพธ์จะได้เป็นค่า IoU ซึ่งจะมีสมการดังนี้
\begin{equation}
IoU(P,G) = \frac{\left| P \cap G \right|}{\left| P \cup  G \right|}					
\end{equation}
โดยที่
\begin{conditions}
IoU			&  ค่าที่ใช้สำหรับวัดผลความใกล้เคียงระหว่างสองกรอบสี่เหลี่ยม    \\
P			&  พื้นที่ของกรอบสี่เหลี่ยมที่ทำนายได้	\\
G			&  พื้นที่ของกรอบสี่เหลี่ยมจริงของรูปภาพ					\\
\end{conditions}





