AVA\textsuperscript{\cite{ava}} คือ ชุดข้อมูลที่รวบรวมวิดิโอที่มีความยาว 15 นาที ถูกแบ่งด้วยความถี่ 1 hz (900 keyframes) จากในภาพยนต์โดยยึดการกระทำของมนุษย์เป็นศูนย์กลาง
เพื่อใช้สำหรับสร้างโมเดลที่เข้าใจกิจกรรมของมนุษย์ในวิดิโอว่ามนุษย์กำลังทำอะไรอยู่ ซึ่งข้อดีของ AVA คือ ชุดข้อมูลจะมีคำกำกับเป็นแบบทวิคำกำกับ (multiple label)
และคำกำกับของ AVA มีจำนวน 80 ประเภท สามารถแบ่งได้เป็นสามหมวดหมู่คือ ท่าทาง (Pose), ปฏิสัมพันธ์กับวัตถุ (Interaction with object) 
และปฏิสัมพันธ์กับบุคคล (Interaction with people) ซึ่งสามารถมีคำกำกับได้มากสูงสุดถึง 7 คำกำกับ
\begin{enumerate}
	\item {รายละเอียดชุดข้อมูล}
	\begin{enumerate}
		\item ขั้นตอนการเก็บข้อมูลสำหรับการทำชุดข้อมูลมีขั้นตอนการทำ 5 ขั้นดังนี้
		\begin{enumerate}
			\item การสร้างคำศัพท์การกระทำจะมีหลักการ 3 ข้อในการรวบรวมคำศัพท์ดังนี้
			\begin{enumerate}
				\item เก็บรวบรวมคำศัพท์ทั่วไปที่เกิดขึ้นในชีวิตประจำวัน
				\item จะต้องมีเอกลักษณ์สามารถเห็นได้ชัดเจน เช่น การถือของ
				\item กำหนดรูปแบบของคำศัพท์ขึ้นมา และใช้ความรู้จากชุดข้อมูลอื่นในการทำให้ได้หมวดหมู่การกระทำของมนุษย์ที่ครอบคลุม
			\end{enumerate}
			\item ภาพยนต์และส่วนที่เลือกมาใช้ทำชุดข้อมูล AVA ทั้งหมดจะถูกนำมาจาก YouTube โดยเริ่มจากการรวบรวมเอารายชื่อของนักแสดงที่มีชื่อเสียง
			ซึ่งจะมีความหลากหลายของเชื้อชาติรวมกันอยู่ วิดีโอที่ถูกคัดเลือกจะมีเกณฑ์ดังนี้
			\begin{enumerate}
				\item วิดีโอต้องอยู่ในหมวด ภาพยนต์ และละครโทรทัศน์
				\item วิดีโอจะต้องมีความยาวมากกว่า 30 นาที
				\item เผยแพร่มาแล้วเป็นระยะเวลาอย่างน้อย 1 ปี
				\item มีจำนวนยอดคนดูมากกว่า 1,000 ครั้ง
				\item ละเว้นวิดีโอบางประเภท เป็นภาพขาว-ดำ มีความละเอียดต่ำ การ์ตูน หรือวิดีโอเกม
			\end{enumerate}
			\item การสร้างกรอบสี่เหลี่ยมครอบมนุษย์ที่อยู่ภายในภาพประกอบด้วย 2 ขั้นตอน
			\begin{enumerate}
				\item สร้างกรอบสี่เหลี่ยมโดยใช้โมเดลปัญญาประดิษฐ์ faster RCNN สำหรับการตรวจจับมนุษย์
				\item ใช้มนุษย์ในการตรวจสอบและแก้ไขกรอบสี่เหลี่ยมที่ผิดพลาด
			\end{enumerate}	
			\item การติดตามตำแหน่งของบุคคล\\
			ทำการติดตามตำแหน่งของบุคคลที่อยู่ในช่วงเวลาเดียวกันด้วยใช้วิธีการแทร็กโดยยึดมนุษย์เป็นศูนย์กลาง โดยการคำนวณค่าความใกล้เคียงกันระหว่างบุคคล 
			โดยใช้ person embedding (ใช้โครงข่ายประสาทเทียมในการหาฟีเจอร์ขั้นสูงและใช้เมทริกซ์ในการหาความสัมพันธ์ของแต่ละคน) จากนั้นจะใช้อัลกอริทึม Hungarian distance (อัลกอริทึ่มสำหรับการหาข้อเสนอที่ดีที่สุด) ในการหาตัวเลือกคู่ของกรอบสี่เหลี่ยมที่ดีที่สุด
			\item การสร้างคำกำกับคุณลักษณะ\\
			การสร้างคำกำกับของการกระทำจะถูกสร้างขึ้นโดยมนุษย์ ซึ่งผู้วิจัยจะใช้โปรแกรมสำหรับช่วยเหลือในการสร้างคำกำกับคุณลักษณะ โดยสามารถกำหนดคำกำกับของการกระทำได้สูงสุดถึง 7 คำต่อ 1 กรอบสี่เหลี่ยม นอกจากนั้นสามารถตั้งสถานะเนื้อหาที่ไม่เหมาะสม หรือ กรอบสี่เหลี่ยมที่ผิดพลาดได้อีกด้วย ซึ่งในทางปฎิบัติเพื่อลดโอกาสที่จะเกิดข้อผิดพลาด จึงแบ่งขั้นตอนในการสร้างคำกำกับออกเป็น 2 ขั้นตอนดังนี้
			\begin{enumerate}
				\setlength\itemsep{-0.25em}
				\item สร้างข้อเสนอสำหรับคำกำกับของการกระทำ
				\item ข้อเสนอจะถูกตรวจสอบข้อเสนอที่ได้จากขั้นตอนแรก ซึ่งจะใช้มนุษย์ในการตรวจสอบ 3 คน โดยคำกำกับจะต้องถูกตรวจสอบด้วยผู้ตรวจสอบอย่างน้อย 2 คน จึงจะถูกยึดเป็นคำกำกับหลัก
			\end{enumerate}
		\end{enumerate}
	\end{enumerate}
	\item {โมเดลปัญญาประดิษฐ์}
	\begin{enumerate}
		\item โมเดลปัญญาประดิษฐ์ที่งานวิจัยนี้ใช้ คือ two stream variant ซึ่งจะทำการประมวลผลทั้ง RGB flow และ optical flow 
		โดยเป็นโครงสร้างของ faster RCNN ที่นำ Inception network เข้ามาใช้
		\item เครื่องมือที่ใช้วัดผลสำหรับงานวิจัยนี้ คือค่า IoU และ 3D IoUs 
		\begin{enumerate}
			\item ค่า IoU คือค่าที่ใช้วัดความสอดคล้องระหว่างสองกรอบสี่เหลี่ยม(กรอบสี่เหลี่ยมจริงของเฟรม และ กรอบสี่เหลี่ยมที่ทำนายขึ้นมา) ซึ่งใช้สำหรับการวัดผลระดับเฟรม 
			\item ค่า 3D IoUs คือค่าที่ใช้วัดความสอดคล้องระหว่างกรอบสี่เหลี่ยมภายใน 2 วิดีโอ ซึ่งใช้สำหรับการวัดผลระดับวิดีโอ โดยเทียบกันระหว่างกรอบสี่เหลี่ยมจริงในช่วงของเฟรมที่ต่อกัน (ground-truth tubes) และ กรอบสี่เหลี่ยมที่ทำนายขึ้นมาในช่วงของเฟรมที่ต่อกัน (linked detection tubes) 
		\end{enumerate}
		\item ประสิทธิภาพของโมเดลปัญญาประดิษฐ์ในปัจจุบัน
		\\ข้อมูลโมเดลปัญญาประดิษฐ์ที่นำมาทดสอบ
		\begin{enumerate}				
			\item Actionness\textsuperscript{\cite{actioness}} เป็นการหาความน่าจะเป็นของการกระทำ โดยใช้โครงสร้างของ hybrid fully convolutional network (HFCN) hybrid fully เป็นโครงสร้างที่ประกอบด้วยโครงข่ายประสาทเทียม 2 ชนิด คือ
			\begin{enumerate}
				\item Appearance-FCN (A-FCN) คือ โครงข่ายประสาทเทียมที่นำมาใช้แสดงลักษณะของวัตถุ(ตำแหน่งวัตถุ, ความตื้นลึกวัตถุ) ที่ปรากฎบนรูป RGB1
				\item MotionFCN (M-FCN) คือ โครงข่ายประสาทเทียมที่แยกการเคลื่อนไหวจากข้อมูลของ optical flow
			 \end{enumerate}
			\item Peng without MR, Peng with MR (Multi-region two-stream R-CNN)\textsuperscript{\cite{peng}} เป็นโมเดลปัญญาประดิษฐ์ที่ใช้สำหรับตรวจจับวิดีโอในชีวิตจริง ซึ่งพื้นฐานของโมเดลนี้เป็น Faster R-CNN โดยโมเดลนี้มีกระบวนการ 3 กระบวนการคือ
			\begin{enumerate}
					\item สร้างข้อเสนอพื้นที่ที่มีการเคลื่อนไหว
					\item สะสม Optical flow จากเฟรมหลายๆเฟรม เพื่อนำไปปรับปรุงการตรวจจับการกระทำ
					\item นำพื้นที่หลายๆส่วนมาวิเคราะห์ผ่านโมเดล Faster R-CNN
			\end{enumerate}
			\item ACT Action Tubelet Detector\textsuperscript{\cite{act}} เป็นการระบุตำแหน่งของการกระทำที่มีระยะเวลาๆสั้นๆ ซึ่งใช้วิธีการตรวจจับระดับเฟรม และ ใช้การติดตามตำแหน่งในการเชื่อมระหว่างเฟรมปัจจุบันไปยังเฟรมถัดไป. ACT ถูกสร้างต่อจาก SSD framework และ ใช้คอนโวลูชันในการสกัดคุณลักษณะในแต่ละเฟรมซึ่งการคิดคะแนนและความน่าจะเป็นของหมวดหมู่จะคิดจากการนำคุณลักษณะเรียงต่อกัน และ หาข้อมูลจากลำดับข้อมูลนั้น
		\end{enumerate}
		จากการทดสอบการเทียบโมเดลปัญญาประดิษฐ์ของงานวิจัยนี้และวิธีการอื่นๆ โดยนำไปทดสอบกับชุดข้อมูลวิดีโอ JHMDB และ UCF101-24 ได้ผลลัพธ์ออกมาดังนี้
			\begin{table}[!ht]
				\centering
				\begin{tabular}{|c|c|c|c|}
					\hline
					{Frame-mAP}&{JHMDB (mAP)}&{UCF101-24 (mAP)}								\\
					\hline
					Actionness 			& 39.9				& 	-						\\
					Peng w/o MR			& 56.9				& 64.8						\\
					Peng w/  MR 			& 58.5				& 65.7						\\
					ACT					& 65.7				& 69.5						\\
					\hline
					2 stream(Our approach)		& \textbf{73.3}		& \textbf{76.3}				\\
					\hline
				\end{tabular}
				\caption{ผลการทดลองของวิธีต่างๆบนคุณลักษณะระดับเฟรม}
				\label{tab: transfer learning}
			\end{table}
		\item ปัญหาที่พบ
		ในปัจจุบันยังไม่มีโมเดลปัญญาประดิษฐ์ที่ทดสอบด้วยชุดข้อมูล AVA และได้ผลการทำงานที่ดี เนื่องจากชุดข้อมูลนี้สนใจการกระทำของมนุษย์ที่มีรายละเอียดเล็กๆน้อยๆ 
		ทำให้ยากต่อการทำนายสำหรับโมเดลปัญญาประดิษฐ์
	\end{enumerate}
\end{enumerate}