การระบุตัวตนของบุคคล คือการระบุตัวตนของบุคคลภายในวิดีโอหรือระหว่างรูปภาพ สามารถนำมาประยุกต์ใช้ในด้านของการรักษาความปลอดภัย การตามหาบุคคล หรือการตรวจสอบการกระทำของบุคคลนั้นในวิดีโอได้

การระบุตัวตนของบุคคลนั้นเป็นปัญหาที่ท้าทาย เนื่องจากคุณลักษณะทั่วไปของบุคคลในรูปภาพไม่เพียงพอต่อการระบุบุคคลภายในภาพว่าเป็นบุคคลคนเดียวกันได้ 
ซึ่งวิธีการที่ใช้สำหรับการระบุตัวตนของบุคคล คือวิธีการที่เรียกว่า Dynamically Matching Local Information (DMLI) ที่สามารถจัดแนวลายละเอียดข้อมูลของภาพ และให้ประสิทธิภาพที่สูงออกมา

การระบุตัวตนของบุคคล จะเริ่มจากการแบ่งภาพออกเป็น 8 ส่วนและใช่คุณลักษณะของภาพมาทำ normalize ซึ่งจะช่วยในการลดความซ้ำซ้อนของข้อมูล ต่อมาข้อมูลที่ทำการ normalize แล้วมาใช้เปรียบเทียบความแตกต่างของคุณลักษณะของรูป
หลังจากนั้นหาค่าเฉลี่ยของความแตกต่างออกมา ถ้าค่าที่ออกมาใกล้เคียงกับ 0 จะหมายถึงบุคคลในรูปทั้งสองเป็นบุคคลเดียวกัน