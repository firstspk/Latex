การระบุตัวตนของบุคคล คือการระบุตัวตนของบุคคลภายในวิดีโอหรือระหว่างรูปภาพ สามารถนำมาประยุกต์ใช้ในด้านของการรักษาความปลอดภัย การตามหาบุคคล หรือการตรวจสอบการกระทำของบุคคลนั้นในวิดีโอได้
\par
การทำ การระบุตัวตนของบุคคล นั้นเป็นปัญหาที่ท้าทาย เนื่องจากคุณลักษณะทั่วไปของบุคคลในรูปภาพไม่เพียงพอต่อการระบุบุคคลภายในภาพว่าเป็นบุคคลคนเดียวกันได้ ซึ่งวิธีการที่ใช้สำหรับในการทำ การระบุตัวตนของบุคคล คือวิธีการที่เรียกว่า Dynamically Matching Local Information (DMLI) ที่สามารถจัดลายละเอียดข้อมูลของภาพที่เหมือนกันได้ และได้ประสิทธิภาพที่สูงออกมา
\par
โดยเริ่มต้นของการทำ การระบุตัวตนของบุคคล จะแบ่งภาพออกเป็นทั้งหมด 8 ส่วนและใช่คุณลักษณะของภาพมาทำ normalize ซึ่งจะช่วยในการลดความซ้ำซ้อนของข้อมูล ต่อมาข้อมูลที่ทำการ normalize มาใช้เปรียบเทียบความแตกต่างของคุณลักษณะของรูป หลังจากนั้นหาค่าเฉลี่ยของความแตกต่างออกมา ถ้าค่าที่ออกมาใกล้เคียง 0 จะพูดได้ว่าบุคคลในรูปนั้นเป็นบุคคลเดียวกัน