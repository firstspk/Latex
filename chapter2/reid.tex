ระบบระบุตัวตนของบุคคล คือการระบุตัวตนของบุคคลภายในวิดีโอหรือระหว่างรูป 2 รูป สามารถนำมาประยุกต์ใช้ในด้านของการรักษาความปลอดภัย 
การตามหาบุคคล หรือการตรวจสอบการกระทำของบุคคลนั้นในวิดีโอได้ ซึ่งการระบุตัวตนของบุคคลนั้นเป็นปัญหาที่ท้าทาย 
เนื่องจากคุณลักษณะทั่วไปของบุคคลในรูปไม่เพียงพอต่อการระบุบุคคลภายในรูปว่าเป็นบุคคลคนเดียวกันได้ 
ซึ่งวิธีการที่ใช้สำหรับการระบุตัวตนของบุคคลเรียกว่า Dynamically Matching Local Information (DMLI) ที่สามารถจัดแนวลายละเอียดข้อมูลของรูป และเพิ่มประสิทธิรูปให้สูงขึ้น

การทำงานของระบบระบุตัวตนของบุคคลจะเริ่มจากการแบ่งรูปออกเป็น 8 ส่วนและนำคุณลักษณะของรูปมาผ่านกระบวนการ normalization เพื่อลดความซ้ำซ้อนของข้อมูล 
แล้วนำมาเปรียบเทียบความแตกต่างของคุณลักษณะของรูป หลังจากนั้นหาค่าเฉลี่ยของความแตกต่างออกมา ถ้าค่าที่ออกมาใกล้เคียงกับ 0 จะหมายถึงบุคคลในรูปทั้งสองเป็นบุคคลเดียวกัน