ในการสร้างโมเดลปัญญาประดิษฐ์นั้นการใช้จำนวนชั้น (layer) เยอะนั้นจะทำให้ได้คุณลักษณะของข้อมูลที่ออกมาเยอะตามไปด้วย แต่การที่คุณลักษณะของข้อมูลเยอะไม่ได้หมายความว่าโมเดลปัญญาประดิษฐ์จะให้ประสิทธิภาพที่ดีเสมอไป ซึ่งสามารถแก้ปัญหานี้ได้โดยใช้ Residual Network (ResNet) ที่เป็น Convolution Neuron Network (CNN) ประเภทหนึ่งที่ ที่ส่วนใหญ่จะนำมาใช้กับข้อมูลที่เป็นรูปภาพ เช่น การจดจำวัตถุ เป็นต้น โดย ResNet นี้จะสามารถทำการข้ามชั้นของ CNN ที่ไม่จำเป็นได้ โดยในชั้นที่ไม่จำเป็นจะมีการปรับ weight ให้เข้าใกล้ 0 ในขณะที่ train ข้อมูล การข้ามชั้น CNN ที่ไม่จำเป็นจะช่วยลดเวลาที่ใช้ในการ train และทำให้ประสิทธิภาพของโมเดลปัญญาประดิษฐ์ดีขึ้น