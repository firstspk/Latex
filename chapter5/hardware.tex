การออกแบบโครงสร้างหุ่นยนต์ฮิวมานอยด์ UTHAI มุ่งเน้น 2 ส่วนเป็นหลักคือ
\vspace{-3mm}
\begin{enumerate}[label=\arabic*, leftmargin=1.5cm]
	\setlength\itemsep{-0.25em}
	\item สามารถสร้างขึ้นได้ง่าย
	\item น้ำหนักเบา
\end{enumerate}

ผู้วิจัยจึงเลือกที่จะประยุกต์ใช้เทคนิคการพิมพ์ขึ้นรูปสามมิติด้วยเครื่องพิมพ์สามมิติ ในการขึ้นรูปข้อต่อส่วนต่างๆของหุ่นยนต์ฮิวมานอยด์อุทัย
และก้านต่อได้เลือกใช้วัสดุเป็นคาร์บอนไฟเบอร์ ซึ่งเป็นวัสดุที่มีคุณสมบัติคือ เบา และแข็งแรง เมื่อเทียบกับวัสดุอื่นๆ ทำให้หุ่นยนต์ฮิวมานอยด์อุทัยมีน้ำหนักเบา
การเชื่อมต่อระหว่างข้อต่อ กับก้านต่อคาร์บอนไฟเบอร์ ผู้วัจัยใช้วิธีการบีบเพื่อสร้างแรงเสียดทานในการยึดติด
เนื่องจากการเจาะท่อคาร์บอนไฟเบอร์จะทำให้ใยไฟเบอร์ขาด ซึ่งส่งผลต่อความแข็งแรงของท่อคาร์บอนไฟเบอร์เป็นอย่างมาก
อีกทั้งเมื่อมีการเคลื่อนไหวและรับแรงในแนวต่างๆจะทำให้รูที่เจาะขยายและคลอนได้ส่งผลต่อความแม่นยำโดยรวมของหุ่นยนต์
แต่การยึดติดด้วยวิธีการบีบกับท่อคาร์บอนไฟเบอร์นั้นมีปัญหาเกิดขึ้นคือ มีโอกาสที่จะประกอบโครงสร้างของหุ่นยนต์ไม่ตรงเพราะอาจเกิดการหมุนตามแนวยาวของชิ้นส่วน
และหากใช้งานต่อเนื่องจะทำให้เกิดการหมุนเลื่อนตามแนวยาวของท่อได้ แต่มีข้อดีคือ เมื่อมีการบิดตามแนวแกนของท่อคาร์บอนเกิดขึ้นแทนที่ชิ้นส่วนจะเกิดการแตกหักเนื่องจาก 
แรกบิดนั้น ตรงส่วนที่ทำการยึดด้วยการบีบอัดนั้นจะเป็นตัวกลางรับแรงบิดแทน ส่งผลให้การยึดมีการบิดเปลี่ยนรูปตามแนวยาวของท่อ แต่ก็สามารถบิดกลับมาให้คงรูปเดิมได้

ส่วนของการเดินของหุ่นยนต์นั้นได้พบว่าเมื่อหุ่นยนต์มีการเดินจริงจะมี backlash เกิดขึ้นในส่วนของตัวมอเตอร์และชุดเฟือง และการเปลี่ยนรูปร่าง
ของข้อต่อที่เชื่อมกับมอเตอร์ ซึ่งทำให้การทดสอบเดินนั้นเป็นไปได้อย่างยากลำบาก โดยจะต้องเผื่อค่าเพื่อให้หุ่นยนต์เข้าที่ตามท่าทางที่กำหนด  

\subsection*{ข้อเสนอแนะ}

การพัฒนาต่อยอดควรปรับปรุงในส่วนการเชื่อมต่อเข้าด้วยกันระหว่างมอเตอร์และก้านต่อ ซึ่งอาจจะเพิ่มด้วยวิธีเพิ่มรอยบากเพื่อให้ศูนย์ของมอเตอร์ตรงกัน
และการประกอบโครงสร้างควรใช้วัสดุแผ่นยางบางมาคั่นกลางระหว่างหน้าสัมผัสที่ท่อคาร์บอนไฟเบอร์ซึ่งแผ่นยางจะสัมผัสกับชิ้นส่วนที่พิมพ์จากเครื่องพิมพ์สามมิติทำให้มีแรงยึดเกาะ
เพิ่มมากขึ้น

ส่วนของมอเตอร์ เนื่องจากว่าหุ่นยนต์ที่ออกแบบนี้มีการออกแบบให้มีความสูง 1 เมตร ซึ่งทำให้ส่วนขาของหุ่นยนต์มีความยาวที่มาก เป็นผลทำให้
สามารถเกิดโมเมนต์เมื่อกระทำการเดิน จึงส่งผลทำให้หุ่นยนต์เกิด backlash ที่ตัวของมอเตอร์เองเพราะเนื่องจากตัวของมอเตอร์ไม่สามารถควบคุม
ตำแหน่งของตนให้อยู่ในตำแหน่งที่สั่งการไปได้ ในการแก้ไขนี้สามารถแก้ไขได้โดย การลดขนาดความยาวของขาลงเพื่อให้เกิดโมเมนต์ที่มอเตอร์ส่วนสะโพกที่น้อยกว่าเดิม
หรือเปลี่ยนมอเตอร์ให้สามารถรับแรงบิดที่สูงกว่าตัวปัจจุบัน เพราะสามารถคงสภาพมุมที่สั่งการได้อย่างแม่นยำและเกิด backlash น้อยกว่าตัวปัจจุบันได้มาก
และสามารถแก้ไขเรื่องอุณหภูิที่เกิดขึ้นเมื่อมีการใช้งานเป็นเวลานาน ซึ่งตัวมอเตอร์ปัจจุบันเมื่อมีการใช้งานสักระยะหนึ่ง มอเตอร์จะเกิดการอ่อนแรง เนื่องจากความร้อน

