\subsubsection{การสร้างโมเดลปัญญาประดิษฐ์ด้วยโครงสร้าง I3D}
โมเดลปัญญาประดิษฐ์ I3D ที่ถูกสร้างด้วยชุดข้อมูลแบบ optical flow มีประสิทธิภาพที่สูงเนื่องจากข้อมูลที่มีเพียงข้อมูลของการเคลื่อนไหวในชุดของเฟรมเท่านั้น 
แต่ประสิทธิภาพจะต่ำลงเมื่อใช้กับการกระทำที่มีการเคลื่อนไหวเพียงเล็กน้อยจนแทบจะไม่เกิดการเคลื่อนไหว เช่น การนอนหรือการยืน ในขณะที่ใช้ชุดข้อมูลแบบภาพสีปกติ (RGB) 
นั้นจะมีประสิทธิภาพโดยรวมต่ำกว่าโมเดลที่สร้างด้วยชุดข้อมูลแบบ optical flow แต่กลับมีประสิทธิภาพในการจำแนก การนอน และการยืน ได้ดีกว่าแบบ optical flow
จึงสามารถกล่าวได้ว่า โมเดล I3D แบบภาพสีปกตินั้นสามารถใช้งานได้ดีในการกระทำที่มีการเคลื่อนไหวเพียงเล็กน้อยหรือไม่เกิดการเคลื่อนไหว ในขณะที่แบบ optical flow 
สามารถใช้งานได้อย่างมีประสิทธิภาพบนการกระทำที่มีการเคลื่อนไหวค่อนข้างชัดเจน และการใช้โมเดลทั้งสองแบบในการทำนายผลร่วมกันนั้นจะทำให้ประสิทธิภาพในการจำแนกการกระทำมากขึ้น
